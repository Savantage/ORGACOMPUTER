\section{Arquitectura del conjunto de Instrucciones}

\subsection{Definiciones}
\begin{itemize}
\item Arquitectura de computadoras: Son las características computacionales visibles al programador, es decir, los atributos que tienen impacto directo en la ejecución lógica de un programa.
\item Organización de computadoras: Implementación de la arquitectura (microarquitectura). Define las unidades operativas y sus interconexiones (señales de control, interfaces entre el CPU y los periféricos, tecnología de memoria, trayecto de datos, etc.)
\end{itemize}


\subsection{Modelo de capas}

\begin{center}
\begin{tabular}{ |l|l| } \hline
  \multicolumn{2}{|c|}{Software} \\ \hline
  Level 6 & Problem Oriented Language Level (Translation - Compiler) \\
  Level 5 & Assembly Language Level (Translation - Assembler)\\
  Level 4 & Operating System Machine Level (Partial Interpretation)\\
  Level 3 & Instruction Set Architecture Level (Interpretation/Direct Execution) \\ \hline
  \multicolumn{2}{|c||}{Hardware} \\ \hline
  Level 2 & Micro-Architecture Level (registers)\\
  Level 1 & Digital logic level (gates) \\
  Level 0 & Device Level (individual transistors)\\ \hline
\end{tabular}
\end{center}

\subsection{ISA - Instruction Set Architecture}

\subsubsection{Arquitectura de computadoras}
\begin{itemize}
\item Repertorio de instrucciones
\item  Especificacion de su operacion
\item  Registros
\item  Tipos de datos
\item  Modos de direccionamiento
\item  Formato de instrucciones
\item  Memoria
	\begin{itemize}
	\item  Word size
	\item  Big/Little  ENdian
	\item  Direccionamiento
	\item   Espacio de direcciones (address space)
	\end{itemize}
\end{itemize}

\subsubsection{Organizacion de computadoras}
\begin{itemize}
\item Diferentes implementaciones de una misma arquitectura
	\begin{itemize}
	\item  Costos
	\item  Velocidad de procesamiento
	\item  Consumo de energía
	\end{itemize}
\item Microarquitectura
	\begin{itemize}
	\item  Cableada (hardware - latches, contadores, decodificadores, etc.)
	\item  Microprogramada (software - microprograma)
	\end{itemize}
\end{itemize}

\subsubsection{Familia de computadoras}
\begin{itemize}
\item Misma arquitectura base, distintas organizaciones (implementaciones)
\item Modelos con prestaciones y precios diferentes pero compatibles entre si
\end{itemize}

\paragraph{Ejemplos}\mbox{}\\\\
\begin{itemize}
\item Intel 80x86
\item IBM Mainframe (360/370/390/zArch)
\item PowerPC
\item Sparc
\item Arm
\end{itemize}

\subsubsection{Clasificación de computadoras según su poder de cálculo}

\paragraph{Supercomputadoras}\mbox{}\\\\%%
\begin{itemize}
\item extremadamente rápidas
\item manejan volúmenes de datos enormes
\item poseen miles de cpus
\item usos específicos (aplicaciones científicas, simulaciones, campo militar)
\end{itemize}

\subparagraph{Ejemplos}\mbox{}\\\\
\begin{itemize}
\item Summit USA
\item Sierra USA
\item Sunway TaihuLight China
\item DeepBlue IBM USA
\end{itemize}

\paragraph{Macrocomputadoras o Mainframes}\mbox{}\\\\%%
\begin{itemize}
\item Muy rápidas
\item Manejan volúmenesde datos muy grandes
\item Poseen cientos de CPU
\item Muy alta disponibilidad
\item Usos comerciales y científicos
	\begin{itemize}
	\item Sistemas de gestión bancarios
	\item Telecomunicaciones
	\item Instituciones gubernamentales
	\end{itemize}
\end{itemize}

\subparagraph{Ejemplos}\mbox{}\\\\
\begin{itemize}
\item IBM Mainframe
\end{itemize}

\paragraph{Minicomputadoras o servidores middle range}\mbox{}\\\\%%
\begin{itemize}
\item Rápidas
\item Manejan volúmenesde datos muy grandes
\item Poseen decenas de CPUS
\item Usos comerciales
	\begin{itemize}
	\item Empresas medianas y grandes
	\item Varios equipos en una misma empresa
	\end{itemize}
\end{itemize}

\subparagraph{Ejemplos}\mbox{}\\\\
\begin{itemize}
\item IBM RS/6000
\item SUn UltraSparc
\item HP-NonStop (Itanium)
\end{itemize}

\paragraph{Minicomputadoras / PC}\mbox{}\\\\%%
\begin{itemize}
\item Uso individual o redes pequeñas a medianas
\item Manejan volúmenes de datos no muy grandes
\item Poseen uno o varios CPU
\item Uso hogareño, educativo, comercial, recreativo
	\begin{itemize}
	\item estaciones de trabajo en empresas
	\item Computadora en el hogar
	\item Negocios/Colegios
	\item Consolas de videojuego
	\end{itemize}
\end{itemize}

\subparagraph{Ejemplos}\mbox{}\\\\
\begin{itemize}
\item IBM PC compatible
\item Apple Macintosh
\item Video consolas
\end{itemize}

\paragraph{Computadoras portátiles / notebooks / netbooks}\mbox{}\\\\%%
\begin{itemize}
\item Uso individual portátil
\item Manejan volúmenes de datos no muy grandes
\item Poseen uno o varios CPU
\item Uso hogareño, educativo, comercial, recreativo
	\begin{itemize}
	\item Estaciones de trabajo
	\item Computadora en el hogar
	\item Negocios/Colegios
	\item Consolas de videojuego
	\end{itemize}
\end{itemize}

\subparagraph{Ejemplos}\mbox{}\\\\
\begin{itemize}
\item Sony
\item Toshiba
\item HP
\item Dell
\end{itemize}

\paragraph{Computadoras de mano}\mbox{}\\\\%%
\begin{itemize}
\item Uso individual portátil acotado
\item Manejan volúmenes de datos pequeños
\item Poseen uno o varios CPU
\item Uso hogareño, educativo, comercial, recreativo
	\begin{itemize}
	\item Acopio de datos en vía pública
	\item Información personal
	\item Visualización de contenidos
	\item Consolas de videojuego
	\end{itemize}
\end{itemize}

\subparagraph{Ejemplos}\mbox{}\\\\
\begin{itemize}
\item Smartphones
\item Tablets
\item Dispositivos Usables (Samsun gear 2 - reloj inteligente)
\end{itemize}
