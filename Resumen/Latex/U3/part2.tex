\section{Arquitectura Harvard}
/*Foto arquitecutra */	
\subsection{Highlights}
\begin{itemize}
\item Las instrucciones y los datos se almacenan en memorias diferentes
\item Hay dos conexiones entre la uniad de control de la CPU y cada sistema de memoria.
\item Las instrucciones se pueden cargar al mismo tiempo que los datos (instruction fetch y data access en paralelo por distintos buses)
\item Se manejan distintos espacios de direcciones para instrucciones y datos lo que idifculta la programación
\item Implementado en algunos microcontroladores PIC y en procesadores de señales digitles (DSP) 
\item Usado en los DSP para streaming de datos
	\begin{itemize}
	\item Mayor ancho de banda de memoria
	\item Ancho de banda más predecible
	\end{itemize}
\end{itemize}


\subsection{Instruction types}
\subsubsection{Data handling and memory operations}
load, store , move
\begin{itemize}
	\item Set a register to a fixed constant value.
	\item Copy data from a memory location to a register, or vice versa (a machine instruction is often called move; however, the term is misleading). Used to store the contents of a register, the result of a computation, or to retrieve stored data to perform a computation on it later. Often called load and store operations.
	\item Read and write data from hardware devices.
\end{itemize}

\subsubsection{Arithmetic and logic operations}
Aritméticas y lógicas add , subtract , multiply , divide (BPF c/s, Decimal, BPFlot and, or , xor
\begin{itemize}
	\item Add, subtract, multiply, or divide the values of two registers, placing the result in a register, possibly setting one or more condition codes in a status register.
	\item Increment, decrement in some ISAs, saving operand fetch in trivial cases.
	\item Perform bitwise operations, e.g., taking the conjunction and disjunction of corresponding bits in a pair of registers, taking the negation of each bit in a register.
	\item Compare two values in registers (for example, to see if one is less, or if they are equal).
	\item Floating-point instructions for arithmetic on floating-point numbers.
\end{itemize}

\subsubsection{Control flow operations}
branch , jump , compare, call , return
\begin{itemize}
	\item Branch to another location in the program and execute instructions there.
	\item Conditionally branch to another location if a certain condition holds.
	\item Indirectly branch to another location.
	\item Call another block of code, while saving the location of the next instruction as a point to return to.
\end{itemize}

\subsubsection{Coprocessor instructions}
\begin{itemize}
	\item Load/store data to and from a coprocessor, or exchanging with CPU registers.
	\item Perform coprocessor operations.
\end{itemize}

\subsubsection{Complex instructions}
\begin{itemize}
	\item Transferring multiple registers to or from memory (especially the stack) at once
	\item Moving large blocks of memory (e.g. string copy or DMA transfer)
	\item complicated integer and floating-point arithmetic (e.g. square root, or transcendental functions such as logarithm, sine, cosine, etc.)
	\item SIMD instructions, a single instruction performing an operation on many homogeneous values in parallel, possibly in dedicated SIMD registers
	\item performing an atomic test-and-set instruction or other read-modify-write atomic instruction
	\item instructions that perform ALU operations with an operand from memory rather than a register
\end{itemize}


\subsection{ISA - Instruction Set Architecture}
An instruction set architecture (ISA) is an abstract model of a computer. An instruction set architecture is distinguished from a microarchitecture, which is the set of processor design techniques used, in a particular processor, to implement the instruction set. Processors with different microarchitectures can share a common instruction set. For example, the Intel Pentium and the Advanced Micro Devices Athlon implement nearly identical versions of the x86 instruction set, but have radically different internal designs.

\subsubsection{Machine instructions characteristics}
The operation of the processor is determined by the instructions it executes, referred to as machine instructions or computer instructions. The collection of different instructions that the processor can execute is referred to as the processor’s instruction set.

\subsubsection{Repertorio de instrucciones}
How many and which operations to provide, and how complex operations should be.

\subsubsection{Especificación de su operación}
\begin{itemize}
\item Operation code: Specifies the operation to be performed (e.g., ADD, I/O). The operation is specified by a binary code, known as the operation code, or opcode.
\item  Source operand reference: The operation may involve one or more source operands, that is, operands that are inputs for the operation.
\item  Result operand reference: The operation may produce a result.
\item  Next instruction reference: This tells the processor where to fetch the next
instruction after the execution of this instruction is complete.
\end{itemize}

Source and result operands can be in one of four areas:
\subsubsection{Especificación de su operación}
\begin{itemize}
\item Main or virtual memory: As with next instruction references, the main or virtual memory address must be supplied.
\item  Processor register: With rare exceptions, a processor contains one or more registers that may be referenced by machine instructions. If only one register exists, reference to it may be implicit. If more than one register exists, then each register is assigned a unique name or number, and the instruction must contain the number of the desired register.
\item  Immediate: The value of the operand is contained in a field in the instruction being executed.
\item  I/O device: The instruction must specify the I/O module and device for the operation. If memory-mapped I/O is used, this is just another main or virtual memory address.
\end{itemize}

\subsubsection{Clasificación según la ubicación de los operandos}
\begin{itemize}
\item Stack (‘60s a ‘70s
\item Acumulador (antes de ‘60s)
\item Registro Memoria (‘70s hasta ahora)
\item Registro Registro (Load/Store) (‘60s hasta ahora)
\item Memoria Memoria (‘70s a ‘80s)
\end{itemize}

\subsubsection{Registros}
Number of processor registers that can be referenced by instructions, and their use


\subsubsection{Tipos de datos}
\paragraph{Numéricos}\mbox{}\\\\%%
\begin{itemize}
\item BPF s/s
\item BPF c/s
\item BPFlotante (IEEE 754 o
\item BCD (decimales)
\end{itemize}
\paragraph{Caracteres}\mbox{}\\\\%%
\begin{itemize}
\item ASCII
\item EBCDIC
\item Unicode
\end{itemize}
\paragraph{Datos lógicos}\mbox{}\\\\%%
\paragraph{Direcciones}\mbox{}\\\\%%


\subsubsection{Instruction Sets: Addressing Modes}
/*458 William Stallings 10th edition*/

\paragraph{Immediate}\mbox{}\\\\%%
The simplest form of addressing is immediate addressing, in which the operand value is present in the instruction. The advantage of immediate addressing is that no memory reference other than the instruction fetch is required to obtain the operand, thus saving one memory or cache cycle in the instruction cycle. The disadvantage is that the size of the number is restricted to the size of the address field, which, in most instruction sets,
is small compared with the word length.
Operand = A

\paragraph{Direct}\mbox{}\\\\%%
A very simple form of addressing is direct addressing, in which the address field contains
the effective address of the operand.
The technique was common in earlier generations of computers but is not common
on contemporary architectures. It requires only one memory reference and no
special calculation. The obvious limitation is that it provides only a limited address
space.
EA = A
\paragraph{Indirect}\mbox{}\\\\%%
With direct addressing, the length of the address field is usually less than the word
length, thus limiting the address range. One solution is to have the address field refer
to the address of a word in memory, which in turn contains a full-
length address of the operand. This is known as indirect addressing:
EA = (A)
As defined earlier, the parentheses are to be interpreted as meaning contents
of. The obvious advantage of this approach is that for a word length of N, an address
space of 2N is now available. The disadvantage is that instruction execution requires
two memory references to fetch the operand: one to get its address and a second to
get its value. Although the number of words that can be addressed is now equal to 2N, the
number of different effective addresses that may be referenced at any one time is
limited to 2K, where K is the length of the address field. Typically, this is not a burdensome
restriction, and it can be an asset.
\paragraph{Register}\mbox{}\\\\%%
Register addressing is similar to direct addressing. The only difference is that the
address field refers to a register rather than a main memory address:
EA = R
To clarify, if the contents of a register address field in an instruction is 5, then
register R5 is the intended address, and the operand value is contained in R5. Typically,
an address field that references registers will have from 3 to 5 bits, so that a
total of from 8 to 32 general-
purpose
registers can be referenced.
The advantages of register addressing are that (1) only a small address field
is needed in the instruction, and (2) no time-
consuming
memory references are
required. As was discussed in Chapter 4, the memory access time for a register
internal to the processor is much less than that for a main memory address. The disadvantage
of register addressing is that the address space is very limited.
\paragraph{Register indirect}\mbox{}\\\\%%
Just as register addressing is analogous to direct addressing, register indirect addressing
is analogous to indirect addressing. In both cases, the only difference is whether
the address field refers to a memory location or a register. Thus, for register indirect
address,
EA = (R)
The advantages and limitations of register indirect addressing are basically the same
as for indirect addressing. In both cases, the address space limitation (limited range
of addresses) of the address field is overcome by having that field refer to a word-
length
location containing an address. In addition, register indirect addressing uses
one less memory reference than indirect addressing.
\paragraph{Displacement}\mbox{}\\\\%%
A very powerful mode of addressing combines the capabilities of direct addressing
and register indirect addressing. It is known by a variety of names depending on
the context of its use, but the basic mechanism is the same. We will refer to this as
displacement addressing:
EA = A + (R)
Displacement addressing requires that the instruction have two address fields, at
least one of which is explicit. The value contained in one address field (value = A)
is used directly. The other address field, or an implicit reference based on opcode,
refers to a register whose contents are added to A to produce the effective address.
We will describe three of the most common uses of displacement addressing:
\subparagraph{Relative addressing}\mbox{}\\\\%%
For relative addressing, also called PC-relative addressing,
	the implicitly referenced register is the program counter (PC). That is, the next
	instruction address is added to the address field to produce the EA. Typically, the
	address field is treated as a twos complement number for this operation. Thus, the
	effective address is a displacement relative to the address of the instruction.
	Relative addressing exploits the concept of locality. If most memory references are relatively near to the instruction being executed, then the use of relative addressing saves address bits in the instruction.
\subparagraph{Base-register addressing}\mbox{}\\\\%%
For base-register addressing, the interpretation is
	the following: The referenced register contains a main memory address, and the
	address field contains a displacement (usually an unsigned integer representation)
	from that address. The register reference may be explicit or implicit.
	Base-register addressing also exploits the locality of memory references.
\subparagraph{Indexing}\mbox{}\\\\%%
For indexing, the interpretation is typically the following: The address
	field references a main memory address, and the referenced register contains a
	positive displacement from that address. Note that this usage is just the opposite
	of the interpretation for base-register addressing. Of course, it is more than just
	a matter of user interpretation. Because the address field is considered to be a
	memory address in indexing, it generally contains more bits than an address field
	in a comparable base-register instruction. Also, we will see that there are some
	refinements to indexing that would not be as useful in the base-
	register context.
	Nevertheless, the method of calculating the EA is the same for both base-
	register addressing and indexing, and in both cases the register reference is sometimes
	explicit and sometimes implicit (for different processor types).
	An important use of indexing is to provide an efficient mechanism for performing
	iterative operations. Consider, for example, a list of numbers stored starting
	at location A. Suppose that we would like to add 1 to each element on the list.
	We need to fetch each value, add 1 to it, and store it back. The sequence of effective
	addresses that we need is A, A + 1, A + 2, . . . , up to the last location on the list.
	With indexing, this is easily done. The value A is stored in the instruction’s address
	field, and the chosen register, called an index register, is initialized to 0. After each
	operation, the index register is incremented by 1
\paragraph{Stack}\mbox{}\\\\%%
The final addressing mode that we consider is stack addressing. As defined in
Appendix I, a stack is a linear array of locations. It is sometimes referred to as a
pushdown list or last-in-first-out queue. The stack is a reserved block of locations.
Items are appended to the top of the stack so that, at any given time, the block is partially filled. Associated with the stack is a pointer whose value is the address of the
top of the stack. Alternatively, the top two elements of the stack may be in processor
registers, in which case the stack pointer references the third element of the stack.
The stack pointer is maintained in a register. Thus, references to stack locations in
memory are in fact register indirect addresses.
The stack mode of addressing is a form of implied addressing. The machine
instructions need not include a memory reference but implicitly operate on the top
of the stack.


\subsubsection{Instruction Sets: Addressing Modes}
Definición:“Define el despliegue de los bits que componen la instrucción”
\paragraph{Components:}\mbox{}\\\\%%
\begin{itemize}
\item Opcode
\item 0 a n operandos
\item Modo de direccionamiento de cada operando
\item Flags
\end{itemize}

/*Formato ARM*/ U3 Pág 11
/*Formato x86*/ U3 Pág 12

\subsubsection{Clasificación de la ISA según el número de direcciones.}
\paragraph{3 addresses}\mbox{}\\\\%%
\begin{itemize}
\item Operand 1, Operand 2, Result
\item e.g. a=b+c
\end{itemize}
\paragraph{2 address}\mbox{}\\\\%%
\begin{itemize}
\item One address doubles as operand and result
\item eg a . . = a+c
\end{itemize}
\paragraph{1 address}\mbox{}\\\\%%
\begin{itemize}
\item Implicit second address (accumulator)
\end{itemize}
\paragraph{0 address}\mbox{}\\\\%%
\begin{itemize}
\item All addresses are implicitly defined
\item Stack based computer
\end{itemize}

\subsection{Memoria}
Word size: The “natural” unit of organization of memory. The size of a word is typically equal to the number of bits used to represent an integer and to the instruction length
\subsubsection{Big / Little Endian}
/* Libro página 452 10th edition*/
\subsubsection{Direccionamiento}
Addressing: The mode or modes by which the address of an operand is specified.
\subsubsection{Espacio de direcciones ( address space)}
\begin{itemize}
\item Memory: The memory space includes system main memory. It also includes PCIe I/O devices. Certain ranges of memory addresses map into I/O devices.
\item I/O: This address space is used for legacy PCI devices, with reserved memory address ranges used to address legacy I/O devices.
\item Configuration: This address space enables the TL to read/write configuration registers associated with I/O devices.
\item Message: This address space is for control
\end{itemize}


