\subsection{Memoria}

\subsubsection{Definicion}
Computer memory is any physical device capable of storing information temporarily like RAM (random access memory), or permanently, like ROM (read-only memory). Memory devices utilize integrated circuits and are used by operating systems, software, and hardware.


\subsubsection{Formado por elementos con distintas cualidades}
	\begin{itemize}
	\item Tecnología
	\item Organización
	\item Performance
	\item Costo
	\end{itemize}
\subsubsection{Jerarquía de subsistemas de memoria}
	\begin{itemize}
	\item Internos al sistema (accedidos directamente por el procesador)
		\item Registros
		\item Memoria interna para unidad de control
		\item Memoria Cache
	\item Externo al sistema (accedidos por el procesador a través de un módulo de E/S)
		\item Dispositivos de almacenamiento perifericos,  discos, cintas, etc.
	\end{itemize}
\subsubsection{Puntos importantes}
	\begin{itemize}
	\item Capacidad
	\item Tiempo de acceso 
	\item Costo
	\end{itemize}

/*Grafico pág 4*/

\subsubsection{A medida que se baja la piramide}
	\begin{itemize}
	\item Costo por bit decreciente
	\item Capacidad decreciente
	\item Tiempo de acceso creciente
	\item Frecuencia de acceso de la memoria por parte del procesador decreciente
	\end{itemize}

\subsubsection{Caracteristicas}
	\begin{itemize}
	\item Capacidad
		\begin{itemize}
		\item bytes/palabras (memoria interna)
		\item bytes (memoria externa)
		\end{itemize}
	\item Unidad de transferencia
		\item número de líneas eléctricas del módulo de memoria, típicamente el tamaño de palabra o 64, 128, o 256 bits (memoria interna)
		\item bloques (memoria externa)
	\end{itemize}
	
\subsubsection{Metodos de acceso de unidades de datos}
	\begin{itemize}
	\item Acceso secuencial
		\begin{itemize}
		\item Unidades de datos: registros 
		\item Acceso lineal en secuencia
		\item Se deben pasar y descartar todos los registros intermedios antes de acceder al registro deseado
		\item Tiempo de acceso variable
		\item ej cintas magnéticas
		\end{itemize}
	\item Acceso directo
		\begin{itemize}
		\item Dirección única para bloques o registros basada en su posición física
		\item Tiempo de acceso variable
		\item ej discos magnéticos
		\end{itemize}
	\item Acceso aleatorio
		\begin{itemize}
		\item cada posición direccionable de memoria tiene un mecanismo de direccionamiento cableado físicamente
		\item tiempo de acceso constante, independiente de la secuencia de accesos anteriores
		\item ej memoria principal y algunas memorias cache
		\end{itemize}
	\item Acceso asociativo
		\begin{itemize}
		\item Tipo de acceso aleatorio por comparación de patrón de bits
		\item La palabra se busca por una porción de su contenido en vez de por su dirección
		\item Cada posición de memoria tiene un mecanismo de direccionamiento propio
		\item Tiempo de acceso constante, independiente de la secuencia de accesos anteriores o su ubicación
		\item Ej. memorias cache
		\end{itemize}
	\end{itemize}
	
\subsubsection{Parametros de performance}
	\begin{itemize}
	\item Tiempo de acceso (latencia)
		\begin{itemize}
		\item Memorias de acceso aleatorio: tiempo necesario para hacer una operación de lectura o escritura
		\item memorias sin acceso aleatorio: tiempo necesario para posicionar el mecanismo de lectura/escritura en la posición deseada
		\end{itemize}
	\item Tiempo de ciclo de memoria
		\begin{itemize}
		\item memorias de acceso aleatorio: tiempo de acceso mas el tiempo adicional necesario para que una nueva operacion pueda comenzar
		\end{itemize}
	\item Tasa de transferencia
		\begin{itemize}
		\item tasa con la cual los datos son transferidos dentro o fuera de la unidad de memoria
		\item memorias de acceso aleatorio: 1/tiempo de ciclo de memoria
		\item memorias sin acceso aleatorio: $t_n+T_a+n/R$
		\item donde
			\begin{itemize}
			\item $t_n$ = tiempo promedio para leer escribir n bits
			\item $T_a$ = tiempo promedio de acceso
			\item n = numero de bits
			\item R = tasa de transferencia, en bits por segundo
			\end{itemize}
		\end{itemize}
	\end{itemize}

\subsubsection{Tipos fisicos de memoria}
	\begin{itemize}
	\item Memorias semiconductoras (memoria principal y cache)
	\item Memorias de superficie magnética (discos y cintas)
	\item Memorias ópticas (medios ópticos)
	\end{itemize}
	
\subsubsection{Caracteristicas fisicas de memoria}
	\begin{itemize}
	\item Memorias volátiles: se pierde su contenido ante la falta de energía eléctrica (Ej. algunas memorias semiconductoras)
	\item Memorias no volátiles: no se necesita de energía eléctrica para mantener su contenido (Ej. memorias de superficie magnéticas y algunas memorias semiconductoras)
	\item Memorias de solo lectura: (ROM –ReadOnlyMemory) no se puede borrar su contenido (Ej. algunas memorias semiconductoras)
	\end{itemize}
	
\subsubsection{Principio de localidad de referencia}
Durante la ejecución de un programa, las referencias a memoria que hace el procesador tanto para instrucciones como datos tienden a estar agrupadas	(Ej. loops, subrutinas, tablas, vectores)


\subsubsection{Memoria cache}
	\begin{itemize}
	\item Memoria semiconductora más rápida (y costosa) que la principal
	\item Se ubica entre el procesador y la memoria principal
	\item Permite mejorar la performance general de acceso a memoria principal
	\item Contiene una copia de porciones de memoria principal
	\item como funciona
		\begin{itemize}
		\item CPU trata de leer una palabra de la memoria principal
		\item Se chequea primero si existe en la memoria cache.
			\begin{itemize}
			\item Si es así se la entrega al CPU
			\item Sino se lee un bloque de memoria principal (número fijo de palabras), se incorpora a la cache y la palabra buscada se entrega al CPU
			\end{itemize}
		\end{itemize}
	\end{itemize}
	Por el principio de localidad de referencia es probable que próximas palabras buscadas estén dentro del bloque de memoria subido a la cache
	
\subsubsection{Estructura Memoria principal}
	\begin{itemize}
	\item $2^n$ palabras direccionables(direccion única de n-bits para cada una)
	\item Bloques fijos de Kpalabras cada uno (M bloques)
	\end{itemize}
\subsubsection{Estructura Memoria Cache}
	\begin{itemize}
	\item mbloques llamados líneas
	\item Cada linea contiene:
		\begin{itemize}
		\item Kpalabras
		\item Tag(conjunto de bits para indicar qué bloque está almacenado, usualmente una porción de la dirección de memoria principal)
		\item Bits de control (Ej. bit para indicar si la línea se modificó desde la última vez que se cargó en la cache)
		\end{itemize}
	\end{itemize}