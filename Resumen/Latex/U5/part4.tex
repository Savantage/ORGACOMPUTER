\subsection{MODULO DE E/S }

\subsubsection{Que hace}
Conecta a los periféricos con la CPU y la memoria a través del bus del sistema o switch central y permite la comunicación entre ellos
\subsubsection{Para que sirve}
Oculta detalles de timing, formatos y electro mecánica de los dispositivos periféricos
\subsubsection{Por que existe?}
\begin{itemize}
\item Amplia variedad de periféricos con distintos métodos de operación
\item La tasa de transferencia de los periféricos es generalmente mucho más lenta que la de la memoria y procesador
\item Los periféricos usan distintos formatos de datos y tamaños de palabra
\end{itemize}

/*4pdf*/ 230 stalling

\subsubsection{Interface interna - bus del sistema}
\begin{itemize}
\item Datos
\item Direcciones
\item Control
\end{itemize}

\subsubsection{Interface externa - perifericos}
\begin{itemize}
\item Datos
\item Estado
\item Control
\end{itemize}


\subsubsection{Funciones}
\paragraph{Control and Timing}\mbox{}\\\\%%
ontrola flujo de tráfico entre CPU/Memoria y periféricos
\paragraph{Comunicación con el procesador}\mbox{}\\\\%%
Decodificación de comandos: The I/O module accepts commands from the processor, typically sent as signals on the control bus. For example, an I/O module for a disk drive might accept the following commands: READ SECTOR, WRITE SECTOR, SEEK track number, and SCAN record ID. The latter two commands each include a parameter that is sent on the data bus
\begin{itemize}
	\item Datos: Data are exchanged between the processor and the I/O module over the data bus.
	\item Información de estado: Because peripherals are so slow, it is important to know the	status of the I/O module. For example, if an I/O module is asked to send data to the processor (read), it may not be ready to do so because it is still working on the previous I/O command. This fact can be reported with a status signal.	Common status signals are BUSY and READY. There may also be signals to report various error conditions.
	\item Reconocimiento de direcciones: Just as each word of memory has an address, so does each I/O device. Thus, an I/O module must recognize one unique address for each peripheral it controls.
\end{itemize}
\paragraph{Comunicación con el dispositivo}\mbox{}\\\\%%
\begin{itemize}
\item Comandos
\item Información de estado
\item Datos
\end{itemize}
\paragraph{Buffering de datos}\mbox{}\\\\%%
\paragraph{Detección de errores}\mbox{}\\\\%%

\subsubsection{The control of the transfer of data from an external device to the processor might involve the following sequence of steps}
\begin{enumerate}
\item The processor interrogates the I/O module to check the status of the attached device.
\item The I/O module returns the device status.
\item If the device is operational and ready to transmit, the processor requests the transfer of data, by means of a command to the IO module.
\item  The I/O module obtains a unit of data (e.g., 8 or 16 bits) from the external device.
\item The data are transferred from the I/O module to the processor.
\end{enumerate}

/*Diagrama I/O pag 234 Will, 7pdf */

\subsubsection{Tecnicas para operaciones de E/S}
\begin{enumerate}
\item  E/S programada
\item  E/S manejada por interrupciones
\item  Acceso directo a memoria (DMA)
\end{enumerate}

/*pag 237*/

When large volumes of data are to be moved, a more efficient technique is
required: direct memory access (DMA).
DMA involves an additional module on the system bus. The DMA module
(Figure 7.12) is capable of mimicking the processor and, indeed, of taking over control
of the system from the processor. It needs to do this to transfer data to and from
memory over the system bus. For this purpose, the DMA module must use the bus
only when the processor does not need it, or it must force the processor to suspend
operation temporarily. The latter technique is more common and is referred to as
cycle stealing, because the DMA module in effect steals a bus cycle.
When the processor wishes to read or write a block of data, it issues a command
to the DMA module, by sending to the DMA module the following information:
\begin{enumerate}
\item  Whether a read or write is requested, using the read or write control line
between the processor and the DMA module.
\item  The address of the I/O device involved, communicated on the data lines.
\item  The starting location in memory to read from or write to, communicated on
the data lines and stored by the DMA module in its address register.
\item  The number of words to be read or written, again communicated via the data
lines and stored in the data count register.
\end{enumerate}
/*pag 249*/ 
The processor then continues with other work. It has delegated this I/O operation
to the DMA module. The DMA module transfers the entire block of data,
one word at a time, directly to or from memory, without going through the processor.
When the transfer is complete, the DMA module sends an interrupt signal to
the processor. Thus, the processor is involved only at the beginning and end of the
transfer

/*  250 Figure 7.13 DMA and Interrupt Breakpoints during an Instruction Cycle*/

/* 251 Figure 7.14 Alternative DMA Configurations*/

The DMA mechanism can be configured in a variety of ways. Some possibilities
are shown in Figure 7.14. In the first example, all modules share the same system
bus. The DMA module, acting as a surrogate processor, uses programmed I/O to
exchange data between memory and an I/O module through the DMA module. This
configuration, while it may be inexpensive, is clearly inefficient. As with processor-
controlled programmed I/O, each transfer of a word consumes two bus cycles
The number of required bus cycles can be cut substantially by integrating the
DMA and I/O functions. As Figure 7.14b indicates, this means that there is a path
between the DMA module and one or more I/O modules that does not include
the system bus. The DMA logic may actually be a part of an I/O module, or it may
be a separate module that controls one or more I/O modules. This concept can
be taken one step further by connecting I/O modules to the DMA module using
an I/O bus (Figure 7.14c). This reduces the number of I/O interfaces in the DMA
module to one and provides for an easily expandable configuration. In both of
these cases (Figures 7.14b and c), the system bus that the DMA module shares with
the processor and memory is used by the DMA module only to exchange data with
memory. The exchange of data between the DMA and I/O modules takes place off
the system bus.

\subsubsection{I /O Channels and Processors}
The Evolution of the I/O Function
As computer systems have evolved, there has been a pattern of increasing complexity
and sophistication of individual components. Nowhere is this more evident than
in the I/O function. We have already seen part of that evolution. The evolutionary
steps can be summarized as follows:
\begin{enumerate}
\item The CPU directly controls a peripheral device. This is seen in simple
microprocessor-controlled devices.
\item A controller or I/O module is added. The CPU uses programmed I/O without
interrupts. With this step, the CPU becomes somewhat divorced from the specific
details of external device interfaces.
\item  The same configuration as in step 2 is used, but now interrupts are employed.
The CPU need not spend time waiting for an I/O operation to be performed,
thus increasing efficiency.
\item The I/O module is given direct access to memory via DMA. It can now move
a block of data to or from memory without involving the CPU, except at the
beginning and end of the transfer.
\item  The I/O module is enhanced to become a processor in its own right, with a
specialized instruction set tailored for I/O. The CPU directs the I/O processor
to execute an I/O program in memory. The I/O processor fetches and executes
these instructions without CPU intervention. This allows the CPU to specify a
sequence of I/O activities and to be interrupted only when the entire sequence
has been performed.
\item The I/O module has a local memory of its own and is, in fact, a computer in its
own right. With this architecture, a large set of I/O devices can be controlled,
with minimal CPU involvement. A common use for such an architecture has
been to control communication with interactive terminals. The I/O processor
takes care of most of the tasks involved in controlling the terminals.
\end{enumerate}

\subsubsection{Characteristics of I/O Channels}
	The I/O channel represents an extension of the DMA concept. An I/O channel
	has the ability to execute I/O instructions, which gives it complete control over
	I/O operations. In a computer system with such devices, the CPU does not execute
	I/O instructions. Such instructions are stored in main memory to be executed by a
	special-purpose processor in the I/O channel itself. Thus, the CPU initiates an I/O
	transfer by instructing the I/O channel to execute a program in memory. The program
	will specify the device or devices, the area or areas of memory for storage,
	priority, and actions to be taken for certain error conditions. The I/O channel follows
	these instructions and controls the data transfer

	Two types of I/O channels are common, as illustrated in Figure 7.18. A
	selector channel controls multiple high-speed devices and, at any one time, is
	dedicated to the transfer of data with one of those devices. Thus, the I/O channel
	selects one device and effects the data transfer. Each device, or a small set of
	devices, is handled by a controller, or I/O module, that is much like the I/O modules
	we have been discussing. Thus, the I/O channel serves in place of the CPU
	in controlling these I/O controllers. A multiplexor channel can handle I/O with
	multiple devices at the same time. For low-speed devices, a byte multiplexor
	accepts or transmits characters as fast as possible to multiple devices. For example,
	the resultant character stream from three devices with different rates and individual
	streams A1A2A3A4 c, B1B2B3B4 c, and C1C2C3C4 c might be A1B1C1A2C2A3B2C3A4, and so on.
	

	/*263*/
