
\subsection{Interrupciones}
\subsubsection{Definicion}
Mecanismos por los cuales otros modulos (E/S y memoria) interrumpen el normal procesamiento del CPU
\subsubsection{¿Para que existen?}
Para mejorar la eficiencia de procesamiento de un computador
\subsubsection{Clases de interrupciones}
\begin{itemize}
	\item programa
	\item  reloj
	\item e/s
	\item fallas de hardware
\end{itemize}

\subsubsection{Ciclo de instruccion}
\begin{itemize}
	\item Fetch instruction
	\item  Decode instruction
	\item Fetch operand
	\item Execute instruction
	\item Store result
	\item ----------------> interrupt breakpoint
	\item process interrupt
\end{itemize}

\subsubsection{Transferencia de control al S.O. (Handler)}
/*4pdf*/
	
\subsubsection{Procesamiento de interrupciones }
/*5pdf*/

/*6pdf ejemplo*/
	
\subsubsection{Multiples interrupciones}


\paragraph{Deshabilitar interrupciones (secuencia)}\mbox{}\\\\%%
/*7pdf*/
\paragraph{Priorizar interrupciones (anidadas)}\mbox{}\\\\%%
/*8pdf*/
	
\paragraph{Múltiples interrupciones - ejemplo}\mbox{}\\\\%%
Tres dispositivos de E/S
\begin{itemize}
	\item Línea de comunicación (Prioridad 1)
	\item  Disco (Prioridad 2)
	\item Impresora (Prioridad 3)
\end{itemize}
Eventos
\begin{itemize}
	\item T=10 Interrupción de Impresora
	\item T=15 Interrupción de línea de comunicación
	\item T=20 Interrupción de disco
\end{itemize}

/*10pdf*/
