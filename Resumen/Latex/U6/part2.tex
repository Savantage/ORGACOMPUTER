\subsection{Medios Opticos}
In 1983, one of the most successful consumer products of all time was introduced:
the compact disk (CD) digital audio system. The CD is a nonerasable disk that can
store more than 60 minutes of audio information on one side. The huge commercial
success of the CD enabled the development of low-cost optical-disk storage
technology that has revolutionized computer data storage. A variety of optical-disk
systems have been introduced (Table 6.6). We briefly review each of these.

CD-ROM Both the audio CD and the CD-ROM (compact disk read-only
memory) share a similar technology. The main difference is that CD-ROM players
are more rugged and have error correction devices to ensure that data are properly
transferred from disk to computer. Both types of disk are made the same way. The
disk is formed from a resin, such as polycarbonate. Digitally recorded information
(either music or computer data) is imprinted as a series of microscopic pits on the
surface of the polycarbonate. This is done, first of all, with a finely focused, highintensity
laser to create a master disk. The master is used, in turn, to make a die to
stamp out copies onto polycarbonate. The pitted surface is then coated with a highly
reflective surface, usually aluminum or gold. This shiny surface is protected against
dust and scratches by a top coat of clear acrylic. Finally, a label can be silkscreened
onto the acrylic.
Information is retrieved from a CD or CD-ROM by a low-powered laser
housed in an optical-disk player, or drive unit. The laser shines through the clear
polycarbonate while a motor spins the disk past it (Figure 6.12). The intensity of
the reflected light of the laser changes as it encounters a pit. Specifically, if the laser
beam falls on a pit, which has a somewhat rough surface, the light scatters and a low
intensity is reflected back to the source. The areas between pits are called lands.
A land is a smooth surface, which reflects back at higher intensity. The change
between pits and lands is detected by a photosensor and converted into a digital
signal. The sensor tests the surface at regular intervals. The beginning or end of
a pit represents a 1; when no change in elevation occurs between intervals, a 0 is
recorded.
Recall that on a magnetic disk, information is recorded in concentric tracks.
With the simplest constant angular velocity (CAV) system, the number of bits per
track is constant. An increase in density is achieved with multiple zoned recording,
in which the surface is divided into a number of zones, with zones farther from the
center containing more bits than zones closer to the center. Although this technique
increases capacity, it is still not optimal.
To achieve greater capacity, CDs and CD-ROMs do not organize information
on concentric tracks. Instead, the disk contains a single spiral track, beginning near
the center and spiraling out to the outer edge of the disk. Sectors near the outside
of the disk are the same length as those near the inside. Thus, information is packed
evenly across the disk in segments of the same size and these are scanned at the
same rate by rotating the disk at a variable speed. The pits are then read by the laser
at a constant linear velocity (CLV). The disk rotates more slowly for accesses near
the outer edge than for those near the center. Thus, the capacity of a track and the
rotational delay both increase for positions nearer the outer edge of the disk. The
data capacity for a CD-ROM is about 680 MB.

\begin{itemize}
\item CD: Compact Disk. A nonerasable disk that stores digitized audio information. The standard system uses 12-cm disks and can record more than 60 minutes of uninterrupted playing time.
\item CD-ROM:  Compact Disk Read-Only Memory. A nonerasable disk used for storing computer data. The standard system uses 12-cm disks and can hold more than 650 Mbytes.
\item CD-R: CD Recordable. Similar to a CD-ROM. The user can write to the disk only once. 
\item CD-RW: CD Rewritable. Similar to a CD-ROM. The user can erase and rewrite to the disk multiple times.
\item DVD: Digital Versatile Disk. A technology for producing digitized, compressed representation of video information,
as well as large volumes of other digital data. Both 8 and 12 cm diameters are used, with a
double-sided capacity of up to 17 Gbytes. The basic DVD is read-only (DVD-ROM).
\item DVD-R: DVD Recordable. Similar to a DVD-ROM. The user can write to the disk only once. Only one-sided
disks can be used.
\item DVD-RW: DVD Rewritable. Similar to a DVD-ROM. The user can erase and rewrite to the disk multiple times.
Only one-sided disks can be used.
\item Blu-ray DVD: High-definition video disk. Provides considerably greater data storage density than DVD, using a 405-nm
(blue-violet) laser. A single layer on a single side can store 25 Gbytes.
\end{itemize}

/*211*/
/*212*/

Data on the CD-ROM are organized as a sequence of blocks. A typical block
format is shown in Figure 6.13. It consists of the following fields:
\begin{itemize}
\item Sync: The sync field identifies the beginning of a block. It consists of a byte of
all 0s, 10 bytes of all 1s, and a byte of all 0s.
\item Header: The header contains the block address and the mode byte. Mode
0 specifies a blank data field; mode 1 specifies the use of an error-correcting
code and 2048 bytes of data; mode 2 specifies 2336 bytes of user data with no
error-correcting code.
\item Data: User data.
\item Auxiliary: Additional user data in mode 2. In mode 1, this is a 288-byte errorcorrecting
code.
\end{itemize}
With the use of CLV, random access becomes more difficult. Locating a specific
address involves moving the head to the general area, adjusting the rotation
speed and reading the address, and then making minor adjustments to find and
access the specific sector. 
CD-ROM is appropriate for the distribution of large amounts of data to a
large number of users. Because of the expense of the initial writing process, it is not
appropriate for individualized applications. Compared with traditional magnetic
disks, the CD-ROM has two advantages:
\begin{itemize}
\item The optical disk together with the information stored on it can be mass replicated
inexpensively—unlike a magnetic disk. The database on a magnetic disk
has to be reproduced by copying one disk at a time using two disk drives.
\item The optical disk is removable, allowing the disk itself to be used for archival
storage. Most magnetic disks are nonremovable. The information on nonremovable
magnetic disks must first be copied to another storage medium
before the disk drive/disk can be used to store new information.
The disadvantages of CD-ROM are as follows:
\item It is read-only and cannot be updated.
\item It has an access time much longer than that of a magnetic disk drive, as much
as half a second.
\end{itemize} 
\paragraph{CD RECORDABLE}\mbox{}\\\\%%
To accommodate applications in which only one or a small
number of copies of a set of data is needed, the write-once read-many CD, known
as the CD recordable (CD-R), has been developed. For CD-R, a disk is prepared
in such a way that it can be subsequently written once with a laser beam of
modest -intensity. Thus, with a some what more expensive disk controller than for
CD-ROM, the customer can write once as well as read the disk.
The CD-R medium is similar to but not identical to that of a CD or
CD-ROM. For CDs and CD-ROMs, information is recorded by the pitting of
the surface of the medium, which changes reflectivity. For a CD-R, the medium
includes a dye layer. The dye is used to change reflectivity and is activated
by a high-intensity laser. The resulting disk can be read on a CD-R drive or a
CD-ROM drive.
The CD-R optical disk is attractive for archival storage of documents and files.
It provides a permanent record of large volumes of user data.
\paragraph{ CD REWRITABLE }\mbox{}\\\\%%
The CD-RW optical disk can be repeatedly written and
overwritten, as with a magnetic disk. Although a number of approaches have been
tried, the only pure optical approach that has proved attractive is called phase
change. The phase change disk uses a material that has two significantly different
reflectivities in two different phase states. There is an amorphous state, in which the
molecules exhibit a random orientation that reflects light poorly; and a crystalline
state, which has a smooth surface that reflects light well. A beam of laser light can
change the material from one phase to the other. The primary disadvantage of
phase change optical disks is that the material eventually and permanently loses
its desirable properties. Current materials can be used for between 500,000 and
1,000,000 erase cycles.
The CD-RW has the obvious advantage over CD-ROM and CD-R that it can
be rewritten and thus used as a true secondary storage. As such, it competes with
magnetic disk. A key advantage of the optical disk is that the engineering tolerances
for optical disks are much less severe than for high-capacity magnetic disks. Thus,
they exhibit higher reliability and longer life.
\paragraph{Digital Versatile Disk }\mbox{}\\\\%%
With the capacious digital versatile disk (DVD), the electronics industry has at last
found an acceptable replacement for the analog VHS video tape. The DVD has
replaced the videotape used in video cassette recorders (VCRs) and, more important
for this discussion, replace the CD-ROM in personal computers and servers.
The DVD takes video into the digital age. It delivers movies with impressive picture
quality, and it can be randomly accessed like audio CDs, which DVD machines
can also play. Vast volumes of data can be crammed onto the disk, currently seven
times as much as a CD-ROM. With DVD’s huge storage capacity and vivid quality,
PC games have become more realistic and educational software incorporates more
video. Following in the wake of these developments has been a new crest of traffic
over the Internet and corporate intranets, as this material is incorporated into
Web sites.
The DVD’s greater capacity is due to three differences from CDs (Figure 6.14):
\begin{itemize}
\item Bits are packed more closely on a DVD. The spacing between loops of a spiral on a
CD is 1.6 $\mu $m and the minimum distance between pits along the spiral is 0.834 $\mu $m.

/*214*/

The DVD uses a laser with shorter wavelength and achieves a loop spacing of
0.74 $\mu$m and a minimum distance between pits of 0.4 $\mu$m. The result of these
two improvements is about a seven-fold increase in capacity, to about 4.7 GB.
\item The DVD employs a second layer of pits and lands on top of the first layer. A
dual-layer DVD has a semireflective layer on top of the reflective layer, and
by adjusting focus, the lasers in DVD drives can read each layer separately.
This technique almost doubles the capacity of the disk, to about 8.5 GB. The
lower reflectivity of the second layer limits its storage capacity so that a full
doubling is not achieved.
\item The DVD-ROM can be two sided, whereas data are recorded on only one side
of a CD. This brings total capacity up to 17 GB.
As with the CD, DVDs come in writeable as well as read-only versions (Table 6.6).
\end{itemize}
\paragraph{High-Definition Optical Disks}\mbox{}\\\\%%

High-definition optical disks are designed to store high-definition videos and to
provide significantly greater storage capacity compared to DVDs. The higher bit
density is achieved by using a laser with a shorter wavelength, in the blue-violet

/*215*/
range. The data pits, which constitute the digital 1s and 0s, are smaller on the highdefinition
optical disks compared to DVD because of the shorter laser wavelength.
Two competing disk formats and technologies initially competed for market
acceptance: HD DVD and Blu-ray DVD. The Blu-ray scheme ultimately achieved
market dominance. The HD DVD scheme can store 15 GB on a single layer on a
single side. Blu-ray positions the data layer on the disk closer to the laser (shown on
the right-hand side of each diagram in Figure 6.15). This enables a tighter focus and
less distortion and thus smaller pits and tracks. Blu-ray can store 25 GB on a single
layer. Three versions are available: read only (BD-ROM), recordable once (BD-R),
and rerecordable (BD-RE).

\paragraph{Sony / Panasonic: Tecnología “ Archival Disc”}\mbox{}\\\\%%

$https://en.wikipedia.org/wiki/Archival_Disc$
\begin{itemize}
\item Primera generación
\item Segunda generación
\item Tercera generación
\end{itemize}

\paragraph{Sony “ Optical Disc Archive”}\mbox{}\\\\%%

$https://en.wikipedia.org/wiki/Optical_Disc_Archive$










