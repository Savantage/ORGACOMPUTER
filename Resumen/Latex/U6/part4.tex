\subsection{MAGNETIC TAPE}
\subsubsection{Definicion}
Magnetic tape is a medium for magnetic recording, made of a thin, magnetizable coating on a long, narrow strip of plastic film.

\subsubsection{Medio}
\begin{itemize}
\item Poliester flexible cubierto de material magnetizable
	\begin{itemize}
	\item Carretes abiertos
	\item Paquetes cerrados (cartuchos)
	\end{itemize}
\item Ancho de cinta entre 0.38 cm (0.25 pulgadas) y 1.27 cm (0.5 pulgadas)
\item Acceso secuencial a la información: si estoy en el registro 1 y quiero llegar al N tengo que “leer” los N-1 del medio
\item Si quiero leer un registro anterior tengo que rebobinar y volver a buscar el registro
\end{itemize}

\subsubsection{Técnicas de grabación}
	Grabación en paralelo
		\begin{itemize}
		\item Técnica usada originalmente
		\item Cabeza de grabación estacionaria
		\item Se graban pistas en paralelo a lo largo de la cinta
		\item Al principio eran de 9 pistas (8 bits de datos y 1 bit de paridad para detectar errores)
		\item Luego fueron 18 (palabra) o 36 (doble palabra) pistas
		\end{itemize}
	Grabación en serie
		\begin{itemize}
		\item Sistema moderno de grabación
		\item Cabeza de grabación estacionaria
		\item Se escriben los datos a lo largo de una pista primero hasta llegar al final de la cinta y luego se pasa a otra
		\item Grabación en “serpentina”
		\item Pueden grabarse n pistas adyacentes en simultáneo (n entre 2 y 8)
		\end{itemize}
	Grabación helicoidal
		\begin{itemize}
		\item Cabeza de grabación rotatoria
		\item Símil video casseteras
		\item Evita problema de movimiento veloz de la cinta de las otras técnicas
		\item La cinta se mueve en forma lenta mientras que la cabeza rota en forma rápida
		\item Las pistas pueden estar más cercanas unas a otras
		\end{itemize}

\subsubsection{Modos de operación}
	\begin{itemize}
	\item Modo start-stop por bloque
		\begin{itemize}
		\item Viejo uso de grabación por registro/bloque
		\item La cinta se usaba para guardar archivos para procesamiento posterior
		\item Se podía actualizar un registro/bloque particular siempre y cuando no cambiara su tamaño
		\item Los datos se grababan en bloques físicos
		\item Entre los bloques había espacios (IRG – Inter Record Gap) para sincronización de la unidad
		\end{itemize}
	\item Modo streaming
		\begin{itemize}
		\item Uso para backup o archivo de información
		\item No se requiere operación de start-stop por bloque
		\item No se requiere actualización de bloques particulares dentro de un archivo
		\item Se escriben archivos completos como un “stream” de datos contiguo
		\item La información de graba físicamente en bloques pero no se pueden localizar o modificar bloques particulares
		\end{itemize}
	\end{itemize}

\subsubsection{Usos y características}
	\begin{itemize}
	\item Fue el primer medio de almacenamiento secundario
	\item Aun es usado para backup y archivo de información (30 años o más de duración) dado su bajo costo por byte y su capacidad de almacenamiento
	\item Es el medio más lento de la pirámide de jerarquía de memoria
	\item Marcas físicas en las cintas
		\begin{itemize}
		\item BOT (Beginning of tape)
		\item EOT (End of tape)
		\end{itemize}
	\end{itemize}
