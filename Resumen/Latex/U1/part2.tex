\section{Formato y Configuracion}

\subsection{Definición}
\begin{itemize}
\item Formato: Representación computacional
\item Configuración: Representación en una determinada base de un número en un formato
\end{itemize}

\subsection{Expansión y truncamiento}
\subsection{Definición}
\begin{itemize}
\item Expandir formato: Significa completar la representación computacional sin alterar el numero representado
en el mismo. 
\item Truncar formato: Descartar digitos de su representación sin alterar el número representado en el mismo.
\end{itemize}

\subsection{Formatos}

%Binario punto fijo sin signo
\subsubsection{Binario punto fijo sin signo}
\begin{itemize}
\item Base: 2
\item Representa: números enteros positivos
\item Máximo: $2^{n-1}_{10}$
\item Mínimo: 0
\end{itemize}

\paragraph{Almacenar}
\begin{enumerate}
\item Pasar el numero a base 2
\item completar con ceros a izquierda la capacidad del formato
\end{enumerate}
\paragraph{Recuperar}
\begin{enumerate}
\item Pasamos el numero de base 2 a la base deseada 
\end{enumerate}

%Binario de punto fijo con signo
\subsubsection{Binario de punto fijo con signo}
\begin{itemize}
\item Base: 2
\item Representa: Enteros positivos y negativos
\item Primer bit: reservado para el signo
\item Máximo: $2^{n-1}-1$
\item Mínimo: $-2^{n-1}$
\end{itemize}

\paragraph{Almacenar}
\begin{enumerate}
\item Pasar el numero a base 2
\item completar con ceros a izquierda la capacidad del formato
\item Si es un numero negativo hacerle el complemento a 2
\end{enumerate}
\paragraph{Recuperar}
\begin{enumerate}
\item Si el bit de signo es cero, se pasa de base 2 a base 10
\item Si el bit es 1, el numero es negativo, por lo que debemos complementarlo
\item Quitamos los ceros a la izquierda
\item Numero de base 2 a base 10
\item Colocamos el signo
\end{enumerate}

\paragraph{Expansión}
\begin{enumerate}
\item Se completa con el bit de signo a la izquierda
\end{enumerate}
\paragraph{Truncamiento}
\begin{enumerate}
\item Se extraen bits a la izquierda siempre y cuando no se esté alterando el bit de signo del número.
\end{enumerate}

%Empaquetado
\subsubsection{Empaquetado}
\begin{itemize}
\item Base: 16
\item Representa: Enteros positivos y negativos
\item Máximo: $10^{2n-1} -1$
\item Mínimo: $-10^{2n-1} +1$
\end{itemize}

\paragraph{Almacenar}
\begin{enumerate}
\item numero a base 10
\item Colocal cada digito decimal en un nibble dejando el ultimo nibble ya que en el mismo se almacena el signo.
\item Colocar en el ultimo nible el signo, CAFE = + , DF = -
\item Se rellena con 0 hasta alcanzar la cantidad de bytes usados
\end{enumerate}
\paragraph{Recuperar}
\begin{enumerate}
\item los pasos en orden inverso
\end{enumerate}

%Zoneado
\subsubsection{Zoneado}
\begin{itemize}
\item Base: 16
\item Representa: Enteros positivos y negativos
\item Máximo: $10^{n} -1$
\item Mínimo: $-10^{n} +1$
\end{itemize}

\paragraph{Almacenar}
\begin{enumerate}
\item numero a base 10
\item colocar cada uno de los digitos decimales en un nibble derecho
\item completar todos los nibbles de izquierda con F salvo el ultimo que se completa con el signo siguiendo las mismas reglas que para empaquetados.  CAFE = + , DF = -
\item se rellena con F0 hasta alcanzar la cantidad de bytes usados
\end{enumerate}
\paragraph{Recuperar}
\begin{enumerate}
\item los pasos en orden inverso
\end{enumerate}

\subsubsection{Binario de punto Flotante}
es la manera que tiene una arquitectura de representar a los numeros reales.
su notación cientifica se expresa de la siguiente manera $M x B^E$
M: Mantissa B: base E: Exponente

Un numero binario está normalizado si el digito de la izquierda del punto es igual a 1.

\paragraph{IEEE754}
\begin{itemize}
\item Precision Simple: signo: 1 bit exponente: 8 bits fraccion: 23 bits Exponente: Exceso 127
\item Precision Doble: signo: 1 bit exponente 11bits fraccion: 52 bits Exponente: Exceso 1023
\end{itemize}

\paragraph{Ancho de paso}\mbox{}\\\\
Marca cual es la distancia entre un flotante y su siguiente numero representable en el formato

\paragraph{Overflow}\mbox{}\\\\
el exponente excede el limite superior, tanto para mantisas positivas como para negatvias, dando lugar a +inf, - inf
\paragraph{Underflow}\mbox{}\\\\
el exponente excede el minimo valor permitido y caga en el intervalo (-inf, -0) y (+0,+inf)

\paragraph{Desnormalizados - Subnormales}\mbox{}\\\\
tienen como exponente al cero, y el bit implicito a la izquierda del punto binario, es ahora un cero implicito. la diferencia entre los desnormalizados y los normalizados es que, estos ultimos no permiten al cero como exponente. Los normalizados tienen 24 bits significativos, mientras que los normalizados poseen 23.

\begin{table}[h]
\begin{tabular}{lllll}
\cline{1-4}
\multicolumn{1}{|l|}{Normalizado} & \multicolumn{1}{l|}{+/-} & \multicolumn{1}{l|}{0<exp<max} &  \multicolumn{1}{l|}{cualquier patron de bits} &  \\ \cline{1-4}
\multicolumn{1}{|l|}{Desnormalizado} & \multicolumn{1}{l|}{+/-} & \multicolumn{1}{l|}{0} &  \multicolumn{1}{l|}{cualquier patron de bits != 0} &  \\ \cline{1-4}
\multicolumn{1}{|l|}{Cero} & \multicolumn{1}{l|}{+/-} & \multicolumn{1}{l|}{0} &  \multicolumn{1}{l|}{0} &  \\ \cline{1-4}
\multicolumn{1}{|l|}{Infinito} & \multicolumn{1}{l|}{+/-} & \multicolumn{1}{l|}{11...11} &  \multicolumn{1}{l|}{0} &  \\ \cline{1-4}
\multicolumn{1}{|l|}{NAN} & \multicolumn{1}{l|}{+/-} & \multicolumn{1}{l|}{11...11} &  \multicolumn{1}{l|}{Cualquier patron de bits != 0} &  \\ \cline{1-4}
\end{tabular}
\end{table}

\begin{itemize}
\item Infinito dividido Infinito = NAN
\item Cero + = 0 00000000 00000000000000000000000
\item Cero - = 1 00000000 00000000000000000000000
\item Infinito + = 0 11111111 00000000000000000000000
\item Infinito - = 1 11111111 00000000000000000000000
\item No normalizados/Subnormales (no se asume que haya que añadir un 1 al significando para obtener su valor).
\end{itemize}

\paragraph{Valores no numericos}\mbox{}\\\\
NaN (Not a number). 2 tipos, QNaN (quiet nan) y SNaN (signalling nan) Qnan = indeterminado Snan = operacion no valida

\begin{itemize}
\item Infinito dividido Infinito = NAN
\item qnan = 0 11111111 10000100000000000000000
\item snan = 1 11111111 00100010001001010101010
\end{itemize}

\begin{table}[h]
\begin{tabular}{lllll}
\cline{1-2}
\multicolumn{1}{|l|}{Operacion} & \multicolumn{1}{l|}{Resultado} &   \\ \cline{1-2}
\multicolumn{1}{|l|}{n $\pm$ infinito } & \multicolumn{1}{l|}{0} &   \\ \cline{1-2}
\multicolumn{1}{|l|}{$\pm$ infinito $\cdot$ $\pm$infinito } & \multicolumn{1}{l|}{$\pm$infinito} &   \\ \cline{1-2}
\multicolumn{1}{|l|}{ $\pm$n $\div$ 0   } & \multicolumn{1}{l|}{ $\pm$infinito } &   \\ \cline{1-2}
\multicolumn{1}{|l|}{ Infinito + Infinito   } & \multicolumn{1}{l|}{ Infinito } &   \\ \cline{1-2}
\multicolumn{1}{|l|}{Cualquier operación contra un NaN   } & \multicolumn{1}{l|}{ NaN } &   \\ \cline{1-2}
\multicolumn{1}{|l|}{ $\pm$0 $\div$ $\pm$0   } & \multicolumn{1}{l|}{ NaN } &   \\ \cline{1-2}
\multicolumn{1}{|l|}{ Infinito - Infinito   } & \multicolumn{1}{l|}{  NaN} &   \\ \cline{1-2}
\multicolumn{1}{|l|}{ $\pm$Infinito  $\div$ $\pm$Infinito  } & \multicolumn{1}{l|}{  NaN} &   \\ \cline{1-2}
\multicolumn{1}{|l|}{ $\pm$Infinito  $\cdot$ $\pm$0 } & \multicolumn{1}{l|}{  NaN} &   \\ \cline{1-2}
\end{tabular}
\end{table}

\paragraph{Almacenar -6,12510}\mbox{}\\
\begin{enumerate}    
  \item El bit 31 tomará el valor del signo del número. 
  \item Pasar a binario la mantisa decimal.
  \begin{itemize}
    \item $6=110_2$
    \item $0,125=0,001_2$
    \item $6,125=110,001_2$
  \end{itemize}
  \item Normalizar
  \begin{itemize}
    \item Desplazamiento a la derecha  -> Exponente negativo
    \item Desplazamiento a la izquierda -> Exponente positivo
    \item 1,10001 , exponente = 2
    \item 2 expresado en Exceso 127 es 129 $0000001_2$
  \end{itemize}
  \item Mantisa representada con bit implícito: 1,10001 -> 10001
   \item El número final es 1 10000001 100010000000000000000002 (Se agregan a la derecha los “0” necesarios para completar los 23 bits de la mantisa)
\end{enumerate}

\paragraph{Recuperar}\mbox{}\\\\
\begin{enumerate}    
  \item Se realizan los pasos en orden inverso
\end{enumerate}

\paragraph{Binario de Punto Flotante (IBM mainframe)}
\begin{itemize}
\item Base: 16
\item Representa: enteros con coma decimal positivos y negativos
\item Precision : imple 4 bytes, doble, 8 bytes, extendida 16 bytes.
\item estructura: S nnnnnnn dddddd
\begin{itemize}
    \item s = signo 1- 0+
    \item n = digitos de la caracteristica, en total son 7 bits se usan para calcular el exponente 
    \item C = E + 4016 donde E corresponde al exponente
    \item d =  mantisa normalizada -> 0,dddddd x $10^e_{16}$
  \end{itemize}
\end{itemize}

\paragraph{Almacenar - 321,54 10 -> Binario de punto flotante precisión simple}\mbox{}\\\\
\begin{enumerate}    
  \item $321,54_{10} = 141,8A3_{16}$
  \item $0,1418A3 x 10^3 _{16}$
  \item $C = E + 40_{16} = 3 + 40_{16} = 43_{16} en base 2 sería 100011_{2}$
  \item Agregamos el bit de signo: $1100011_{2}$ que en base 16 es $C3_{16}$
  \item Resultado final : $C31418A3_{16}$
\end{enumerate}

\paragraph{Ancho de paso }\mbox{}\\\\
distancia entre un flotante y su siguiente numero representable en el formato

\paragraph{Absorcion}\mbox{}\\\\
se da en las operaciones de suma y resta entre flotantes

\begin{itemize}    
  \item $A = 0,15A4 x 105_{16}$
  \item $B = 0,54F x 10^-2_{16}$
\end{itemize}

para poder operar entre flotantes debemos igualar los exponentes llevandolos al mayor de todos
\begin{itemize}    
  \item $A+B = 0.15A400 * 10^5 + 0,00...00 * 10^5 = 0,15A4000 * 10^5 _{16}$
\end{itemize}
lo minimo que se puede sumar es el ancho de paso del numero de mayor exponente