\section{Sistemas de numeración}

\subsection{Definición}

Se conoce como un sistema de numeración un conjunto finito de símbolos que se emplea con algún método para asignar numerales , o símbolos numéricos, a los números. Hay diversos sistemas que han sido, o son actualmente empleados. 

\subsection{Clasificación}

\subsubsection{Sistemas de numeracion no posicionales}
El valor de un simbolo es independiente de la posicion que ocupa.

Ejemplo: Numeros romanos

\subsubsection{Sistemas de numeracion posicionales}\mbox{}\\
 El valor de un simbolo depende de la posicion que ocupa.
 El punto fraccionario es llamado punto decimal, en base diez, y punto binario, en base binaria.
 Ejemplo: Numeros decimales
 
\subsection{Dígitos en distintos sistemas}
\begin{itemize}
	\item Binario: 0, 1
	\item Octal: 0, 1, 2, 3, 4, 5, 6, 7
	\item Hexadecimal: 0, 1, 2, 3, 4, 5, 6, 7, 8, 9, A, B, C, D, E, F
\end{itemize}
 
\subsection{Cambio de base}
  
\subsubsection{Numeros enteros}\mbox{}\\

\paragraph{De base b a base 10}\mbox{}\\
\paragraph{De base 10 a base n}\mbox{}\\
\paragraph{De base n a base m}\mbox{}\\

\subsubsection{Numeros fraccionarios}\mbox{}\\

\paragraph{De base b a base 10}\mbox{}\\
\paragraph{De base 10 a base b}\mbox{}\\
\paragraph{De base n a base m}\mbox{}\\

\subsubsection{Casos especiales de cambio de base}
\begin{itemize}
	\item \[ b^x  = p \Rightarrow 2^3 = 8 \] tomo de a 3 dígitos
	\item \[ b^{1/x}  = p \Rightarrow 16^{1/4} = 2 \]  se expanden en 4 dígitos
\end{itemize}

\subsection{Números de precisión finita}
En la mayoria de las computadoras la cantidad de digitos para representar un número, se fija en su diseño. Observemos un ejemplo con números enteros positivos de tres digitos.
\begin{itemize}
	\item 600+600 = 1200 muy grande
	\item 003-005 = -002  negativo
	\item 007/002 = 3.5  no es un entero
\end{itemize}
Las exclusiones se dividen en dos grupos, los resultados que no pertenecen al conjunto
y los resultados que son mayores o menos al mayor y mínimo número del conjunto (overflow)

\subsection{Representación de números negativos}

\subsubsection{Signo y magnitud}
Se usa un bit para representa el signo, a menudo es el bit más significativo y por convención se usa el cero para números positivos y 1 para los negativos. Los n-1 bits restantes se usan para representar el significando que es la magnitud del número en valor absoluto.

\paragraph{Ventajas}
\begin{itemize}
\item Posee un rango simetrico
\end{itemize}

\paragraph{Desventajas}
\begin{itemize}
\item No permite operar arimeticamente
\item Posee doble representación del cero
\end{itemize}

\subsubsection{Complemento a uno}
Consiste en aplicarle un NOT bit a bit al número. Se usa un bit para representar el signo, a menudo es el bit más significativo, por convención se usa un cero para los positivos y un 1 para los negativos. Los n-1 bit restantes para representar el significando que es la magnitud del numero en valor absoluto para el caso de números positivos o bien, es el complemento a uno del valor absoluto del número en caso de ser negativo.

\paragraph{Ventajas}
\begin{itemize}
\item Posee un rango simetrico
\item Permite operar arimeticamente y para obtener el resultado correcto al operar se debe sumar el acarreo obtenido al final de la suma/resta realizadas en caso de haberlo obtenido, este acarreo es conocido con el nombre de end-around carry.
\end{itemize}

\paragraph{Desventajas}
\begin{itemize}
\item Posee doble representación del cero
\end{itemize}

\subsubsection{Complemento a la base}
El complemento de un número dado en una base es aquel que sumado al número original da la base a la n, siendo n la cantidad de dígitos que componene a ese número. El concepto de complemento puede ser usado para transformar una operación de resta de dos números en la suma de uno de ellos más el complemento del otro.\\\\

\paragraph{Complemento a 10 de $13579_{10}$}\mbox{}\\\\
$100000 - 13579 = 86421_{10}$

\paragraph{A-B}\mbox{}\\\\
$
A = 10376 \;\;\;\;\;\;\;\;B = 234\\
A-B = C\\
A+B_{comp} = C+D  \;\;\;\;\;\;\;\;\;\; B_{comp} = 100000-234=99766\\
10376 + 99766 = D + 100000\\
10142 = D\\
$

\subsubsection{Complemento a dos}
Permite representar números negatvios en el sistema binario y la realización de restas mediante sumas. Estos números negativos están representados a través de su complmento. El complemento a 2 de un número se obtiene de sumar 1 al número negado bit a bit.

\paragraph{Complemento a dos de $01101_2$}\mbox{}\\\\
$10010 + 00001 = 10011_2$

\paragraph{Ventajas}
\begin{itemize}
\item No posee doble representación del cero
\item Permite operar arimeticamente
\end{itemize}

\paragraph{Desventajas}
\begin{itemize}
\item Posee un rango asimetrico
\end{itemize}

\subsubsection{Exceso a la base}
Consiste en tomar el valor real del número a representar sumarle la base elevada según la cantidad de dígitos menos 1 que se tienen disponibles. Esto se lo conoce como representación en Exceso a base $B^{n-1}$, puesto que a cada número se le suma
el mismo valor y está en exceso por dicho valor.

\paragraph{Guardar}\mbox{}\\\\
El exceso es $2^{8-1}=128\\
-97_{10} + 128 = 31_{10}\\
31_{10} = 00011111_{2}$

\paragraph{Obtener}\mbox{}\\\\
El exceso es $2^{8-1}=128\\
181-128 = 53_{10}\\
53_{10} = 00110101_{2}$

\paragraph{Ventajas}
\begin{itemize}
\item No hay empaquetación del número 
\item Permite operar arimeticamente
\end{itemize}

\paragraph{Desventajas}
\begin{itemize}
\item Requiere operaciónes arimeticas intermedias
\item Posee un rango asimetrico
\end{itemize}



