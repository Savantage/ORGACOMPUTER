\section{Maquina Elemental}

\subsection{Definiciones}
\begin{itemize}
\item Data: Data is raw, unorganized facts that need to be processed. Data can be something simple and seemingly random and useless until it is organized.
\item Information: When data is processed, organized, structured or presented in a given context so as to make it useful, it is called information
\item Bit: Unidad de información
\item Digito binario: unidad de información más pequeña con la que pueden trabajar los dispositivos electrónicos que constituyen una computadora digital.
\item Biestable: Un biestable, es un multivibrador capaz de permanecer en uno de dos estados posibles durante un tiempo indefinido en ausencia de perturbaciones.
\item Byte: Unidad de información de base utilizada en computación y en telecomunicaciones, y que resulta equivalente a un conjunto ordenado de ocho bits
\item Proceso: Serie de operaciones llevadas acabo por una computadora.
\item Computadora: Es una máquina que consta de elementos mecánicos, eléctricos y electrónicos capaz de procesar gran cantidad de información a alta velocidad.
\end{itemize}

\subsection{Clasificación de computadoras según su generación}

\subsubsection{Generación 0 - Computadoras Mecánicas (1642 - 1945)}
\begin{itemize}
\item Pascal
	\begin{itemize}
	\item Máquina mecánica de cálculo 1642
	\item Babbage
	\end{itemize}
\item Máquina de diferencias 1790
	\begin{itemize}
	\item Máquina analítica (propósito general) Primera programadora (ADA) 1834
	\end{itemize}
\item Aiken
	\begin{itemize}
	\item Máquina analítica Mark I (Harvard 1944)
	\end{itemize}
\end{itemize}

\subsubsection{Generación 1 - Tubos de vacío (1945-1955)}
\begin{itemize}
\item Turing
	\begin{itemize}
	\item Colossus - Computadora electrónica (1943)
	\end{itemize}
\item Mauchley/Eckert
	\begin{itemize}
	\item ENIAC - Programable por switches (1946)
	\end{itemize}
\item Von Neumann
	\begin{itemize}
	\item IAS Machine - Programa almacenado (1952)
	\end{itemize}
\end{itemize}

\subsubsection{Generación 2 - Transistores (1955-1965)}
\begin{itemize}
\item DEC
	\begin{itemize}
	\item PDP 1 - Primera minicomputadora (1961)
	\item PDP 8 - Primera mini masiva / Omnibus (1965)
	\end{itemize}
\item IBM
	\begin{itemize}
	\item 7094 - Dominio en uso cientifico (1962)
	\end{itemize}
\item CDC
	\begin{itemize}
	\item 6600 - Primera supercomputadora (1964)
	\end{itemize}
\item Burroughs
	\begin{itemize}
	\item B5000 - Primera computadora diseñada para un lenguale de alto nivel (1963)
	\end{itemize}
\end{itemize}

\subsubsection{Generación 3 - Circuitos Integrados (1965 - 1980)}
\begin{itemize}
\item DEC
	\begin{itemize}
	\item PDP 11 - Minicomputadora dominante (1970)
	\end{itemize}
\item IBM
	\begin{itemize}
	\item SYSTEM/360 - Familia de computadoras (1964)
		\begin{itemize}
		\item Uso comercial y cientifico
		\item Distintos modelos compatibles entre sí con distintas capacidades
		\item multiprogramación
		\item simulaba otras arquitecturas con microarquitectura programada (7094 y 1401)
		\end{itemize}
	\end{itemize}
\end{itemize}

\subsubsection{Generación 4 - Integración a muy gran escala VLSI (1980 - ?)}
\begin{itemize}
\item IBM
	\begin{itemize}
	\item PC - Comienzo era computadores personales (1981)
		\begin{itemize}
		\item Intel 8088
		\item MS-DOS
		\end{itemize}
	\end{itemize}
\item Apple
	\begin{itemize}
	\item Apple Lisa - Primera computadora con GUI (1983)
	\end{itemize}
\item DEC
	\begin{itemize}
	\item Alpha - Primera computadora RISC de 64 bits (1992)
	\end{itemize}	
\item COMMODORE/ATARI
	\begin{itemize}
	\item Computadoras hogareñas sin estandar
	\end{itemize}	
\end{itemize}

\subsubsection{Generación 5 - Computadoras "invisibles"}
\begin{itemize}
\item Apple
	\begin{itemize}
	\item Newton - Primera palmtop (1993)
	\end{itemize}
\item Computadores embebidos
	\begin{itemize}
	\item Relojes inteligentes
	\item Celulares
	\item Electrodomesticos
	\end{itemize}
\end{itemize}








