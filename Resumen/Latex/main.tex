%%%%%%%%%%%%%%%%%%%%%%%%%%%%%%%%%%%%%%%%%
% Academic Title Page
% LaTeX Template
% Version 2.0 (17/7/17)
%
% This template was downloaded from:
% http://www.LaTeXTemplates.com
%
% Original author:
% WikiBooks (LaTeX - Title Creation) with modifications by:
% Vel (vel@latextemplates.com)
%
% License:
% CC BY-NC-SA 3.0 (http://creativecommons.org/licenses/by-nc-sa/3.0/)
% 
% Instructions for using this template:
% This title page is capable of being compiled as is. This is not useful for 
% including it in another document. To do this, you have two options: 
%
% 1) Copy/paste everything between \begin{document} and \end{document} 
% starting at \begin{titlepage} and paste this into another LaTeX file where you 
% want your title page.
% OR
% 2) Remove everything outside the \begin{titlepage} and \end{titlepage}, rename
% this file and move it to the same directory as the LaTeX file you wish to add it to. 
% Then add \input{./<new filename>.tex} to your LaTeX file where you want your
% title page.
%
%%%%%%%%%%%%%%%%%%%%%%%%%%%%%%%%%%%%%%%%%

%----------------------------------------------------------------------------------------
%	PACKAGES AND OTHER DOCUMENT CONFIGURATIONS
%----------------------------------------------------------------------------------------


\documentclass[11pt]{article}
\usepackage{geometry}
\usepackage{graphicx}
\usepackage{url}
\usepackage[utf8]{inputenc} % Required for inputting international characters
\usepackage[T1]{fontenc} % Output font encoding for international characters

\usepackage{mathpazo} % Palatino font

\begin{document}
\setcounter{secnumdepth}{5}
%----------------------------------------------------------------------------------------
%	TITLE PAGE
%----------------------------------------------------------------------------------------

\begin{titlepage} % Suppresses displaying the page number on the title page and the subsequent page counts as page 1
	\newcommand{\HRule}{\rule{\linewidth}{0.5mm}} % Defines a new command for horizontal lines, change thickness here
	
	\center % Centre everything on the page
	
	%------------------------------------------------
	%	Headings
	%------------------------------------------------
	
	\textsc{\LARGE Universidad de Buenos Aires}\\[1.5cm] % Main heading such as the name of your university/college
	
	\textsc{\Large Facultad de Ingeniería}\\[0.5cm] % Major heading such as course name
	
	\textsc{\large Resumen}\\[0.5cm] % Minor heading such as course title
	
	%------------------------------------------------
	%	Title
	%------------------------------------------------
	
	\HRule\\[0.4cm]
	
	{\huge\bfseries Organización del Computador \newline 75.03 \& 95.57}\\[0.4cm] % Title of your document
	
	\HRule\\[1.5cm]
	
	%------------------------------------------------
	%	Author(s)
	%------------------------------------------------
	
	%\begin{minipage}{0.4\textwidth}
	%	\begin{flushleft}
	%		\large
	%		\textit{Autor}\\
	%		\textsc{Anzu} % Your name
	%	\end{flushleft}
	%\end{minipage}
	%~
	%\begin{minipage}{0.4\textwidth}
	%	\begin{flushright}
	%		\large
	%		\textit{Supervisor}\\
	%		\textsc{Anzu} % Supervisor's name
	%	\end{flushright}
	%\end{minipage}
	
	% If you don't want a supervisor, uncomment the two lines below and comment the code above
	{\large\textit{Autor}}\\
	\textsc{Anzu} % Your name
	
	%------------------------------------------------
	%	Date
	%------------------------------------------------
	
	\vfill\vfill\vfill % Position the date 3/4 down the remaining page
	
	{\large\today} % Date, change the \today to a set date if you want to be precise
	
	%------------------------------------------------
	%	Logo
	%------------------------------------------------
	
	%\vfill\vfill
	%\includegraphics[width=0.2\textwidth]{placeholder.jpg}\\[1cm] % Include a department/university logo - this will require the graphicx package
	 
	%----------------------------------------------------------------------------------------
	
	\vfill % Push the date up 1/4 of the remaining page
	
\end{titlepage}

%----------------------------------------------------------------------------------------

\tableofcontents

\newpage

\section{Intel x86}
\section{Formato y Configuracion}

\subsection{Definición}
\begin{itemize}
\item Formato: Representación computacional
\item Configuración: Representación en una determinada base de un número en un formato
\end{itemize}

\subsection{Expansión y truncamiento}
\subsection{Definición}
\begin{itemize}
\item Expandir formato: Significa completar la representación computacional sin alterar el numero representado
en el mismo. 
\item Truncar formato: Descartar digitos de su representación sin alterar el número representado en el mismo.
\end{itemize}

\subsection{Formatos}

%Binario punto fijo sin signo
\subsubsection{Binario punto fijo sin signo}
\begin{itemize}
\item Base: 2
\item Representa: números enteros positivos
\item Máximo: $2^{n-1}_{10}$
\item Mínimo: 0
\end{itemize}

\paragraph{Almacenar}
\begin{enumerate}
\item Pasar el numero a base 2
\item completar con ceros a izquierda la capacidad del formato
\end{enumerate}
\paragraph{Recuperar}
\begin{enumerate}
\item Pasamos el numero de base 2 a la base deseada 
\end{enumerate}

%Binario de punto fijo con signo
\subsubsection{Binario de punto fijo con signo}
\begin{itemize}
\item Base: 2
\item Representa: Enteros positivos y negativos
\item Primer bit: reservado para el signo
\item Máximo: $2^{n-1}-1$
\item Mínimo: $-2^{n-1}$
\end{itemize}

\paragraph{Almacenar}
\begin{enumerate}
\item Pasar el numero a base 2
\item completar con ceros a izquierda la capacidad del formato
\item Si es un numero negativo hacerle el complemento a 2
\end{enumerate}
\paragraph{Recuperar}
\begin{enumerate}
\item Si el bit de signo es cero, se pasa de base 2 a base 10
\item Si el bit es 1, el numero es negativo, por lo que debemos complementarlo
\item Quitamos los ceros a la izquierda
\item Numero de base 2 a base 10
\item Colocamos el signo
\end{enumerate}

\paragraph{Expansión}
\begin{enumerate}
\item Se completa con el bit de signo a la izquierda
\end{enumerate}
\paragraph{Truncamiento}
\begin{enumerate}
\item Se extraen bits a la izquierda siempre y cuando no se esté alterando el bit de signo del número.
\end{enumerate}

%Empaquetado
\subsubsection{Empaquetado}
\begin{itemize}
\item Base: 16
\item Representa: Enteros positivos y negativos
\item Máximo: $10^{2n-1} -1$
\item Mínimo: $-10^{2n-1} +1$
\end{itemize}

\paragraph{Almacenar}
\begin{enumerate}
\item numero a base 10
\item Colocal cada digito decimal en un nibble dejando el ultimo nibble ya que en el mismo se almacena el signo.
\item Colocar en el ultimo nible el signo, CAFE = + , DF = -
\item Se rellena con 0 hasta alcanzar la cantidad de bytes usados
\end{enumerate}
\paragraph{Recuperar}
\begin{enumerate}
\item los pasos en orden inverso
\end{enumerate}

%Zoneado
\subsubsection{Zoneado}
\begin{itemize}
\item Base: 16
\item Representa: Enteros positivos y negativos
\item Máximo: $10^{n} -1$
\item Mínimo: $-10^{n} +1$
\end{itemize}

\paragraph{Almacenar}
\begin{enumerate}
\item numero a base 10
\item colocar cada uno de los digitos decimales en un nibble derecho
\item completar todos los nibbles de izquierda con F salvo el ultimo que se completa con el signo siguiendo las mismas reglas que para empaquetados.  CAFE = + , DF = -
\item se rellena con F0 hasta alcanzar la cantidad de bytes usados
\end{enumerate}
\paragraph{Recuperar}
\begin{enumerate}
\item los pasos en orden inverso
\end{enumerate}

\subsubsection{Binario de punto Flotante}
es la manera que tiene una arquitectura de representar a los numeros reales.
su notación cientifica se expresa de la siguiente manera $M x B^E$
M: Mantissa B: base E: Exponente

Un numero binario está normalizado si el digito de la izquierda del punto es igual a 1.

\paragraph{IEEE754}
\begin{itemize}
\item Precision Simple: signo: 1 bit exponente: 8 bits fraccion: 23 bits Exponente: Exceso 127
\item Precision Doble: signo: 1 bit exponente 11bits fraccion: 52 bits Exponente: Exceso 1023
\end{itemize}

\paragraph{Ancho de paso}\mbox{}\\\\
Marca cual es la distancia entre un flotante y su siguiente numero representable en el formato

\paragraph{Overflow}\mbox{}\\\\
el exponente excede el limite superior, tanto para mantisas positivas como para negatvias, dando lugar a +inf, - inf
\paragraph{Underflow}\mbox{}\\\\
el exponente excede el minimo valor permitido y caga en el intervalo (-inf, -0) y (+0,+inf)

\paragraph{Desnormalizados - Subnormales}\mbox{}\\\\
tienen como exponente al cero, y el bit implicito a la izquierda del punto binario, es ahora un cero implicito. la diferencia entre los desnormalizados y los normalizados es que, estos ultimos no permiten al cero como exponente. Los normalizados tienen 24 bits significativos, mientras que los normalizados poseen 23.

\begin{table}[h]
\begin{tabular}{lllll}
\cline{1-4}
\multicolumn{1}{|l|}{Normalizado} & \multicolumn{1}{l|}{+/-} & \multicolumn{1}{l|}{0<exp<max} &  \multicolumn{1}{l|}{cualquier patron de bits} &  \\ \cline{1-4}
\multicolumn{1}{|l|}{Desnormalizado} & \multicolumn{1}{l|}{+/-} & \multicolumn{1}{l|}{0} &  \multicolumn{1}{l|}{cualquier patron de bits != 0} &  \\ \cline{1-4}
\multicolumn{1}{|l|}{Cero} & \multicolumn{1}{l|}{+/-} & \multicolumn{1}{l|}{0} &  \multicolumn{1}{l|}{0} &  \\ \cline{1-4}
\multicolumn{1}{|l|}{Infinito} & \multicolumn{1}{l|}{+/-} & \multicolumn{1}{l|}{11...11} &  \multicolumn{1}{l|}{0} &  \\ \cline{1-4}
\multicolumn{1}{|l|}{NAN} & \multicolumn{1}{l|}{+/-} & \multicolumn{1}{l|}{11...11} &  \multicolumn{1}{l|}{Cualquier patron de bits != 0} &  \\ \cline{1-4}
\end{tabular}
\end{table}

\begin{itemize}
\item Infinito dividido Infinito = NAN
\item Cero + = 0 00000000 00000000000000000000000
\item Cero - = 1 00000000 00000000000000000000000
\item Infinito + = 0 11111111 00000000000000000000000
\item Infinito - = 1 11111111 00000000000000000000000
\item No normalizados/Subnormales (no se asume que haya que añadir un 1 al significando para obtener su valor).
\end{itemize}

\paragraph{Valores no numericos}\mbox{}\\\\
NaN (Not a number). 2 tipos, QNaN (quiet nan) y SNaN (signalling nan) Qnan = indeterminado Snan = operacion no valida

\begin{itemize}
\item Infinito dividido Infinito = NAN
\item qnan = 0 11111111 10000100000000000000000
\item snan = 1 11111111 00100010001001010101010
\end{itemize}

\begin{table}[h]
\begin{tabular}{lllll}
\cline{1-2}
\multicolumn{1}{|l|}{Operacion} & \multicolumn{1}{l|}{Resultado} &   \\ \cline{1-2}
\multicolumn{1}{|l|}{n $\pm$ infinito } & \multicolumn{1}{l|}{0} &   \\ \cline{1-2}
\multicolumn{1}{|l|}{$\pm$ infinito $\cdot$ $\pm$infinito } & \multicolumn{1}{l|}{$\pm$infinito} &   \\ \cline{1-2}
\multicolumn{1}{|l|}{ $\pm$n $\div$ 0   } & \multicolumn{1}{l|}{ $\pm$infinito } &   \\ \cline{1-2}
\multicolumn{1}{|l|}{ Infinito + Infinito   } & \multicolumn{1}{l|}{ Infinito } &   \\ \cline{1-2}
\multicolumn{1}{|l|}{Cualquier operación contra un NaN   } & \multicolumn{1}{l|}{ NaN } &   \\ \cline{1-2}
\multicolumn{1}{|l|}{ $\pm$0 $\div$ $\pm$0   } & \multicolumn{1}{l|}{ NaN } &   \\ \cline{1-2}
\multicolumn{1}{|l|}{ Infinito - Infinito   } & \multicolumn{1}{l|}{  NaN} &   \\ \cline{1-2}
\multicolumn{1}{|l|}{ $\pm$Infinito  $\div$ $\pm$Infinito  } & \multicolumn{1}{l|}{  NaN} &   \\ \cline{1-2}
\multicolumn{1}{|l|}{ $\pm$Infinito  $\cdot$ $\pm$0 } & \multicolumn{1}{l|}{  NaN} &   \\ \cline{1-2}
\end{tabular}
\end{table}

\paragraph{Almacenar -6,12510}\mbox{}\\
\begin{enumerate}    
  \item El bit 31 tomará el valor del signo del número. 
  \item Pasar a binario la mantisa decimal.
  \begin{itemize}
    \item $6=110_2$
    \item $0,125=0,001_2$
    \item $6,125=110,001_2$
  \end{itemize}
  \item Normalizar
  \begin{itemize}
    \item Desplazamiento a la derecha  -> Exponente negativo
    \item Desplazamiento a la izquierda -> Exponente positivo
    \item 1,10001 , exponente = 2
    \item 2 expresado en Exceso 127 es 129 $0000001_2$
  \end{itemize}
  \item Mantisa representada con bit implícito: 1,10001 -> 10001
   \item El número final es 1 10000001 100010000000000000000002 (Se agregan a la derecha los “0” necesarios para completar los 23 bits de la mantisa)
\end{enumerate}

\paragraph{Recuperar}\mbox{}\\\\
\begin{enumerate}    
  \item Se realizan los pasos en orden inverso
\end{enumerate}

\paragraph{Binario de Punto Flotante (IBM mainframe)}
\begin{itemize}
\item Base: 16
\item Representa: enteros con coma decimal positivos y negativos
\item Precision : imple 4 bytes, doble, 8 bytes, extendida 16 bytes.
\item estructura: S nnnnnnn dddddd
\begin{itemize}
    \item s = signo 1- 0+
    \item n = digitos de la caracteristica, en total son 7 bits se usan para calcular el exponente 
    \item C = E + 4016 donde E corresponde al exponente
    \item d =  mantisa normalizada -> 0,dddddd x $10^e_{16}$
  \end{itemize}
\end{itemize}

\paragraph{Almacenar - 321,54 10 -> Binario de punto flotante precisión simple}\mbox{}\\\\
\begin{enumerate}    
  \item $321,54_{10} = 141,8A3_{16}$
  \item $0,1418A3 x 10^3 _{16}$
  \item $C = E + 40_{16} = 3 + 40_{16} = 43_{16} en base 2 sería 100011_{2}$
  \item Agregamos el bit de signo: $1100011_{2}$ que en base 16 es $C3_{16}$
  \item Resultado final : $C31418A3_{16}$
\end{enumerate}

\paragraph{Ancho de paso }\mbox{}\\\\
distancia entre un flotante y su siguiente numero representable en el formato

\paragraph{Absorcion}\mbox{}\\\\
se da en las operaciones de suma y resta entre flotantes

\begin{itemize}    
  \item $A = 0,15A4 x 105_{16}$
  \item $B = 0,54F x 10^-2_{16}$
\end{itemize}

para poder operar entre flotantes debemos igualar los exponentes llevandolos al mayor de todos
\begin{itemize}    
  \item $A+B = 0.15A400 * 10^5 + 0,00...00 * 10^5 = 0,15A4000 * 10^5 _{16}$
\end{itemize}
lo minimo que se puede sumar es el ancho de paso del numero de mayor exponente

\subsection{Interrupciones}
\subsubsection{Definicion}
Mecanismos por los cuales otros modulos (E/S y memoria) interrumpen el normal procesamiento del CPU
\subsubsection{¿Para que existen?}
Para mejorar la eficiencia de procesamiento de un computador
\subsubsection{Clases de interrupciones}
\begin{itemize}
	\item programa
	\item  reloj
	\item e/s
	\item fallas de hardware
\end{itemize}

\subsubsection{Ciclo de instruccion}
\begin{itemize}
	\item Fetch instruction
	\item  Decode instruction
	\item Fetch operand
	\item Execute instruction
	\item Store result
	\item ----------------> interrupt breakpoint
	\item process interrupt
\end{itemize}

\subsubsection{Transferencia de control al S.O. (Handler)}
/*4pdf*/
	
\subsubsection{Procesamiento de interrupciones }
/*5pdf*/

/*6pdf ejemplo*/
	
\subsubsection{Multiples interrupciones}


\paragraph{Deshabilitar interrupciones (secuencia)}\mbox{}\\\\%%
/*7pdf*/
\paragraph{Priorizar interrupciones (anidadas)}\mbox{}\\\\%%
/*8pdf*/
	
\paragraph{Múltiples interrupciones - ejemplo}\mbox{}\\\\%%
Tres dispositivos de E/S
\begin{itemize}
	\item Línea de comunicación (Prioridad 1)
	\item  Disco (Prioridad 2)
	\item Impresora (Prioridad 3)
\end{itemize}
Eventos
\begin{itemize}
	\item T=10 Interrupción de Impresora
	\item T=15 Interrupción de línea de comunicación
	\item T=20 Interrupción de disco
\end{itemize}

/*10pdf*/


\section{Intel x86}
\section{Formato y Configuracion}

\subsection{Definición}
\begin{itemize}
\item Formato: Representación computacional
\item Configuración: Representación en una determinada base de un número en un formato
\end{itemize}

\subsection{Expansión y truncamiento}
\subsection{Definición}
\begin{itemize}
\item Expandir formato: Significa completar la representación computacional sin alterar el numero representado
en el mismo. 
\item Truncar formato: Descartar digitos de su representación sin alterar el número representado en el mismo.
\end{itemize}

\subsection{Formatos}

%Binario punto fijo sin signo
\subsubsection{Binario punto fijo sin signo}
\begin{itemize}
\item Base: 2
\item Representa: números enteros positivos
\item Máximo: $2^{n-1}_{10}$
\item Mínimo: 0
\end{itemize}

\paragraph{Almacenar}
\begin{enumerate}
\item Pasar el numero a base 2
\item completar con ceros a izquierda la capacidad del formato
\end{enumerate}
\paragraph{Recuperar}
\begin{enumerate}
\item Pasamos el numero de base 2 a la base deseada 
\end{enumerate}

%Binario de punto fijo con signo
\subsubsection{Binario de punto fijo con signo}
\begin{itemize}
\item Base: 2
\item Representa: Enteros positivos y negativos
\item Primer bit: reservado para el signo
\item Máximo: $2^{n-1}-1$
\item Mínimo: $-2^{n-1}$
\end{itemize}

\paragraph{Almacenar}
\begin{enumerate}
\item Pasar el numero a base 2
\item completar con ceros a izquierda la capacidad del formato
\item Si es un numero negativo hacerle el complemento a 2
\end{enumerate}
\paragraph{Recuperar}
\begin{enumerate}
\item Si el bit de signo es cero, se pasa de base 2 a base 10
\item Si el bit es 1, el numero es negativo, por lo que debemos complementarlo
\item Quitamos los ceros a la izquierda
\item Numero de base 2 a base 10
\item Colocamos el signo
\end{enumerate}

\paragraph{Expansión}
\begin{enumerate}
\item Se completa con el bit de signo a la izquierda
\end{enumerate}
\paragraph{Truncamiento}
\begin{enumerate}
\item Se extraen bits a la izquierda siempre y cuando no se esté alterando el bit de signo del número.
\end{enumerate}

%Empaquetado
\subsubsection{Empaquetado}
\begin{itemize}
\item Base: 16
\item Representa: Enteros positivos y negativos
\item Máximo: $10^{2n-1} -1$
\item Mínimo: $-10^{2n-1} +1$
\end{itemize}

\paragraph{Almacenar}
\begin{enumerate}
\item numero a base 10
\item Colocal cada digito decimal en un nibble dejando el ultimo nibble ya que en el mismo se almacena el signo.
\item Colocar en el ultimo nible el signo, CAFE = + , DF = -
\item Se rellena con 0 hasta alcanzar la cantidad de bytes usados
\end{enumerate}
\paragraph{Recuperar}
\begin{enumerate}
\item los pasos en orden inverso
\end{enumerate}

%Zoneado
\subsubsection{Zoneado}
\begin{itemize}
\item Base: 16
\item Representa: Enteros positivos y negativos
\item Máximo: $10^{n} -1$
\item Mínimo: $-10^{n} +1$
\end{itemize}

\paragraph{Almacenar}
\begin{enumerate}
\item numero a base 10
\item colocar cada uno de los digitos decimales en un nibble derecho
\item completar todos los nibbles de izquierda con F salvo el ultimo que se completa con el signo siguiendo las mismas reglas que para empaquetados.  CAFE = + , DF = -
\item se rellena con F0 hasta alcanzar la cantidad de bytes usados
\end{enumerate}
\paragraph{Recuperar}
\begin{enumerate}
\item los pasos en orden inverso
\end{enumerate}

\subsubsection{Binario de punto Flotante}
es la manera que tiene una arquitectura de representar a los numeros reales.
su notación cientifica se expresa de la siguiente manera $M x B^E$
M: Mantissa B: base E: Exponente

Un numero binario está normalizado si el digito de la izquierda del punto es igual a 1.

\paragraph{IEEE754}
\begin{itemize}
\item Precision Simple: signo: 1 bit exponente: 8 bits fraccion: 23 bits Exponente: Exceso 127
\item Precision Doble: signo: 1 bit exponente 11bits fraccion: 52 bits Exponente: Exceso 1023
\end{itemize}

\paragraph{Ancho de paso}\mbox{}\\\\
Marca cual es la distancia entre un flotante y su siguiente numero representable en el formato

\paragraph{Overflow}\mbox{}\\\\
el exponente excede el limite superior, tanto para mantisas positivas como para negatvias, dando lugar a +inf, - inf
\paragraph{Underflow}\mbox{}\\\\
el exponente excede el minimo valor permitido y caga en el intervalo (-inf, -0) y (+0,+inf)

\paragraph{Desnormalizados - Subnormales}\mbox{}\\\\
tienen como exponente al cero, y el bit implicito a la izquierda del punto binario, es ahora un cero implicito. la diferencia entre los desnormalizados y los normalizados es que, estos ultimos no permiten al cero como exponente. Los normalizados tienen 24 bits significativos, mientras que los normalizados poseen 23.

\begin{table}[h]
\begin{tabular}{lllll}
\cline{1-4}
\multicolumn{1}{|l|}{Normalizado} & \multicolumn{1}{l|}{+/-} & \multicolumn{1}{l|}{0<exp<max} &  \multicolumn{1}{l|}{cualquier patron de bits} &  \\ \cline{1-4}
\multicolumn{1}{|l|}{Desnormalizado} & \multicolumn{1}{l|}{+/-} & \multicolumn{1}{l|}{0} &  \multicolumn{1}{l|}{cualquier patron de bits != 0} &  \\ \cline{1-4}
\multicolumn{1}{|l|}{Cero} & \multicolumn{1}{l|}{+/-} & \multicolumn{1}{l|}{0} &  \multicolumn{1}{l|}{0} &  \\ \cline{1-4}
\multicolumn{1}{|l|}{Infinito} & \multicolumn{1}{l|}{+/-} & \multicolumn{1}{l|}{11...11} &  \multicolumn{1}{l|}{0} &  \\ \cline{1-4}
\multicolumn{1}{|l|}{NAN} & \multicolumn{1}{l|}{+/-} & \multicolumn{1}{l|}{11...11} &  \multicolumn{1}{l|}{Cualquier patron de bits != 0} &  \\ \cline{1-4}
\end{tabular}
\end{table}

\begin{itemize}
\item Infinito dividido Infinito = NAN
\item Cero + = 0 00000000 00000000000000000000000
\item Cero - = 1 00000000 00000000000000000000000
\item Infinito + = 0 11111111 00000000000000000000000
\item Infinito - = 1 11111111 00000000000000000000000
\item No normalizados/Subnormales (no se asume que haya que añadir un 1 al significando para obtener su valor).
\end{itemize}

\paragraph{Valores no numericos}\mbox{}\\\\
NaN (Not a number). 2 tipos, QNaN (quiet nan) y SNaN (signalling nan) Qnan = indeterminado Snan = operacion no valida

\begin{itemize}
\item Infinito dividido Infinito = NAN
\item qnan = 0 11111111 10000100000000000000000
\item snan = 1 11111111 00100010001001010101010
\end{itemize}

\begin{table}[h]
\begin{tabular}{lllll}
\cline{1-2}
\multicolumn{1}{|l|}{Operacion} & \multicolumn{1}{l|}{Resultado} &   \\ \cline{1-2}
\multicolumn{1}{|l|}{n $\pm$ infinito } & \multicolumn{1}{l|}{0} &   \\ \cline{1-2}
\multicolumn{1}{|l|}{$\pm$ infinito $\cdot$ $\pm$infinito } & \multicolumn{1}{l|}{$\pm$infinito} &   \\ \cline{1-2}
\multicolumn{1}{|l|}{ $\pm$n $\div$ 0   } & \multicolumn{1}{l|}{ $\pm$infinito } &   \\ \cline{1-2}
\multicolumn{1}{|l|}{ Infinito + Infinito   } & \multicolumn{1}{l|}{ Infinito } &   \\ \cline{1-2}
\multicolumn{1}{|l|}{Cualquier operación contra un NaN   } & \multicolumn{1}{l|}{ NaN } &   \\ \cline{1-2}
\multicolumn{1}{|l|}{ $\pm$0 $\div$ $\pm$0   } & \multicolumn{1}{l|}{ NaN } &   \\ \cline{1-2}
\multicolumn{1}{|l|}{ Infinito - Infinito   } & \multicolumn{1}{l|}{  NaN} &   \\ \cline{1-2}
\multicolumn{1}{|l|}{ $\pm$Infinito  $\div$ $\pm$Infinito  } & \multicolumn{1}{l|}{  NaN} &   \\ \cline{1-2}
\multicolumn{1}{|l|}{ $\pm$Infinito  $\cdot$ $\pm$0 } & \multicolumn{1}{l|}{  NaN} &   \\ \cline{1-2}
\end{tabular}
\end{table}

\paragraph{Almacenar -6,12510}\mbox{}\\
\begin{enumerate}    
  \item El bit 31 tomará el valor del signo del número. 
  \item Pasar a binario la mantisa decimal.
  \begin{itemize}
    \item $6=110_2$
    \item $0,125=0,001_2$
    \item $6,125=110,001_2$
  \end{itemize}
  \item Normalizar
  \begin{itemize}
    \item Desplazamiento a la derecha  -> Exponente negativo
    \item Desplazamiento a la izquierda -> Exponente positivo
    \item 1,10001 , exponente = 2
    \item 2 expresado en Exceso 127 es 129 $0000001_2$
  \end{itemize}
  \item Mantisa representada con bit implícito: 1,10001 -> 10001
   \item El número final es 1 10000001 100010000000000000000002 (Se agregan a la derecha los “0” necesarios para completar los 23 bits de la mantisa)
\end{enumerate}

\paragraph{Recuperar}\mbox{}\\\\
\begin{enumerate}    
  \item Se realizan los pasos en orden inverso
\end{enumerate}

\paragraph{Binario de Punto Flotante (IBM mainframe)}
\begin{itemize}
\item Base: 16
\item Representa: enteros con coma decimal positivos y negativos
\item Precision : imple 4 bytes, doble, 8 bytes, extendida 16 bytes.
\item estructura: S nnnnnnn dddddd
\begin{itemize}
    \item s = signo 1- 0+
    \item n = digitos de la caracteristica, en total son 7 bits se usan para calcular el exponente 
    \item C = E + 4016 donde E corresponde al exponente
    \item d =  mantisa normalizada -> 0,dddddd x $10^e_{16}$
  \end{itemize}
\end{itemize}

\paragraph{Almacenar - 321,54 10 -> Binario de punto flotante precisión simple}\mbox{}\\\\
\begin{enumerate}    
  \item $321,54_{10} = 141,8A3_{16}$
  \item $0,1418A3 x 10^3 _{16}$
  \item $C = E + 40_{16} = 3 + 40_{16} = 43_{16} en base 2 sería 100011_{2}$
  \item Agregamos el bit de signo: $1100011_{2}$ que en base 16 es $C3_{16}$
  \item Resultado final : $C31418A3_{16}$
\end{enumerate}

\paragraph{Ancho de paso }\mbox{}\\\\
distancia entre un flotante y su siguiente numero representable en el formato

\paragraph{Absorcion}\mbox{}\\\\
se da en las operaciones de suma y resta entre flotantes

\begin{itemize}    
  \item $A = 0,15A4 x 105_{16}$
  \item $B = 0,54F x 10^-2_{16}$
\end{itemize}

para poder operar entre flotantes debemos igualar los exponentes llevandolos al mayor de todos
\begin{itemize}    
  \item $A+B = 0.15A400 * 10^5 + 0,00...00 * 10^5 = 0,15A4000 * 10^5 _{16}$
\end{itemize}
lo minimo que se puede sumar es el ancho de paso del numero de mayor exponente

\section{Intel x86}
\section{Formato y Configuracion}

\subsection{Definición}
\begin{itemize}
\item Formato: Representación computacional
\item Configuración: Representación en una determinada base de un número en un formato
\end{itemize}

\subsection{Expansión y truncamiento}
\subsection{Definición}
\begin{itemize}
\item Expandir formato: Significa completar la representación computacional sin alterar el numero representado
en el mismo. 
\item Truncar formato: Descartar digitos de su representación sin alterar el número representado en el mismo.
\end{itemize}

\subsection{Formatos}

%Binario punto fijo sin signo
\subsubsection{Binario punto fijo sin signo}
\begin{itemize}
\item Base: 2
\item Representa: números enteros positivos
\item Máximo: $2^{n-1}_{10}$
\item Mínimo: 0
\end{itemize}

\paragraph{Almacenar}
\begin{enumerate}
\item Pasar el numero a base 2
\item completar con ceros a izquierda la capacidad del formato
\end{enumerate}
\paragraph{Recuperar}
\begin{enumerate}
\item Pasamos el numero de base 2 a la base deseada 
\end{enumerate}

%Binario de punto fijo con signo
\subsubsection{Binario de punto fijo con signo}
\begin{itemize}
\item Base: 2
\item Representa: Enteros positivos y negativos
\item Primer bit: reservado para el signo
\item Máximo: $2^{n-1}-1$
\item Mínimo: $-2^{n-1}$
\end{itemize}

\paragraph{Almacenar}
\begin{enumerate}
\item Pasar el numero a base 2
\item completar con ceros a izquierda la capacidad del formato
\item Si es un numero negativo hacerle el complemento a 2
\end{enumerate}
\paragraph{Recuperar}
\begin{enumerate}
\item Si el bit de signo es cero, se pasa de base 2 a base 10
\item Si el bit es 1, el numero es negativo, por lo que debemos complementarlo
\item Quitamos los ceros a la izquierda
\item Numero de base 2 a base 10
\item Colocamos el signo
\end{enumerate}

\paragraph{Expansión}
\begin{enumerate}
\item Se completa con el bit de signo a la izquierda
\end{enumerate}
\paragraph{Truncamiento}
\begin{enumerate}
\item Se extraen bits a la izquierda siempre y cuando no se esté alterando el bit de signo del número.
\end{enumerate}

%Empaquetado
\subsubsection{Empaquetado}
\begin{itemize}
\item Base: 16
\item Representa: Enteros positivos y negativos
\item Máximo: $10^{2n-1} -1$
\item Mínimo: $-10^{2n-1} +1$
\end{itemize}

\paragraph{Almacenar}
\begin{enumerate}
\item numero a base 10
\item Colocal cada digito decimal en un nibble dejando el ultimo nibble ya que en el mismo se almacena el signo.
\item Colocar en el ultimo nible el signo, CAFE = + , DF = -
\item Se rellena con 0 hasta alcanzar la cantidad de bytes usados
\end{enumerate}
\paragraph{Recuperar}
\begin{enumerate}
\item los pasos en orden inverso
\end{enumerate}

%Zoneado
\subsubsection{Zoneado}
\begin{itemize}
\item Base: 16
\item Representa: Enteros positivos y negativos
\item Máximo: $10^{n} -1$
\item Mínimo: $-10^{n} +1$
\end{itemize}

\paragraph{Almacenar}
\begin{enumerate}
\item numero a base 10
\item colocar cada uno de los digitos decimales en un nibble derecho
\item completar todos los nibbles de izquierda con F salvo el ultimo que se completa con el signo siguiendo las mismas reglas que para empaquetados.  CAFE = + , DF = -
\item se rellena con F0 hasta alcanzar la cantidad de bytes usados
\end{enumerate}
\paragraph{Recuperar}
\begin{enumerate}
\item los pasos en orden inverso
\end{enumerate}

\subsubsection{Binario de punto Flotante}
es la manera que tiene una arquitectura de representar a los numeros reales.
su notación cientifica se expresa de la siguiente manera $M x B^E$
M: Mantissa B: base E: Exponente

Un numero binario está normalizado si el digito de la izquierda del punto es igual a 1.

\paragraph{IEEE754}
\begin{itemize}
\item Precision Simple: signo: 1 bit exponente: 8 bits fraccion: 23 bits Exponente: Exceso 127
\item Precision Doble: signo: 1 bit exponente 11bits fraccion: 52 bits Exponente: Exceso 1023
\end{itemize}

\paragraph{Ancho de paso}\mbox{}\\\\
Marca cual es la distancia entre un flotante y su siguiente numero representable en el formato

\paragraph{Overflow}\mbox{}\\\\
el exponente excede el limite superior, tanto para mantisas positivas como para negatvias, dando lugar a +inf, - inf
\paragraph{Underflow}\mbox{}\\\\
el exponente excede el minimo valor permitido y caga en el intervalo (-inf, -0) y (+0,+inf)

\paragraph{Desnormalizados - Subnormales}\mbox{}\\\\
tienen como exponente al cero, y el bit implicito a la izquierda del punto binario, es ahora un cero implicito. la diferencia entre los desnormalizados y los normalizados es que, estos ultimos no permiten al cero como exponente. Los normalizados tienen 24 bits significativos, mientras que los normalizados poseen 23.

\begin{table}[h]
\begin{tabular}{lllll}
\cline{1-4}
\multicolumn{1}{|l|}{Normalizado} & \multicolumn{1}{l|}{+/-} & \multicolumn{1}{l|}{0<exp<max} &  \multicolumn{1}{l|}{cualquier patron de bits} &  \\ \cline{1-4}
\multicolumn{1}{|l|}{Desnormalizado} & \multicolumn{1}{l|}{+/-} & \multicolumn{1}{l|}{0} &  \multicolumn{1}{l|}{cualquier patron de bits != 0} &  \\ \cline{1-4}
\multicolumn{1}{|l|}{Cero} & \multicolumn{1}{l|}{+/-} & \multicolumn{1}{l|}{0} &  \multicolumn{1}{l|}{0} &  \\ \cline{1-4}
\multicolumn{1}{|l|}{Infinito} & \multicolumn{1}{l|}{+/-} & \multicolumn{1}{l|}{11...11} &  \multicolumn{1}{l|}{0} &  \\ \cline{1-4}
\multicolumn{1}{|l|}{NAN} & \multicolumn{1}{l|}{+/-} & \multicolumn{1}{l|}{11...11} &  \multicolumn{1}{l|}{Cualquier patron de bits != 0} &  \\ \cline{1-4}
\end{tabular}
\end{table}

\begin{itemize}
\item Infinito dividido Infinito = NAN
\item Cero + = 0 00000000 00000000000000000000000
\item Cero - = 1 00000000 00000000000000000000000
\item Infinito + = 0 11111111 00000000000000000000000
\item Infinito - = 1 11111111 00000000000000000000000
\item No normalizados/Subnormales (no se asume que haya que añadir un 1 al significando para obtener su valor).
\end{itemize}

\paragraph{Valores no numericos}\mbox{}\\\\
NaN (Not a number). 2 tipos, QNaN (quiet nan) y SNaN (signalling nan) Qnan = indeterminado Snan = operacion no valida

\begin{itemize}
\item Infinito dividido Infinito = NAN
\item qnan = 0 11111111 10000100000000000000000
\item snan = 1 11111111 00100010001001010101010
\end{itemize}

\begin{table}[h]
\begin{tabular}{lllll}
\cline{1-2}
\multicolumn{1}{|l|}{Operacion} & \multicolumn{1}{l|}{Resultado} &   \\ \cline{1-2}
\multicolumn{1}{|l|}{n $\pm$ infinito } & \multicolumn{1}{l|}{0} &   \\ \cline{1-2}
\multicolumn{1}{|l|}{$\pm$ infinito $\cdot$ $\pm$infinito } & \multicolumn{1}{l|}{$\pm$infinito} &   \\ \cline{1-2}
\multicolumn{1}{|l|}{ $\pm$n $\div$ 0   } & \multicolumn{1}{l|}{ $\pm$infinito } &   \\ \cline{1-2}
\multicolumn{1}{|l|}{ Infinito + Infinito   } & \multicolumn{1}{l|}{ Infinito } &   \\ \cline{1-2}
\multicolumn{1}{|l|}{Cualquier operación contra un NaN   } & \multicolumn{1}{l|}{ NaN } &   \\ \cline{1-2}
\multicolumn{1}{|l|}{ $\pm$0 $\div$ $\pm$0   } & \multicolumn{1}{l|}{ NaN } &   \\ \cline{1-2}
\multicolumn{1}{|l|}{ Infinito - Infinito   } & \multicolumn{1}{l|}{  NaN} &   \\ \cline{1-2}
\multicolumn{1}{|l|}{ $\pm$Infinito  $\div$ $\pm$Infinito  } & \multicolumn{1}{l|}{  NaN} &   \\ \cline{1-2}
\multicolumn{1}{|l|}{ $\pm$Infinito  $\cdot$ $\pm$0 } & \multicolumn{1}{l|}{  NaN} &   \\ \cline{1-2}
\end{tabular}
\end{table}

\paragraph{Almacenar -6,12510}\mbox{}\\
\begin{enumerate}    
  \item El bit 31 tomará el valor del signo del número. 
  \item Pasar a binario la mantisa decimal.
  \begin{itemize}
    \item $6=110_2$
    \item $0,125=0,001_2$
    \item $6,125=110,001_2$
  \end{itemize}
  \item Normalizar
  \begin{itemize}
    \item Desplazamiento a la derecha  -> Exponente negativo
    \item Desplazamiento a la izquierda -> Exponente positivo
    \item 1,10001 , exponente = 2
    \item 2 expresado en Exceso 127 es 129 $0000001_2$
  \end{itemize}
  \item Mantisa representada con bit implícito: 1,10001 -> 10001
   \item El número final es 1 10000001 100010000000000000000002 (Se agregan a la derecha los “0” necesarios para completar los 23 bits de la mantisa)
\end{enumerate}

\paragraph{Recuperar}\mbox{}\\\\
\begin{enumerate}    
  \item Se realizan los pasos en orden inverso
\end{enumerate}

\paragraph{Binario de Punto Flotante (IBM mainframe)}
\begin{itemize}
\item Base: 16
\item Representa: enteros con coma decimal positivos y negativos
\item Precision : imple 4 bytes, doble, 8 bytes, extendida 16 bytes.
\item estructura: S nnnnnnn dddddd
\begin{itemize}
    \item s = signo 1- 0+
    \item n = digitos de la caracteristica, en total son 7 bits se usan para calcular el exponente 
    \item C = E + 4016 donde E corresponde al exponente
    \item d =  mantisa normalizada -> 0,dddddd x $10^e_{16}$
  \end{itemize}
\end{itemize}

\paragraph{Almacenar - 321,54 10 -> Binario de punto flotante precisión simple}\mbox{}\\\\
\begin{enumerate}    
  \item $321,54_{10} = 141,8A3_{16}$
  \item $0,1418A3 x 10^3 _{16}$
  \item $C = E + 40_{16} = 3 + 40_{16} = 43_{16} en base 2 sería 100011_{2}$
  \item Agregamos el bit de signo: $1100011_{2}$ que en base 16 es $C3_{16}$
  \item Resultado final : $C31418A3_{16}$
\end{enumerate}

\paragraph{Ancho de paso }\mbox{}\\\\
distancia entre un flotante y su siguiente numero representable en el formato

\paragraph{Absorcion}\mbox{}\\\\
se da en las operaciones de suma y resta entre flotantes

\begin{itemize}    
  \item $A = 0,15A4 x 105_{16}$
  \item $B = 0,54F x 10^-2_{16}$
\end{itemize}

para poder operar entre flotantes debemos igualar los exponentes llevandolos al mayor de todos
\begin{itemize}    
  \item $A+B = 0.15A400 * 10^5 + 0,00...00 * 10^5 = 0,15A4000 * 10^5 _{16}$
\end{itemize}
lo minimo que se puede sumar es el ancho de paso del numero de mayor exponente

\section{Intel x86}
\section{Formato y Configuracion}

\subsection{Definición}
\begin{itemize}
\item Formato: Representación computacional
\item Configuración: Representación en una determinada base de un número en un formato
\end{itemize}

\subsection{Expansión y truncamiento}
\subsection{Definición}
\begin{itemize}
\item Expandir formato: Significa completar la representación computacional sin alterar el numero representado
en el mismo. 
\item Truncar formato: Descartar digitos de su representación sin alterar el número representado en el mismo.
\end{itemize}

\subsection{Formatos}

%Binario punto fijo sin signo
\subsubsection{Binario punto fijo sin signo}
\begin{itemize}
\item Base: 2
\item Representa: números enteros positivos
\item Máximo: $2^{n-1}_{10}$
\item Mínimo: 0
\end{itemize}

\paragraph{Almacenar}
\begin{enumerate}
\item Pasar el numero a base 2
\item completar con ceros a izquierda la capacidad del formato
\end{enumerate}
\paragraph{Recuperar}
\begin{enumerate}
\item Pasamos el numero de base 2 a la base deseada 
\end{enumerate}

%Binario de punto fijo con signo
\subsubsection{Binario de punto fijo con signo}
\begin{itemize}
\item Base: 2
\item Representa: Enteros positivos y negativos
\item Primer bit: reservado para el signo
\item Máximo: $2^{n-1}-1$
\item Mínimo: $-2^{n-1}$
\end{itemize}

\paragraph{Almacenar}
\begin{enumerate}
\item Pasar el numero a base 2
\item completar con ceros a izquierda la capacidad del formato
\item Si es un numero negativo hacerle el complemento a 2
\end{enumerate}
\paragraph{Recuperar}
\begin{enumerate}
\item Si el bit de signo es cero, se pasa de base 2 a base 10
\item Si el bit es 1, el numero es negativo, por lo que debemos complementarlo
\item Quitamos los ceros a la izquierda
\item Numero de base 2 a base 10
\item Colocamos el signo
\end{enumerate}

\paragraph{Expansión}
\begin{enumerate}
\item Se completa con el bit de signo a la izquierda
\end{enumerate}
\paragraph{Truncamiento}
\begin{enumerate}
\item Se extraen bits a la izquierda siempre y cuando no se esté alterando el bit de signo del número.
\end{enumerate}

%Empaquetado
\subsubsection{Empaquetado}
\begin{itemize}
\item Base: 16
\item Representa: Enteros positivos y negativos
\item Máximo: $10^{2n-1} -1$
\item Mínimo: $-10^{2n-1} +1$
\end{itemize}

\paragraph{Almacenar}
\begin{enumerate}
\item numero a base 10
\item Colocal cada digito decimal en un nibble dejando el ultimo nibble ya que en el mismo se almacena el signo.
\item Colocar en el ultimo nible el signo, CAFE = + , DF = -
\item Se rellena con 0 hasta alcanzar la cantidad de bytes usados
\end{enumerate}
\paragraph{Recuperar}
\begin{enumerate}
\item los pasos en orden inverso
\end{enumerate}

%Zoneado
\subsubsection{Zoneado}
\begin{itemize}
\item Base: 16
\item Representa: Enteros positivos y negativos
\item Máximo: $10^{n} -1$
\item Mínimo: $-10^{n} +1$
\end{itemize}

\paragraph{Almacenar}
\begin{enumerate}
\item numero a base 10
\item colocar cada uno de los digitos decimales en un nibble derecho
\item completar todos los nibbles de izquierda con F salvo el ultimo que se completa con el signo siguiendo las mismas reglas que para empaquetados.  CAFE = + , DF = -
\item se rellena con F0 hasta alcanzar la cantidad de bytes usados
\end{enumerate}
\paragraph{Recuperar}
\begin{enumerate}
\item los pasos en orden inverso
\end{enumerate}

\subsubsection{Binario de punto Flotante}
es la manera que tiene una arquitectura de representar a los numeros reales.
su notación cientifica se expresa de la siguiente manera $M x B^E$
M: Mantissa B: base E: Exponente

Un numero binario está normalizado si el digito de la izquierda del punto es igual a 1.

\paragraph{IEEE754}
\begin{itemize}
\item Precision Simple: signo: 1 bit exponente: 8 bits fraccion: 23 bits Exponente: Exceso 127
\item Precision Doble: signo: 1 bit exponente 11bits fraccion: 52 bits Exponente: Exceso 1023
\end{itemize}

\paragraph{Ancho de paso}\mbox{}\\\\
Marca cual es la distancia entre un flotante y su siguiente numero representable en el formato

\paragraph{Overflow}\mbox{}\\\\
el exponente excede el limite superior, tanto para mantisas positivas como para negatvias, dando lugar a +inf, - inf
\paragraph{Underflow}\mbox{}\\\\
el exponente excede el minimo valor permitido y caga en el intervalo (-inf, -0) y (+0,+inf)

\paragraph{Desnormalizados - Subnormales}\mbox{}\\\\
tienen como exponente al cero, y el bit implicito a la izquierda del punto binario, es ahora un cero implicito. la diferencia entre los desnormalizados y los normalizados es que, estos ultimos no permiten al cero como exponente. Los normalizados tienen 24 bits significativos, mientras que los normalizados poseen 23.

\begin{table}[h]
\begin{tabular}{lllll}
\cline{1-4}
\multicolumn{1}{|l|}{Normalizado} & \multicolumn{1}{l|}{+/-} & \multicolumn{1}{l|}{0<exp<max} &  \multicolumn{1}{l|}{cualquier patron de bits} &  \\ \cline{1-4}
\multicolumn{1}{|l|}{Desnormalizado} & \multicolumn{1}{l|}{+/-} & \multicolumn{1}{l|}{0} &  \multicolumn{1}{l|}{cualquier patron de bits != 0} &  \\ \cline{1-4}
\multicolumn{1}{|l|}{Cero} & \multicolumn{1}{l|}{+/-} & \multicolumn{1}{l|}{0} &  \multicolumn{1}{l|}{0} &  \\ \cline{1-4}
\multicolumn{1}{|l|}{Infinito} & \multicolumn{1}{l|}{+/-} & \multicolumn{1}{l|}{11...11} &  \multicolumn{1}{l|}{0} &  \\ \cline{1-4}
\multicolumn{1}{|l|}{NAN} & \multicolumn{1}{l|}{+/-} & \multicolumn{1}{l|}{11...11} &  \multicolumn{1}{l|}{Cualquier patron de bits != 0} &  \\ \cline{1-4}
\end{tabular}
\end{table}

\begin{itemize}
\item Infinito dividido Infinito = NAN
\item Cero + = 0 00000000 00000000000000000000000
\item Cero - = 1 00000000 00000000000000000000000
\item Infinito + = 0 11111111 00000000000000000000000
\item Infinito - = 1 11111111 00000000000000000000000
\item No normalizados/Subnormales (no se asume que haya que añadir un 1 al significando para obtener su valor).
\end{itemize}

\paragraph{Valores no numericos}\mbox{}\\\\
NaN (Not a number). 2 tipos, QNaN (quiet nan) y SNaN (signalling nan) Qnan = indeterminado Snan = operacion no valida

\begin{itemize}
\item Infinito dividido Infinito = NAN
\item qnan = 0 11111111 10000100000000000000000
\item snan = 1 11111111 00100010001001010101010
\end{itemize}

\begin{table}[h]
\begin{tabular}{lllll}
\cline{1-2}
\multicolumn{1}{|l|}{Operacion} & \multicolumn{1}{l|}{Resultado} &   \\ \cline{1-2}
\multicolumn{1}{|l|}{n $\pm$ infinito } & \multicolumn{1}{l|}{0} &   \\ \cline{1-2}
\multicolumn{1}{|l|}{$\pm$ infinito $\cdot$ $\pm$infinito } & \multicolumn{1}{l|}{$\pm$infinito} &   \\ \cline{1-2}
\multicolumn{1}{|l|}{ $\pm$n $\div$ 0   } & \multicolumn{1}{l|}{ $\pm$infinito } &   \\ \cline{1-2}
\multicolumn{1}{|l|}{ Infinito + Infinito   } & \multicolumn{1}{l|}{ Infinito } &   \\ \cline{1-2}
\multicolumn{1}{|l|}{Cualquier operación contra un NaN   } & \multicolumn{1}{l|}{ NaN } &   \\ \cline{1-2}
\multicolumn{1}{|l|}{ $\pm$0 $\div$ $\pm$0   } & \multicolumn{1}{l|}{ NaN } &   \\ \cline{1-2}
\multicolumn{1}{|l|}{ Infinito - Infinito   } & \multicolumn{1}{l|}{  NaN} &   \\ \cline{1-2}
\multicolumn{1}{|l|}{ $\pm$Infinito  $\div$ $\pm$Infinito  } & \multicolumn{1}{l|}{  NaN} &   \\ \cline{1-2}
\multicolumn{1}{|l|}{ $\pm$Infinito  $\cdot$ $\pm$0 } & \multicolumn{1}{l|}{  NaN} &   \\ \cline{1-2}
\end{tabular}
\end{table}

\paragraph{Almacenar -6,12510}\mbox{}\\
\begin{enumerate}    
  \item El bit 31 tomará el valor del signo del número. 
  \item Pasar a binario la mantisa decimal.
  \begin{itemize}
    \item $6=110_2$
    \item $0,125=0,001_2$
    \item $6,125=110,001_2$
  \end{itemize}
  \item Normalizar
  \begin{itemize}
    \item Desplazamiento a la derecha  -> Exponente negativo
    \item Desplazamiento a la izquierda -> Exponente positivo
    \item 1,10001 , exponente = 2
    \item 2 expresado en Exceso 127 es 129 $0000001_2$
  \end{itemize}
  \item Mantisa representada con bit implícito: 1,10001 -> 10001
   \item El número final es 1 10000001 100010000000000000000002 (Se agregan a la derecha los “0” necesarios para completar los 23 bits de la mantisa)
\end{enumerate}

\paragraph{Recuperar}\mbox{}\\\\
\begin{enumerate}    
  \item Se realizan los pasos en orden inverso
\end{enumerate}

\paragraph{Binario de Punto Flotante (IBM mainframe)}
\begin{itemize}
\item Base: 16
\item Representa: enteros con coma decimal positivos y negativos
\item Precision : imple 4 bytes, doble, 8 bytes, extendida 16 bytes.
\item estructura: S nnnnnnn dddddd
\begin{itemize}
    \item s = signo 1- 0+
    \item n = digitos de la caracteristica, en total son 7 bits se usan para calcular el exponente 
    \item C = E + 4016 donde E corresponde al exponente
    \item d =  mantisa normalizada -> 0,dddddd x $10^e_{16}$
  \end{itemize}
\end{itemize}

\paragraph{Almacenar - 321,54 10 -> Binario de punto flotante precisión simple}\mbox{}\\\\
\begin{enumerate}    
  \item $321,54_{10} = 141,8A3_{16}$
  \item $0,1418A3 x 10^3 _{16}$
  \item $C = E + 40_{16} = 3 + 40_{16} = 43_{16} en base 2 sería 100011_{2}$
  \item Agregamos el bit de signo: $1100011_{2}$ que en base 16 es $C3_{16}$
  \item Resultado final : $C31418A3_{16}$
\end{enumerate}

\paragraph{Ancho de paso }\mbox{}\\\\
distancia entre un flotante y su siguiente numero representable en el formato

\paragraph{Absorcion}\mbox{}\\\\
se da en las operaciones de suma y resta entre flotantes

\begin{itemize}    
  \item $A = 0,15A4 x 105_{16}$
  \item $B = 0,54F x 10^-2_{16}$
\end{itemize}

para poder operar entre flotantes debemos igualar los exponentes llevandolos al mayor de todos
\begin{itemize}    
  \item $A+B = 0.15A400 * 10^5 + 0,00...00 * 10^5 = 0,15A4000 * 10^5 _{16}$
\end{itemize}
lo minimo que se puede sumar es el ancho de paso del numero de mayor exponente

\section{Intel x86}
\section{Formato y Configuracion}

\subsection{Definición}
\begin{itemize}
\item Formato: Representación computacional
\item Configuración: Representación en una determinada base de un número en un formato
\end{itemize}

\subsection{Expansión y truncamiento}
\subsection{Definición}
\begin{itemize}
\item Expandir formato: Significa completar la representación computacional sin alterar el numero representado
en el mismo. 
\item Truncar formato: Descartar digitos de su representación sin alterar el número representado en el mismo.
\end{itemize}

\subsection{Formatos}

%Binario punto fijo sin signo
\subsubsection{Binario punto fijo sin signo}
\begin{itemize}
\item Base: 2
\item Representa: números enteros positivos
\item Máximo: $2^{n-1}_{10}$
\item Mínimo: 0
\end{itemize}

\paragraph{Almacenar}
\begin{enumerate}
\item Pasar el numero a base 2
\item completar con ceros a izquierda la capacidad del formato
\end{enumerate}
\paragraph{Recuperar}
\begin{enumerate}
\item Pasamos el numero de base 2 a la base deseada 
\end{enumerate}

%Binario de punto fijo con signo
\subsubsection{Binario de punto fijo con signo}
\begin{itemize}
\item Base: 2
\item Representa: Enteros positivos y negativos
\item Primer bit: reservado para el signo
\item Máximo: $2^{n-1}-1$
\item Mínimo: $-2^{n-1}$
\end{itemize}

\paragraph{Almacenar}
\begin{enumerate}
\item Pasar el numero a base 2
\item completar con ceros a izquierda la capacidad del formato
\item Si es un numero negativo hacerle el complemento a 2
\end{enumerate}
\paragraph{Recuperar}
\begin{enumerate}
\item Si el bit de signo es cero, se pasa de base 2 a base 10
\item Si el bit es 1, el numero es negativo, por lo que debemos complementarlo
\item Quitamos los ceros a la izquierda
\item Numero de base 2 a base 10
\item Colocamos el signo
\end{enumerate}

\paragraph{Expansión}
\begin{enumerate}
\item Se completa con el bit de signo a la izquierda
\end{enumerate}
\paragraph{Truncamiento}
\begin{enumerate}
\item Se extraen bits a la izquierda siempre y cuando no se esté alterando el bit de signo del número.
\end{enumerate}

%Empaquetado
\subsubsection{Empaquetado}
\begin{itemize}
\item Base: 16
\item Representa: Enteros positivos y negativos
\item Máximo: $10^{2n-1} -1$
\item Mínimo: $-10^{2n-1} +1$
\end{itemize}

\paragraph{Almacenar}
\begin{enumerate}
\item numero a base 10
\item Colocal cada digito decimal en un nibble dejando el ultimo nibble ya que en el mismo se almacena el signo.
\item Colocar en el ultimo nible el signo, CAFE = + , DF = -
\item Se rellena con 0 hasta alcanzar la cantidad de bytes usados
\end{enumerate}
\paragraph{Recuperar}
\begin{enumerate}
\item los pasos en orden inverso
\end{enumerate}

%Zoneado
\subsubsection{Zoneado}
\begin{itemize}
\item Base: 16
\item Representa: Enteros positivos y negativos
\item Máximo: $10^{n} -1$
\item Mínimo: $-10^{n} +1$
\end{itemize}

\paragraph{Almacenar}
\begin{enumerate}
\item numero a base 10
\item colocar cada uno de los digitos decimales en un nibble derecho
\item completar todos los nibbles de izquierda con F salvo el ultimo que se completa con el signo siguiendo las mismas reglas que para empaquetados.  CAFE = + , DF = -
\item se rellena con F0 hasta alcanzar la cantidad de bytes usados
\end{enumerate}
\paragraph{Recuperar}
\begin{enumerate}
\item los pasos en orden inverso
\end{enumerate}

\subsubsection{Binario de punto Flotante}
es la manera que tiene una arquitectura de representar a los numeros reales.
su notación cientifica se expresa de la siguiente manera $M x B^E$
M: Mantissa B: base E: Exponente

Un numero binario está normalizado si el digito de la izquierda del punto es igual a 1.

\paragraph{IEEE754}
\begin{itemize}
\item Precision Simple: signo: 1 bit exponente: 8 bits fraccion: 23 bits Exponente: Exceso 127
\item Precision Doble: signo: 1 bit exponente 11bits fraccion: 52 bits Exponente: Exceso 1023
\end{itemize}

\paragraph{Ancho de paso}\mbox{}\\\\
Marca cual es la distancia entre un flotante y su siguiente numero representable en el formato

\paragraph{Overflow}\mbox{}\\\\
el exponente excede el limite superior, tanto para mantisas positivas como para negatvias, dando lugar a +inf, - inf
\paragraph{Underflow}\mbox{}\\\\
el exponente excede el minimo valor permitido y caga en el intervalo (-inf, -0) y (+0,+inf)

\paragraph{Desnormalizados - Subnormales}\mbox{}\\\\
tienen como exponente al cero, y el bit implicito a la izquierda del punto binario, es ahora un cero implicito. la diferencia entre los desnormalizados y los normalizados es que, estos ultimos no permiten al cero como exponente. Los normalizados tienen 24 bits significativos, mientras que los normalizados poseen 23.

\begin{table}[h]
\begin{tabular}{lllll}
\cline{1-4}
\multicolumn{1}{|l|}{Normalizado} & \multicolumn{1}{l|}{+/-} & \multicolumn{1}{l|}{0<exp<max} &  \multicolumn{1}{l|}{cualquier patron de bits} &  \\ \cline{1-4}
\multicolumn{1}{|l|}{Desnormalizado} & \multicolumn{1}{l|}{+/-} & \multicolumn{1}{l|}{0} &  \multicolumn{1}{l|}{cualquier patron de bits != 0} &  \\ \cline{1-4}
\multicolumn{1}{|l|}{Cero} & \multicolumn{1}{l|}{+/-} & \multicolumn{1}{l|}{0} &  \multicolumn{1}{l|}{0} &  \\ \cline{1-4}
\multicolumn{1}{|l|}{Infinito} & \multicolumn{1}{l|}{+/-} & \multicolumn{1}{l|}{11...11} &  \multicolumn{1}{l|}{0} &  \\ \cline{1-4}
\multicolumn{1}{|l|}{NAN} & \multicolumn{1}{l|}{+/-} & \multicolumn{1}{l|}{11...11} &  \multicolumn{1}{l|}{Cualquier patron de bits != 0} &  \\ \cline{1-4}
\end{tabular}
\end{table}

\begin{itemize}
\item Infinito dividido Infinito = NAN
\item Cero + = 0 00000000 00000000000000000000000
\item Cero - = 1 00000000 00000000000000000000000
\item Infinito + = 0 11111111 00000000000000000000000
\item Infinito - = 1 11111111 00000000000000000000000
\item No normalizados/Subnormales (no se asume que haya que añadir un 1 al significando para obtener su valor).
\end{itemize}

\paragraph{Valores no numericos}\mbox{}\\\\
NaN (Not a number). 2 tipos, QNaN (quiet nan) y SNaN (signalling nan) Qnan = indeterminado Snan = operacion no valida

\begin{itemize}
\item Infinito dividido Infinito = NAN
\item qnan = 0 11111111 10000100000000000000000
\item snan = 1 11111111 00100010001001010101010
\end{itemize}

\begin{table}[h]
\begin{tabular}{lllll}
\cline{1-2}
\multicolumn{1}{|l|}{Operacion} & \multicolumn{1}{l|}{Resultado} &   \\ \cline{1-2}
\multicolumn{1}{|l|}{n $\pm$ infinito } & \multicolumn{1}{l|}{0} &   \\ \cline{1-2}
\multicolumn{1}{|l|}{$\pm$ infinito $\cdot$ $\pm$infinito } & \multicolumn{1}{l|}{$\pm$infinito} &   \\ \cline{1-2}
\multicolumn{1}{|l|}{ $\pm$n $\div$ 0   } & \multicolumn{1}{l|}{ $\pm$infinito } &   \\ \cline{1-2}
\multicolumn{1}{|l|}{ Infinito + Infinito   } & \multicolumn{1}{l|}{ Infinito } &   \\ \cline{1-2}
\multicolumn{1}{|l|}{Cualquier operación contra un NaN   } & \multicolumn{1}{l|}{ NaN } &   \\ \cline{1-2}
\multicolumn{1}{|l|}{ $\pm$0 $\div$ $\pm$0   } & \multicolumn{1}{l|}{ NaN } &   \\ \cline{1-2}
\multicolumn{1}{|l|}{ Infinito - Infinito   } & \multicolumn{1}{l|}{  NaN} &   \\ \cline{1-2}
\multicolumn{1}{|l|}{ $\pm$Infinito  $\div$ $\pm$Infinito  } & \multicolumn{1}{l|}{  NaN} &   \\ \cline{1-2}
\multicolumn{1}{|l|}{ $\pm$Infinito  $\cdot$ $\pm$0 } & \multicolumn{1}{l|}{  NaN} &   \\ \cline{1-2}
\end{tabular}
\end{table}

\paragraph{Almacenar -6,12510}\mbox{}\\
\begin{enumerate}    
  \item El bit 31 tomará el valor del signo del número. 
  \item Pasar a binario la mantisa decimal.
  \begin{itemize}
    \item $6=110_2$
    \item $0,125=0,001_2$
    \item $6,125=110,001_2$
  \end{itemize}
  \item Normalizar
  \begin{itemize}
    \item Desplazamiento a la derecha  -> Exponente negativo
    \item Desplazamiento a la izquierda -> Exponente positivo
    \item 1,10001 , exponente = 2
    \item 2 expresado en Exceso 127 es 129 $0000001_2$
  \end{itemize}
  \item Mantisa representada con bit implícito: 1,10001 -> 10001
   \item El número final es 1 10000001 100010000000000000000002 (Se agregan a la derecha los “0” necesarios para completar los 23 bits de la mantisa)
\end{enumerate}

\paragraph{Recuperar}\mbox{}\\\\
\begin{enumerate}    
  \item Se realizan los pasos en orden inverso
\end{enumerate}

\paragraph{Binario de Punto Flotante (IBM mainframe)}
\begin{itemize}
\item Base: 16
\item Representa: enteros con coma decimal positivos y negativos
\item Precision : imple 4 bytes, doble, 8 bytes, extendida 16 bytes.
\item estructura: S nnnnnnn dddddd
\begin{itemize}
    \item s = signo 1- 0+
    \item n = digitos de la caracteristica, en total son 7 bits se usan para calcular el exponente 
    \item C = E + 4016 donde E corresponde al exponente
    \item d =  mantisa normalizada -> 0,dddddd x $10^e_{16}$
  \end{itemize}
\end{itemize}

\paragraph{Almacenar - 321,54 10 -> Binario de punto flotante precisión simple}\mbox{}\\\\
\begin{enumerate}    
  \item $321,54_{10} = 141,8A3_{16}$
  \item $0,1418A3 x 10^3 _{16}$
  \item $C = E + 40_{16} = 3 + 40_{16} = 43_{16} en base 2 sería 100011_{2}$
  \item Agregamos el bit de signo: $1100011_{2}$ que en base 16 es $C3_{16}$
  \item Resultado final : $C31418A3_{16}$
\end{enumerate}

\paragraph{Ancho de paso }\mbox{}\\\\
distancia entre un flotante y su siguiente numero representable en el formato

\paragraph{Absorcion}\mbox{}\\\\
se da en las operaciones de suma y resta entre flotantes

\begin{itemize}    
  \item $A = 0,15A4 x 105_{16}$
  \item $B = 0,54F x 10^-2_{16}$
\end{itemize}

para poder operar entre flotantes debemos igualar los exponentes llevandolos al mayor de todos
\begin{itemize}    
  \item $A+B = 0.15A400 * 10^5 + 0,00...00 * 10^5 = 0,15A4000 * 10^5 _{16}$
\end{itemize}
lo minimo que se puede sumar es el ancho de paso del numero de mayor exponente

\subsection{Interrupciones}
\subsubsection{Definicion}
Mecanismos por los cuales otros modulos (E/S y memoria) interrumpen el normal procesamiento del CPU
\subsubsection{¿Para que existen?}
Para mejorar la eficiencia de procesamiento de un computador
\subsubsection{Clases de interrupciones}
\begin{itemize}
	\item programa
	\item  reloj
	\item e/s
	\item fallas de hardware
\end{itemize}

\subsubsection{Ciclo de instruccion}
\begin{itemize}
	\item Fetch instruction
	\item  Decode instruction
	\item Fetch operand
	\item Execute instruction
	\item Store result
	\item ----------------> interrupt breakpoint
	\item process interrupt
\end{itemize}

\subsubsection{Transferencia de control al S.O. (Handler)}
/*4pdf*/
	
\subsubsection{Procesamiento de interrupciones }
/*5pdf*/

/*6pdf ejemplo*/
	
\subsubsection{Multiples interrupciones}


\paragraph{Deshabilitar interrupciones (secuencia)}\mbox{}\\\\%%
/*7pdf*/
\paragraph{Priorizar interrupciones (anidadas)}\mbox{}\\\\%%
/*8pdf*/
	
\paragraph{Múltiples interrupciones - ejemplo}\mbox{}\\\\%%
Tres dispositivos de E/S
\begin{itemize}
	\item Línea de comunicación (Prioridad 1)
	\item  Disco (Prioridad 2)
	\item Impresora (Prioridad 3)
\end{itemize}
Eventos
\begin{itemize}
	\item T=10 Interrupción de Impresora
	\item T=15 Interrupción de línea de comunicación
	\item T=20 Interrupción de disco
\end{itemize}

/*10pdf*/

\subsection{MODULO DE E/S }

\subsubsection{Que hace}
Conecta a los periféricos con la CPU y la memoria a través del bus del sistema o switch central y permite la comunicación entre ellos
\subsubsection{Para que sirve}
Oculta detalles de timing, formatos y electro mecánica de los dispositivos periféricos
\subsubsection{Por que existe?}
\begin{itemize}
\item Amplia variedad de periféricos con distintos métodos de operación
\item La tasa de transferencia de los periféricos es generalmente mucho más lenta que la de la memoria y procesador
\item Los periféricos usan distintos formatos de datos y tamaños de palabra
\end{itemize}

/*4pdf*/ 230 stalling

\subsubsection{Interface interna - bus del sistema}
\begin{itemize}
\item Datos
\item Direcciones
\item Control
\end{itemize}

\subsubsection{Interface externa - perifericos}
\begin{itemize}
\item Datos
\item Estado
\item Control
\end{itemize}


\subsubsection{Funciones}
\paragraph{Control and Timing}\mbox{}\\\\%%
ontrola flujo de tráfico entre CPU/Memoria y periféricos
\paragraph{Comunicación con el procesador}\mbox{}\\\\%%
Decodificación de comandos: The I/O module accepts commands from the processor, typically sent as signals on the control bus. For example, an I/O module for a disk drive might accept the following commands: READ SECTOR, WRITE SECTOR, SEEK track number, and SCAN record ID. The latter two commands each include a parameter that is sent on the data bus
\begin{itemize}
	\item Datos: Data are exchanged between the processor and the I/O module over the data bus.
	\item Información de estado: Because peripherals are so slow, it is important to know the	status of the I/O module. For example, if an I/O module is asked to send data to the processor (read), it may not be ready to do so because it is still working on the previous I/O command. This fact can be reported with a status signal.	Common status signals are BUSY and READY. There may also be signals to report various error conditions.
	\item Reconocimiento de direcciones: Just as each word of memory has an address, so does each I/O device. Thus, an I/O module must recognize one unique address for each peripheral it controls.
\end{itemize}
\paragraph{Comunicación con el dispositivo}\mbox{}\\\\%%
\begin{itemize}
\item Comandos
\item Información de estado
\item Datos
\end{itemize}
\paragraph{Buffering de datos}\mbox{}\\\\%%
\paragraph{Detección de errores}\mbox{}\\\\%%

\subsubsection{The control of the transfer of data from an external device to the processor might involve the following sequence of steps}
\begin{enumerate}
\item The processor interrogates the I/O module to check the status of the attached device.
\item The I/O module returns the device status.
\item If the device is operational and ready to transmit, the processor requests the transfer of data, by means of a command to the IO module.
\item  The I/O module obtains a unit of data (e.g., 8 or 16 bits) from the external device.
\item The data are transferred from the I/O module to the processor.
\end{enumerate}

/*Diagrama I/O pag 234 Will, 7pdf */

\subsubsection{Tecnicas para operaciones de E/S}
\begin{enumerate}
\item  E/S programada
\item  E/S manejada por interrupciones
\item  Acceso directo a memoria (DMA)
\end{enumerate}

/*pag 237*/

When large volumes of data are to be moved, a more efficient technique is
required: direct memory access (DMA).
DMA involves an additional module on the system bus. The DMA module
(Figure 7.12) is capable of mimicking the processor and, indeed, of taking over control
of the system from the processor. It needs to do this to transfer data to and from
memory over the system bus. For this purpose, the DMA module must use the bus
only when the processor does not need it, or it must force the processor to suspend
operation temporarily. The latter technique is more common and is referred to as
cycle stealing, because the DMA module in effect steals a bus cycle.
When the processor wishes to read or write a block of data, it issues a command
to the DMA module, by sending to the DMA module the following information:
\begin{enumerate}
\item  Whether a read or write is requested, using the read or write control line
between the processor and the DMA module.
\item  The address of the I/O device involved, communicated on the data lines.
\item  The starting location in memory to read from or write to, communicated on
the data lines and stored by the DMA module in its address register.
\item  The number of words to be read or written, again communicated via the data
lines and stored in the data count register.
\end{enumerate}
/*pag 249*/ 
The processor then continues with other work. It has delegated this I/O operation
to the DMA module. The DMA module transfers the entire block of data,
one word at a time, directly to or from memory, without going through the processor.
When the transfer is complete, the DMA module sends an interrupt signal to
the processor. Thus, the processor is involved only at the beginning and end of the
transfer

/*  250 Figure 7.13 DMA and Interrupt Breakpoints during an Instruction Cycle*/

/* 251 Figure 7.14 Alternative DMA Configurations*/

The DMA mechanism can be configured in a variety of ways. Some possibilities
are shown in Figure 7.14. In the first example, all modules share the same system
bus. The DMA module, acting as a surrogate processor, uses programmed I/O to
exchange data between memory and an I/O module through the DMA module. This
configuration, while it may be inexpensive, is clearly inefficient. As with processor-
controlled programmed I/O, each transfer of a word consumes two bus cycles
The number of required bus cycles can be cut substantially by integrating the
DMA and I/O functions. As Figure 7.14b indicates, this means that there is a path
between the DMA module and one or more I/O modules that does not include
the system bus. The DMA logic may actually be a part of an I/O module, or it may
be a separate module that controls one or more I/O modules. This concept can
be taken one step further by connecting I/O modules to the DMA module using
an I/O bus (Figure 7.14c). This reduces the number of I/O interfaces in the DMA
module to one and provides for an easily expandable configuration. In both of
these cases (Figures 7.14b and c), the system bus that the DMA module shares with
the processor and memory is used by the DMA module only to exchange data with
memory. The exchange of data between the DMA and I/O modules takes place off
the system bus.

\subsubsection{I /O Channels and Processors}
The Evolution of the I/O Function
As computer systems have evolved, there has been a pattern of increasing complexity
and sophistication of individual components. Nowhere is this more evident than
in the I/O function. We have already seen part of that evolution. The evolutionary
steps can be summarized as follows:
\begin{enumerate}
\item The CPU directly controls a peripheral device. This is seen in simple
microprocessor-controlled devices.
\item A controller or I/O module is added. The CPU uses programmed I/O without
interrupts. With this step, the CPU becomes somewhat divorced from the specific
details of external device interfaces.
\item  The same configuration as in step 2 is used, but now interrupts are employed.
The CPU need not spend time waiting for an I/O operation to be performed,
thus increasing efficiency.
\item The I/O module is given direct access to memory via DMA. It can now move
a block of data to or from memory without involving the CPU, except at the
beginning and end of the transfer.
\item  The I/O module is enhanced to become a processor in its own right, with a
specialized instruction set tailored for I/O. The CPU directs the I/O processor
to execute an I/O program in memory. The I/O processor fetches and executes
these instructions without CPU intervention. This allows the CPU to specify a
sequence of I/O activities and to be interrupted only when the entire sequence
has been performed.
\item The I/O module has a local memory of its own and is, in fact, a computer in its
own right. With this architecture, a large set of I/O devices can be controlled,
with minimal CPU involvement. A common use for such an architecture has
been to control communication with interactive terminals. The I/O processor
takes care of most of the tasks involved in controlling the terminals.
\end{enumerate}

\subsubsection{Characteristics of I/O Channels}
	The I/O channel represents an extension of the DMA concept. An I/O channel
	has the ability to execute I/O instructions, which gives it complete control over
	I/O operations. In a computer system with such devices, the CPU does not execute
	I/O instructions. Such instructions are stored in main memory to be executed by a
	special-purpose processor in the I/O channel itself. Thus, the CPU initiates an I/O
	transfer by instructing the I/O channel to execute a program in memory. The program
	will specify the device or devices, the area or areas of memory for storage,
	priority, and actions to be taken for certain error conditions. The I/O channel follows
	these instructions and controls the data transfer

	Two types of I/O channels are common, as illustrated in Figure 7.18. A
	selector channel controls multiple high-speed devices and, at any one time, is
	dedicated to the transfer of data with one of those devices. Thus, the I/O channel
	selects one device and effects the data transfer. Each device, or a small set of
	devices, is handled by a controller, or I/O module, that is much like the I/O modules
	we have been discussing. Thus, the I/O channel serves in place of the CPU
	in controlling these I/O controllers. A multiplexor channel can handle I/O with
	multiple devices at the same time. For low-speed devices, a byte multiplexor
	accepts or transmits characters as fast as possible to multiple devices. For example,
	the resultant character stream from three devices with different rates and individual
	streams A1A2A3A4 c, B1B2B3B4 c, and C1C2C3C4 c might be A1B1C1A2C2A3B2C3A4, and so on.
	

	/*263*/

\subsection{ADMINISTRACION DE MEMORIA }

\subsubsection{Sistema Operativo}
Software que administra los recursos del computador, provee servicios y controla la ejecución de otros programas
\subsubsection{Algunos servicios que provee}
\begin{itemize}
\item Schedule de procesos
\item Administración de memoria
\item Monitor
\item Parte residente del Sistema Operativo
\end{itemize}
/*284*/
In a uniprogramming system, main memory is divided into two parts: one part for
the OS (resident monitor) and one part for the program currently being executed.
In a multiprogramming system, the “user” part of memory is subdivided to accommodate
multiple processes. The task of subdivision is carried out dynamically by the
OS and is known as memory management.

\subsubsection{Administración de memoria simple}
\subsubsection{Sistema con uniprogramación}

Se divide la memoria en dos partes
Monitor del S.O.
Programa en ejecución en ese momento
\paragraph{Ventajas:}\mbox{}\\\\%%

Simplicidad

\paragraph{DESVentajas:}\mbox{}\\\\%%
\begin{itemize}
\item Desperdicio de memoria
\item Desaprovechamiento de los recursos del computador
\end{itemize}
Administración de memoria simple 
/*281*/

simple batch systems Early processors were very expensive, and therefore it
was important to maximize processor utilization. The wasted time due to scheduling
and setup time was unacceptable.
To improve utilization, simple batch operating systems were developed. With
such a system, also called a monitor, the user no longer has direct access to the processor.
Rather, the user submits the job on cards or tape to a computer operator,
who batches the jobs together sequentially and places the entire batch on an input
device, for use by the monitor.
To understand how this scheme works, let us look at it from two points of
view: that of the monitor and that of the processor. From the point of view of the
monitor, the monitor controls the sequence of events. For this to be so, much of the
monitor must always be in main memory and available for execution (Figure 8.3).
That portion is referred to as the resident monitor. The rest of the monitor consists
of utilities and common functions that are loaded as subroutines to the user program
at the beginning of any job that requires them. The monitor reads in jobs one
at a time from the input device (typically a card reader or magnetic tape drive). As it
is read in, the current job is placed in the user program area, and control is passed to
this job. When the job is completed, it returns control to the monitor, which immediately
reads in the next job. The results of each job are printed out for delivery to
the user.
Now consider this sequence from the point of view of the processor. At a certain
point in time, the processor is executing instructions from the portion of main memory
containing the monitor. These instructions cause the next job to be read in to
another portion of main memory. Once a job has been read in, the processor will
encounter in the monitor a branch instruction that instructs the processor to continue
execution at the start of the user program. The processor will then execute
the instruction in the user’s program until it encounters an ending or error condition.
Either event causes the processor to fetch its next instruction from the monitor
program. Thus the phrase “control is passed to a job” simply means that the processor
is now fetching and executing instructions in a user program, and “control is
returned to the monitor” means that the processor is now fetching and executing
instructions from the monitor program.
It should be clear that the monitor handles the scheduling problem. A batch of
jobs is queued up, and jobs are executed as rapidly as possible, with no intervening
idle time.
How about the job setup time? The monitor handles this as well. With each
job, instructions are included in a job control language (JCL). This is a special type
of programming language used to provide instructions to the monitor.


\subsubsection{Multiprogramming}
\begin{itemize}
\item Varios procesos de usuario en ejecución a lavez
\item Se divide la memoria de usuario entre los procesos en ejecución
\item Se comparte el tiempo de procesador entre los procesos en ejecución ( timeslice
\item Condiciones de finalización:
	\begin{itemize}
	\item Termina el trabajo
	\item Se detecta un error y se cancela
	\item Requiere una operación de E/S (suspensión)
	\item Termina el timeslice (suspención
	\end{itemize}
\end{itemize}

The simplest scheme for partitioning available memory is to use fixed-
sizepartitions, as shown in Figure 8.13. Note that, although the partitions are of fixed size, they need not be of equal size. When a process is brought into memory, it is placed in the
smallest available partition that will hold it.
Even with the use of unequal fixed-size partitions, there will be wasted memory.
In most cases, a process will not require exactly as much memory as provided by the partition.
A more efficient approach is to use variable-size partitions. When a process is
brought into memory, it is allocated exactly as much memory as it requires and no more.
As this example shows, this method starts out well, but eventually it leads to a
situation in which there are a lot of small holes in memory. As time goes on, memory
becomes more and more fragmented, and memory utilization declines. One
technique for overcoming this problem is compaction: From time to time, the OS
shifts the processes in memory to place all the free memory together in one block.
This is a time-consuming procedure, wasteful of processor time.

\subsubsection{Memory management: partitioning}
\begin{itemize}
\item Sistema con multiprogramación
\item La memoria de usuario se divide en particiones detamaño fijo:
	\begin{itemize}
	\item Iguales
	\item Distintas
	\end{itemize}
\end{itemize}
Ventajas:
\begin{itemize}
\item Permite compartir la memoria entre varios procesos
\end{itemize}
Desventajas:
\begin{itemize}
\item Desperdicio de memoria
\item Fragmentación interna (dentro de una partición)
\item Fragmentación externa (particiones no usadas)
\end{itemize}

\subsubsection{Memory management: swapping}
\begin{itemize}
\item Sistema con multiprogramación
\item Swapping
\item La memoria de usuario se divide en particiones de tamaño variable
\item Compactación para eliminar la fragmentación
\item Se usa un recurso de hardware (registro de reasignación) para la realocación
\item Realocación dinámica en tiempo de ejecución
\end{itemize}
Ventajas:
\begin{itemize}
\item Permite compartir la memoria entre varios procesos
\item Elimina el desperdicio por fragmentación interna.
\item Con la compactación se elimina además la fragmentación externa
\end{itemize}
Desventajas:
\begin{itemize}
\item La tarea de compactación es costosa
\end{itemize}

\subsubsection{Memory management: paging}
Both unequal fixed-size and variable-size
partitions are inefficient in the use of memory.
Suppose, however, that memory is partitioned into equal fixed-size
chunks that are relatively small, and that each process is also divided into small fixed-
size chunks of some size. Then the chunks of a program, known as pages, could be assigned to
available chunks of memory, known as frames, or page frames. At most, then, the
wasted space in memory for that process is a fraction of the last page.

\paragraph{Administración de memoria paginada simple}\mbox{}\\\\%%
\begin{itemize}
\item Sistema con multiprogramación
\item Se divide el address space del proceso en partes iguales (páginas)
\item Se divide la memoria principal en partes iguales ( frames)
\item Hay una tabla de páginas por proceso
\item Hay una lista de frames disponibles
\item Se cargan a memoria las páginas del proceso en los frames disponibles (no es necesario que sean contiguos)
\item Las direcciones lógicas se ven como número de página y un offset
\item Se traducen las direcciones lógicas en físicas con soporte del hardware
\item La paginación es transparente para el programador
\end{itemize}
Ventajas:
\begin{itemize}
\item Permite compartir la memoria entre varios procesos
\item Minimiza la fragmentación interna (solo existe dentro de la última página de cada proceso)
\item Elimina la fragmentación externa
\end{itemize}
Desventajas
\begin{itemize}
\item Se requiere subir todas las páginas del proceso a memoria
\item Se requieren estructuras de datos adicionales para mantener información de páginas y frames
\end{itemize}

\paragraph{Administración de memoria paginada simple}\mbox{}\\\\%%
\begin{itemize}
\item Sistema con multiprogramación
\item Solo se cargan las páginas necesarias para la ejecución de un proceso
\item Cuando se quiere acceder a una posición de memoria de una página no cargada se produce un page fault
\item El page fault dispara una interrupción por hardware atendida por el sistema operativo
\item Se levanta la página solicitada desde memoria secundaria (memoria virtual)
\item Algoritmos para reemplazo de páginas
\item Thrashing : el CPU pasa más tiempo reemplazando páginas que ejecutando instrucciones
\end{itemize}
Ventajas
\begin{itemize}
\item No es necesario cargar todas las páginas de un proceso a la vez
\item Maximiza el uso de la memoria al permitir cargar más procesos a la vez
\item Un proceso puede ocupar más memoria de la efectivamente instalada en el computador
\end{itemize}
Desventajas
\begin{itemize}
\item Mayor complejidad por la necesidad de implementar el reemplazo de páginas
\end{itemize}

\subsubsection{Administración de memoria por segmentación}
\begin{itemize}
\item Sistemas con multiprogramación
\item Generalmente visible al programador
\item La memoria del programa se ve como un conjunto de segmentos (múltiples espacios de direcciones)
\item Los segmentos son de tamaño variable y dinámico
\item El sistema operativo administra una tabla de segmentos por proceso
\item Permite separar datos e instrucciones
\item Permite dar privilegios y protección de memoria como por ej . lectura, escritura, ejecución. ( segmentation faults como mecanismos de excepción de hardware para accesos indebidos)
\item Las referencias a memoria se forman con un número de segmento y un offset dentro de él. Con ayuda de hardware (MMU \item Memory Management Unit ) se hacen las traducciones de las direcciones lógicas a físicas
\item Se pueden usar para implementar memoria virtual (solo se suben a
memoria física algunos segmentos por proceso)
\end{itemize}

Ventajas:
\begin{itemize}
\item Simplifica el manejo de estructuras de datos con crecimiento
\item Permite compartir información entre procesos dentro de un segmento
\item Permite aplicar protección/privilegios sobre un segmento fácilmente
\end{itemize}
Desventajas:
\begin{itemize}
\item Fragmentación externa en la memoria principal por no poder alojar un segmento
\item Hardware más complejo que memoria paginada para la traducción de direcciones
\end{itemize}

\subsubsection{Address Spaces}
The x86 includes hardware for both segmentation and paging. Both mechanisms can
be disabled, allowing the user to choose from four distinct views of memory:

\begin{itemize}
\item Unsegmented unpaged memory: In this case, the virtual address is the same
as the physical address. This is useful, for example, in low- complexity, high- performance controller applications.
\item Unsegmented paged memory: Here memory is viewed as a paged linear
address space. Protection and management of memory is done via paging.
This is favored by some operating systems (e.g., Berkeley UNIX).
\item Segmented unpaged memory: Here memory is viewed as a collection of
logical address spaces. The advantage of this view over a paged approach is
that it affords protection down to the level of a single byte, if necessary. Furthermore,
unlike paging, it guarantees that the translation table needed (the segment table) is on-chip
when the segment is in memory. Hence, segmented unpaged memory results in predictable access times.
\item Segmented paged memory: Segmentation is used to define logical memory
partitions subject to access control, and paging is used to manage the allocation
of memory within the partitions. Operating systems such as UNIX System favor this view.
\end{itemize}
%%%%%%%%%%%%%%%%%%%%


\section{Intel x86}
\section{Formato y Configuracion}

\subsection{Definición}
\begin{itemize}
\item Formato: Representación computacional
\item Configuración: Representación en una determinada base de un número en un formato
\end{itemize}

\subsection{Expansión y truncamiento}
\subsection{Definición}
\begin{itemize}
\item Expandir formato: Significa completar la representación computacional sin alterar el numero representado
en el mismo. 
\item Truncar formato: Descartar digitos de su representación sin alterar el número representado en el mismo.
\end{itemize}

\subsection{Formatos}

%Binario punto fijo sin signo
\subsubsection{Binario punto fijo sin signo}
\begin{itemize}
\item Base: 2
\item Representa: números enteros positivos
\item Máximo: $2^{n-1}_{10}$
\item Mínimo: 0
\end{itemize}

\paragraph{Almacenar}
\begin{enumerate}
\item Pasar el numero a base 2
\item completar con ceros a izquierda la capacidad del formato
\end{enumerate}
\paragraph{Recuperar}
\begin{enumerate}
\item Pasamos el numero de base 2 a la base deseada 
\end{enumerate}

%Binario de punto fijo con signo
\subsubsection{Binario de punto fijo con signo}
\begin{itemize}
\item Base: 2
\item Representa: Enteros positivos y negativos
\item Primer bit: reservado para el signo
\item Máximo: $2^{n-1}-1$
\item Mínimo: $-2^{n-1}$
\end{itemize}

\paragraph{Almacenar}
\begin{enumerate}
\item Pasar el numero a base 2
\item completar con ceros a izquierda la capacidad del formato
\item Si es un numero negativo hacerle el complemento a 2
\end{enumerate}
\paragraph{Recuperar}
\begin{enumerate}
\item Si el bit de signo es cero, se pasa de base 2 a base 10
\item Si el bit es 1, el numero es negativo, por lo que debemos complementarlo
\item Quitamos los ceros a la izquierda
\item Numero de base 2 a base 10
\item Colocamos el signo
\end{enumerate}

\paragraph{Expansión}
\begin{enumerate}
\item Se completa con el bit de signo a la izquierda
\end{enumerate}
\paragraph{Truncamiento}
\begin{enumerate}
\item Se extraen bits a la izquierda siempre y cuando no se esté alterando el bit de signo del número.
\end{enumerate}

%Empaquetado
\subsubsection{Empaquetado}
\begin{itemize}
\item Base: 16
\item Representa: Enteros positivos y negativos
\item Máximo: $10^{2n-1} -1$
\item Mínimo: $-10^{2n-1} +1$
\end{itemize}

\paragraph{Almacenar}
\begin{enumerate}
\item numero a base 10
\item Colocal cada digito decimal en un nibble dejando el ultimo nibble ya que en el mismo se almacena el signo.
\item Colocar en el ultimo nible el signo, CAFE = + , DF = -
\item Se rellena con 0 hasta alcanzar la cantidad de bytes usados
\end{enumerate}
\paragraph{Recuperar}
\begin{enumerate}
\item los pasos en orden inverso
\end{enumerate}

%Zoneado
\subsubsection{Zoneado}
\begin{itemize}
\item Base: 16
\item Representa: Enteros positivos y negativos
\item Máximo: $10^{n} -1$
\item Mínimo: $-10^{n} +1$
\end{itemize}

\paragraph{Almacenar}
\begin{enumerate}
\item numero a base 10
\item colocar cada uno de los digitos decimales en un nibble derecho
\item completar todos los nibbles de izquierda con F salvo el ultimo que se completa con el signo siguiendo las mismas reglas que para empaquetados.  CAFE = + , DF = -
\item se rellena con F0 hasta alcanzar la cantidad de bytes usados
\end{enumerate}
\paragraph{Recuperar}
\begin{enumerate}
\item los pasos en orden inverso
\end{enumerate}

\subsubsection{Binario de punto Flotante}
es la manera que tiene una arquitectura de representar a los numeros reales.
su notación cientifica se expresa de la siguiente manera $M x B^E$
M: Mantissa B: base E: Exponente

Un numero binario está normalizado si el digito de la izquierda del punto es igual a 1.

\paragraph{IEEE754}
\begin{itemize}
\item Precision Simple: signo: 1 bit exponente: 8 bits fraccion: 23 bits Exponente: Exceso 127
\item Precision Doble: signo: 1 bit exponente 11bits fraccion: 52 bits Exponente: Exceso 1023
\end{itemize}

\paragraph{Ancho de paso}\mbox{}\\\\
Marca cual es la distancia entre un flotante y su siguiente numero representable en el formato

\paragraph{Overflow}\mbox{}\\\\
el exponente excede el limite superior, tanto para mantisas positivas como para negatvias, dando lugar a +inf, - inf
\paragraph{Underflow}\mbox{}\\\\
el exponente excede el minimo valor permitido y caga en el intervalo (-inf, -0) y (+0,+inf)

\paragraph{Desnormalizados - Subnormales}\mbox{}\\\\
tienen como exponente al cero, y el bit implicito a la izquierda del punto binario, es ahora un cero implicito. la diferencia entre los desnormalizados y los normalizados es que, estos ultimos no permiten al cero como exponente. Los normalizados tienen 24 bits significativos, mientras que los normalizados poseen 23.

\begin{table}[h]
\begin{tabular}{lllll}
\cline{1-4}
\multicolumn{1}{|l|}{Normalizado} & \multicolumn{1}{l|}{+/-} & \multicolumn{1}{l|}{0<exp<max} &  \multicolumn{1}{l|}{cualquier patron de bits} &  \\ \cline{1-4}
\multicolumn{1}{|l|}{Desnormalizado} & \multicolumn{1}{l|}{+/-} & \multicolumn{1}{l|}{0} &  \multicolumn{1}{l|}{cualquier patron de bits != 0} &  \\ \cline{1-4}
\multicolumn{1}{|l|}{Cero} & \multicolumn{1}{l|}{+/-} & \multicolumn{1}{l|}{0} &  \multicolumn{1}{l|}{0} &  \\ \cline{1-4}
\multicolumn{1}{|l|}{Infinito} & \multicolumn{1}{l|}{+/-} & \multicolumn{1}{l|}{11...11} &  \multicolumn{1}{l|}{0} &  \\ \cline{1-4}
\multicolumn{1}{|l|}{NAN} & \multicolumn{1}{l|}{+/-} & \multicolumn{1}{l|}{11...11} &  \multicolumn{1}{l|}{Cualquier patron de bits != 0} &  \\ \cline{1-4}
\end{tabular}
\end{table}

\begin{itemize}
\item Infinito dividido Infinito = NAN
\item Cero + = 0 00000000 00000000000000000000000
\item Cero - = 1 00000000 00000000000000000000000
\item Infinito + = 0 11111111 00000000000000000000000
\item Infinito - = 1 11111111 00000000000000000000000
\item No normalizados/Subnormales (no se asume que haya que añadir un 1 al significando para obtener su valor).
\end{itemize}

\paragraph{Valores no numericos}\mbox{}\\\\
NaN (Not a number). 2 tipos, QNaN (quiet nan) y SNaN (signalling nan) Qnan = indeterminado Snan = operacion no valida

\begin{itemize}
\item Infinito dividido Infinito = NAN
\item qnan = 0 11111111 10000100000000000000000
\item snan = 1 11111111 00100010001001010101010
\end{itemize}

\begin{table}[h]
\begin{tabular}{lllll}
\cline{1-2}
\multicolumn{1}{|l|}{Operacion} & \multicolumn{1}{l|}{Resultado} &   \\ \cline{1-2}
\multicolumn{1}{|l|}{n $\pm$ infinito } & \multicolumn{1}{l|}{0} &   \\ \cline{1-2}
\multicolumn{1}{|l|}{$\pm$ infinito $\cdot$ $\pm$infinito } & \multicolumn{1}{l|}{$\pm$infinito} &   \\ \cline{1-2}
\multicolumn{1}{|l|}{ $\pm$n $\div$ 0   } & \multicolumn{1}{l|}{ $\pm$infinito } &   \\ \cline{1-2}
\multicolumn{1}{|l|}{ Infinito + Infinito   } & \multicolumn{1}{l|}{ Infinito } &   \\ \cline{1-2}
\multicolumn{1}{|l|}{Cualquier operación contra un NaN   } & \multicolumn{1}{l|}{ NaN } &   \\ \cline{1-2}
\multicolumn{1}{|l|}{ $\pm$0 $\div$ $\pm$0   } & \multicolumn{1}{l|}{ NaN } &   \\ \cline{1-2}
\multicolumn{1}{|l|}{ Infinito - Infinito   } & \multicolumn{1}{l|}{  NaN} &   \\ \cline{1-2}
\multicolumn{1}{|l|}{ $\pm$Infinito  $\div$ $\pm$Infinito  } & \multicolumn{1}{l|}{  NaN} &   \\ \cline{1-2}
\multicolumn{1}{|l|}{ $\pm$Infinito  $\cdot$ $\pm$0 } & \multicolumn{1}{l|}{  NaN} &   \\ \cline{1-2}
\end{tabular}
\end{table}

\paragraph{Almacenar -6,12510}\mbox{}\\
\begin{enumerate}    
  \item El bit 31 tomará el valor del signo del número. 
  \item Pasar a binario la mantisa decimal.
  \begin{itemize}
    \item $6=110_2$
    \item $0,125=0,001_2$
    \item $6,125=110,001_2$
  \end{itemize}
  \item Normalizar
  \begin{itemize}
    \item Desplazamiento a la derecha  -> Exponente negativo
    \item Desplazamiento a la izquierda -> Exponente positivo
    \item 1,10001 , exponente = 2
    \item 2 expresado en Exceso 127 es 129 $0000001_2$
  \end{itemize}
  \item Mantisa representada con bit implícito: 1,10001 -> 10001
   \item El número final es 1 10000001 100010000000000000000002 (Se agregan a la derecha los “0” necesarios para completar los 23 bits de la mantisa)
\end{enumerate}

\paragraph{Recuperar}\mbox{}\\\\
\begin{enumerate}    
  \item Se realizan los pasos en orden inverso
\end{enumerate}

\paragraph{Binario de Punto Flotante (IBM mainframe)}
\begin{itemize}
\item Base: 16
\item Representa: enteros con coma decimal positivos y negativos
\item Precision : imple 4 bytes, doble, 8 bytes, extendida 16 bytes.
\item estructura: S nnnnnnn dddddd
\begin{itemize}
    \item s = signo 1- 0+
    \item n = digitos de la caracteristica, en total son 7 bits se usan para calcular el exponente 
    \item C = E + 4016 donde E corresponde al exponente
    \item d =  mantisa normalizada -> 0,dddddd x $10^e_{16}$
  \end{itemize}
\end{itemize}

\paragraph{Almacenar - 321,54 10 -> Binario de punto flotante precisión simple}\mbox{}\\\\
\begin{enumerate}    
  \item $321,54_{10} = 141,8A3_{16}$
  \item $0,1418A3 x 10^3 _{16}$
  \item $C = E + 40_{16} = 3 + 40_{16} = 43_{16} en base 2 sería 100011_{2}$
  \item Agregamos el bit de signo: $1100011_{2}$ que en base 16 es $C3_{16}$
  \item Resultado final : $C31418A3_{16}$
\end{enumerate}

\paragraph{Ancho de paso }\mbox{}\\\\
distancia entre un flotante y su siguiente numero representable en el formato

\paragraph{Absorcion}\mbox{}\\\\
se da en las operaciones de suma y resta entre flotantes

\begin{itemize}    
  \item $A = 0,15A4 x 105_{16}$
  \item $B = 0,54F x 10^-2_{16}$
\end{itemize}

para poder operar entre flotantes debemos igualar los exponentes llevandolos al mayor de todos
\begin{itemize}    
  \item $A+B = 0.15A400 * 10^5 + 0,00...00 * 10^5 = 0,15A4000 * 10^5 _{16}$
\end{itemize}
lo minimo que se puede sumar es el ancho de paso del numero de mayor exponente

\subsection{Interrupciones}
\subsubsection{Definicion}
Mecanismos por los cuales otros modulos (E/S y memoria) interrumpen el normal procesamiento del CPU
\subsubsection{¿Para que existen?}
Para mejorar la eficiencia de procesamiento de un computador
\subsubsection{Clases de interrupciones}
\begin{itemize}
	\item programa
	\item  reloj
	\item e/s
	\item fallas de hardware
\end{itemize}

\subsubsection{Ciclo de instruccion}
\begin{itemize}
	\item Fetch instruction
	\item  Decode instruction
	\item Fetch operand
	\item Execute instruction
	\item Store result
	\item ----------------> interrupt breakpoint
	\item process interrupt
\end{itemize}

\subsubsection{Transferencia de control al S.O. (Handler)}
/*4pdf*/
	
\subsubsection{Procesamiento de interrupciones }
/*5pdf*/

/*6pdf ejemplo*/
	
\subsubsection{Multiples interrupciones}


\paragraph{Deshabilitar interrupciones (secuencia)}\mbox{}\\\\%%
/*7pdf*/
\paragraph{Priorizar interrupciones (anidadas)}\mbox{}\\\\%%
/*8pdf*/
	
\paragraph{Múltiples interrupciones - ejemplo}\mbox{}\\\\%%
Tres dispositivos de E/S
\begin{itemize}
	\item Línea de comunicación (Prioridad 1)
	\item  Disco (Prioridad 2)
	\item Impresora (Prioridad 3)
\end{itemize}
Eventos
\begin{itemize}
	\item T=10 Interrupción de Impresora
	\item T=15 Interrupción de línea de comunicación
	\item T=20 Interrupción de disco
\end{itemize}

/*10pdf*/

\subsection{MODULO DE E/S }

\subsubsection{Que hace}
Conecta a los periféricos con la CPU y la memoria a través del bus del sistema o switch central y permite la comunicación entre ellos
\subsubsection{Para que sirve}
Oculta detalles de timing, formatos y electro mecánica de los dispositivos periféricos
\subsubsection{Por que existe?}
\begin{itemize}
\item Amplia variedad de periféricos con distintos métodos de operación
\item La tasa de transferencia de los periféricos es generalmente mucho más lenta que la de la memoria y procesador
\item Los periféricos usan distintos formatos de datos y tamaños de palabra
\end{itemize}

/*4pdf*/ 230 stalling

\subsubsection{Interface interna - bus del sistema}
\begin{itemize}
\item Datos
\item Direcciones
\item Control
\end{itemize}

\subsubsection{Interface externa - perifericos}
\begin{itemize}
\item Datos
\item Estado
\item Control
\end{itemize}


\subsubsection{Funciones}
\paragraph{Control and Timing}\mbox{}\\\\%%
ontrola flujo de tráfico entre CPU/Memoria y periféricos
\paragraph{Comunicación con el procesador}\mbox{}\\\\%%
Decodificación de comandos: The I/O module accepts commands from the processor, typically sent as signals on the control bus. For example, an I/O module for a disk drive might accept the following commands: READ SECTOR, WRITE SECTOR, SEEK track number, and SCAN record ID. The latter two commands each include a parameter that is sent on the data bus
\begin{itemize}
	\item Datos: Data are exchanged between the processor and the I/O module over the data bus.
	\item Información de estado: Because peripherals are so slow, it is important to know the	status of the I/O module. For example, if an I/O module is asked to send data to the processor (read), it may not be ready to do so because it is still working on the previous I/O command. This fact can be reported with a status signal.	Common status signals are BUSY and READY. There may also be signals to report various error conditions.
	\item Reconocimiento de direcciones: Just as each word of memory has an address, so does each I/O device. Thus, an I/O module must recognize one unique address for each peripheral it controls.
\end{itemize}
\paragraph{Comunicación con el dispositivo}\mbox{}\\\\%%
\begin{itemize}
\item Comandos
\item Información de estado
\item Datos
\end{itemize}
\paragraph{Buffering de datos}\mbox{}\\\\%%
\paragraph{Detección de errores}\mbox{}\\\\%%

\subsubsection{The control of the transfer of data from an external device to the processor might involve the following sequence of steps}
\begin{enumerate}
\item The processor interrogates the I/O module to check the status of the attached device.
\item The I/O module returns the device status.
\item If the device is operational and ready to transmit, the processor requests the transfer of data, by means of a command to the IO module.
\item  The I/O module obtains a unit of data (e.g., 8 or 16 bits) from the external device.
\item The data are transferred from the I/O module to the processor.
\end{enumerate}

/*Diagrama I/O pag 234 Will, 7pdf */

\subsubsection{Tecnicas para operaciones de E/S}
\begin{enumerate}
\item  E/S programada
\item  E/S manejada por interrupciones
\item  Acceso directo a memoria (DMA)
\end{enumerate}

/*pag 237*/

When large volumes of data are to be moved, a more efficient technique is
required: direct memory access (DMA).
DMA involves an additional module on the system bus. The DMA module
(Figure 7.12) is capable of mimicking the processor and, indeed, of taking over control
of the system from the processor. It needs to do this to transfer data to and from
memory over the system bus. For this purpose, the DMA module must use the bus
only when the processor does not need it, or it must force the processor to suspend
operation temporarily. The latter technique is more common and is referred to as
cycle stealing, because the DMA module in effect steals a bus cycle.
When the processor wishes to read or write a block of data, it issues a command
to the DMA module, by sending to the DMA module the following information:
\begin{enumerate}
\item  Whether a read or write is requested, using the read or write control line
between the processor and the DMA module.
\item  The address of the I/O device involved, communicated on the data lines.
\item  The starting location in memory to read from or write to, communicated on
the data lines and stored by the DMA module in its address register.
\item  The number of words to be read or written, again communicated via the data
lines and stored in the data count register.
\end{enumerate}
/*pag 249*/ 
The processor then continues with other work. It has delegated this I/O operation
to the DMA module. The DMA module transfers the entire block of data,
one word at a time, directly to or from memory, without going through the processor.
When the transfer is complete, the DMA module sends an interrupt signal to
the processor. Thus, the processor is involved only at the beginning and end of the
transfer

/*  250 Figure 7.13 DMA and Interrupt Breakpoints during an Instruction Cycle*/

/* 251 Figure 7.14 Alternative DMA Configurations*/

The DMA mechanism can be configured in a variety of ways. Some possibilities
are shown in Figure 7.14. In the first example, all modules share the same system
bus. The DMA module, acting as a surrogate processor, uses programmed I/O to
exchange data between memory and an I/O module through the DMA module. This
configuration, while it may be inexpensive, is clearly inefficient. As with processor-
controlled programmed I/O, each transfer of a word consumes two bus cycles
The number of required bus cycles can be cut substantially by integrating the
DMA and I/O functions. As Figure 7.14b indicates, this means that there is a path
between the DMA module and one or more I/O modules that does not include
the system bus. The DMA logic may actually be a part of an I/O module, or it may
be a separate module that controls one or more I/O modules. This concept can
be taken one step further by connecting I/O modules to the DMA module using
an I/O bus (Figure 7.14c). This reduces the number of I/O interfaces in the DMA
module to one and provides for an easily expandable configuration. In both of
these cases (Figures 7.14b and c), the system bus that the DMA module shares with
the processor and memory is used by the DMA module only to exchange data with
memory. The exchange of data between the DMA and I/O modules takes place off
the system bus.

\subsubsection{I /O Channels and Processors}
The Evolution of the I/O Function
As computer systems have evolved, there has been a pattern of increasing complexity
and sophistication of individual components. Nowhere is this more evident than
in the I/O function. We have already seen part of that evolution. The evolutionary
steps can be summarized as follows:
\begin{enumerate}
\item The CPU directly controls a peripheral device. This is seen in simple
microprocessor-controlled devices.
\item A controller or I/O module is added. The CPU uses programmed I/O without
interrupts. With this step, the CPU becomes somewhat divorced from the specific
details of external device interfaces.
\item  The same configuration as in step 2 is used, but now interrupts are employed.
The CPU need not spend time waiting for an I/O operation to be performed,
thus increasing efficiency.
\item The I/O module is given direct access to memory via DMA. It can now move
a block of data to or from memory without involving the CPU, except at the
beginning and end of the transfer.
\item  The I/O module is enhanced to become a processor in its own right, with a
specialized instruction set tailored for I/O. The CPU directs the I/O processor
to execute an I/O program in memory. The I/O processor fetches and executes
these instructions without CPU intervention. This allows the CPU to specify a
sequence of I/O activities and to be interrupted only when the entire sequence
has been performed.
\item The I/O module has a local memory of its own and is, in fact, a computer in its
own right. With this architecture, a large set of I/O devices can be controlled,
with minimal CPU involvement. A common use for such an architecture has
been to control communication with interactive terminals. The I/O processor
takes care of most of the tasks involved in controlling the terminals.
\end{enumerate}

\subsubsection{Characteristics of I/O Channels}
	The I/O channel represents an extension of the DMA concept. An I/O channel
	has the ability to execute I/O instructions, which gives it complete control over
	I/O operations. In a computer system with such devices, the CPU does not execute
	I/O instructions. Such instructions are stored in main memory to be executed by a
	special-purpose processor in the I/O channel itself. Thus, the CPU initiates an I/O
	transfer by instructing the I/O channel to execute a program in memory. The program
	will specify the device or devices, the area or areas of memory for storage,
	priority, and actions to be taken for certain error conditions. The I/O channel follows
	these instructions and controls the data transfer

	Two types of I/O channels are common, as illustrated in Figure 7.18. A
	selector channel controls multiple high-speed devices and, at any one time, is
	dedicated to the transfer of data with one of those devices. Thus, the I/O channel
	selects one device and effects the data transfer. Each device, or a small set of
	devices, is handled by a controller, or I/O module, that is much like the I/O modules
	we have been discussing. Thus, the I/O channel serves in place of the CPU
	in controlling these I/O controllers. A multiplexor channel can handle I/O with
	multiple devices at the same time. For low-speed devices, a byte multiplexor
	accepts or transmits characters as fast as possible to multiple devices. For example,
	the resultant character stream from three devices with different rates and individual
	streams A1A2A3A4 c, B1B2B3B4 c, and C1C2C3C4 c might be A1B1C1A2C2A3B2C3A4, and so on.
	

	/*263*/



%%%%%%%%%%%%%%%%%%%%%%%%%%%%%%%%%%%%%%%%%%
% Academic Title Page
% LaTeX Template
% Version 2.0 (17/7/17)
%
% This template was downloaded from:
% http://www.LaTeXTemplates.com
%
% Original author:
% WikiBooks (LaTeX - Title Creation) with modifications by:
% Vel (vel@latextemplates.com)
%
% License:
% CC BY-NC-SA 3.0 (http://creativecommons.org/licenses/by-nc-sa/3.0/)
% 
% Instructions for using this template:
% This title page is capable of being compiled as is. This is not useful for 
% including it in another document. To do this, you have two options: 
%
% 1) Copy/paste everything between \begin{document} and \end{document} 
% starting at \begin{titlepage} and paste this into another LaTeX file where you 
% want your title page.
% OR
% 2) Remove everything outside the \begin{titlepage} and \end{titlepage}, rename
% this file and move it to the same directory as the LaTeX file you wish to add it to. 
% Then add \input{./<new filename>.tex} to your LaTeX file where you want your
% title page.
%
%%%%%%%%%%%%%%%%%%%%%%%%%%%%%%%%%%%%%%%%%

%----------------------------------------------------------------------------------------
%	PACKAGES AND OTHER DOCUMENT CONFIGURATIONS
%----------------------------------------------------------------------------------------


\documentclass[11pt]{article}
\usepackage{geometry}
\usepackage{graphicx}
\usepackage{url}
\usepackage[utf8]{inputenc} % Required for inputting international characters
\usepackage[T1]{fontenc} % Output font encoding for international characters

\usepackage{mathpazo} % Palatino font

\begin{document}
\setcounter{secnumdepth}{5}
%----------------------------------------------------------------------------------------
%	TITLE PAGE
%----------------------------------------------------------------------------------------

\begin{titlepage} % Suppresses displaying the page number on the title page and the subsequent page counts as page 1
	\newcommand{\HRule}{\rule{\linewidth}{0.5mm}} % Defines a new command for horizontal lines, change thickness here
	
	\center % Centre everything on the page
	
	%------------------------------------------------
	%	Headings
	%------------------------------------------------
	
	\textsc{\LARGE Universidad de Buenos Aires}\\[1.5cm] % Main heading such as the name of your university/college
	
	\textsc{\Large Facultad de Ingeniería}\\[0.5cm] % Major heading such as course name
	
	\textsc{\large Resumen}\\[0.5cm] % Minor heading such as course title
	
	%------------------------------------------------
	%	Title
	%------------------------------------------------
	
	\HRule\\[0.4cm]
	
	{\huge\bfseries Organización del Computador \newline 75.03 \& 95.57}\\[0.4cm] % Title of your document
	
	\HRule\\[1.5cm]
	
	%------------------------------------------------
	%	Author(s)
	%------------------------------------------------
	
	%\begin{minipage}{0.4\textwidth}
	%	\begin{flushleft}
	%		\large
	%		\textit{Autor}\\
	%		\textsc{Anzu} % Your name
	%	\end{flushleft}
	%\end{minipage}
	%~
	%\begin{minipage}{0.4\textwidth}
	%	\begin{flushright}
	%		\large
	%		\textit{Supervisor}\\
	%		\textsc{Anzu} % Supervisor's name
	%	\end{flushright}
	%\end{minipage}
	
	% If you don't want a supervisor, uncomment the two lines below and comment the code above
	{\large\textit{Autor}}\\
	\textsc{Anzu} % Your name
	
	%------------------------------------------------
	%	Date
	%------------------------------------------------
	
	\vfill\vfill\vfill % Position the date 3/4 down the remaining page
	
	{\large\today} % Date, change the \today to a set date if you want to be precise
	
	%------------------------------------------------
	%	Logo
	%------------------------------------------------
	
	%\vfill\vfill
	%\includegraphics[width=0.2\textwidth]{placeholder.jpg}\\[1cm] % Include a department/university logo - this will require the graphicx package
	 
	%----------------------------------------------------------------------------------------
	
	\vfill % Push the date up 1/4 of the remaining page
	
\end{titlepage}

%----------------------------------------------------------------------------------------

\tableofcontents

\newpage

\section{Intel x86}
\section{Formato y Configuracion}

\subsection{Definición}
\begin{itemize}
\item Formato: Representación computacional
\item Configuración: Representación en una determinada base de un número en un formato
\end{itemize}

\subsection{Expansión y truncamiento}
\subsection{Definición}
\begin{itemize}
\item Expandir formato: Significa completar la representación computacional sin alterar el numero representado
en el mismo. 
\item Truncar formato: Descartar digitos de su representación sin alterar el número representado en el mismo.
\end{itemize}

\subsection{Formatos}

%Binario punto fijo sin signo
\subsubsection{Binario punto fijo sin signo}
\begin{itemize}
\item Base: 2
\item Representa: números enteros positivos
\item Máximo: $2^{n-1}_{10}$
\item Mínimo: 0
\end{itemize}

\paragraph{Almacenar}
\begin{enumerate}
\item Pasar el numero a base 2
\item completar con ceros a izquierda la capacidad del formato
\end{enumerate}
\paragraph{Recuperar}
\begin{enumerate}
\item Pasamos el numero de base 2 a la base deseada 
\end{enumerate}

%Binario de punto fijo con signo
\subsubsection{Binario de punto fijo con signo}
\begin{itemize}
\item Base: 2
\item Representa: Enteros positivos y negativos
\item Primer bit: reservado para el signo
\item Máximo: $2^{n-1}-1$
\item Mínimo: $-2^{n-1}$
\end{itemize}

\paragraph{Almacenar}
\begin{enumerate}
\item Pasar el numero a base 2
\item completar con ceros a izquierda la capacidad del formato
\item Si es un numero negativo hacerle el complemento a 2
\end{enumerate}
\paragraph{Recuperar}
\begin{enumerate}
\item Si el bit de signo es cero, se pasa de base 2 a base 10
\item Si el bit es 1, el numero es negativo, por lo que debemos complementarlo
\item Quitamos los ceros a la izquierda
\item Numero de base 2 a base 10
\item Colocamos el signo
\end{enumerate}

\paragraph{Expansión}
\begin{enumerate}
\item Se completa con el bit de signo a la izquierda
\end{enumerate}
\paragraph{Truncamiento}
\begin{enumerate}
\item Se extraen bits a la izquierda siempre y cuando no se esté alterando el bit de signo del número.
\end{enumerate}

%Empaquetado
\subsubsection{Empaquetado}
\begin{itemize}
\item Base: 16
\item Representa: Enteros positivos y negativos
\item Máximo: $10^{2n-1} -1$
\item Mínimo: $-10^{2n-1} +1$
\end{itemize}

\paragraph{Almacenar}
\begin{enumerate}
\item numero a base 10
\item Colocal cada digito decimal en un nibble dejando el ultimo nibble ya que en el mismo se almacena el signo.
\item Colocar en el ultimo nible el signo, CAFE = + , DF = -
\item Se rellena con 0 hasta alcanzar la cantidad de bytes usados
\end{enumerate}
\paragraph{Recuperar}
\begin{enumerate}
\item los pasos en orden inverso
\end{enumerate}

%Zoneado
\subsubsection{Zoneado}
\begin{itemize}
\item Base: 16
\item Representa: Enteros positivos y negativos
\item Máximo: $10^{n} -1$
\item Mínimo: $-10^{n} +1$
\end{itemize}

\paragraph{Almacenar}
\begin{enumerate}
\item numero a base 10
\item colocar cada uno de los digitos decimales en un nibble derecho
\item completar todos los nibbles de izquierda con F salvo el ultimo que se completa con el signo siguiendo las mismas reglas que para empaquetados.  CAFE = + , DF = -
\item se rellena con F0 hasta alcanzar la cantidad de bytes usados
\end{enumerate}
\paragraph{Recuperar}
\begin{enumerate}
\item los pasos en orden inverso
\end{enumerate}

\subsubsection{Binario de punto Flotante}
es la manera que tiene una arquitectura de representar a los numeros reales.
su notación cientifica se expresa de la siguiente manera $M x B^E$
M: Mantissa B: base E: Exponente

Un numero binario está normalizado si el digito de la izquierda del punto es igual a 1.

\paragraph{IEEE754}
\begin{itemize}
\item Precision Simple: signo: 1 bit exponente: 8 bits fraccion: 23 bits Exponente: Exceso 127
\item Precision Doble: signo: 1 bit exponente 11bits fraccion: 52 bits Exponente: Exceso 1023
\end{itemize}

\paragraph{Ancho de paso}\mbox{}\\\\
Marca cual es la distancia entre un flotante y su siguiente numero representable en el formato

\paragraph{Overflow}\mbox{}\\\\
el exponente excede el limite superior, tanto para mantisas positivas como para negatvias, dando lugar a +inf, - inf
\paragraph{Underflow}\mbox{}\\\\
el exponente excede el minimo valor permitido y caga en el intervalo (-inf, -0) y (+0,+inf)

\paragraph{Desnormalizados - Subnormales}\mbox{}\\\\
tienen como exponente al cero, y el bit implicito a la izquierda del punto binario, es ahora un cero implicito. la diferencia entre los desnormalizados y los normalizados es que, estos ultimos no permiten al cero como exponente. Los normalizados tienen 24 bits significativos, mientras que los normalizados poseen 23.

\begin{table}[h]
\begin{tabular}{lllll}
\cline{1-4}
\multicolumn{1}{|l|}{Normalizado} & \multicolumn{1}{l|}{+/-} & \multicolumn{1}{l|}{0<exp<max} &  \multicolumn{1}{l|}{cualquier patron de bits} &  \\ \cline{1-4}
\multicolumn{1}{|l|}{Desnormalizado} & \multicolumn{1}{l|}{+/-} & \multicolumn{1}{l|}{0} &  \multicolumn{1}{l|}{cualquier patron de bits != 0} &  \\ \cline{1-4}
\multicolumn{1}{|l|}{Cero} & \multicolumn{1}{l|}{+/-} & \multicolumn{1}{l|}{0} &  \multicolumn{1}{l|}{0} &  \\ \cline{1-4}
\multicolumn{1}{|l|}{Infinito} & \multicolumn{1}{l|}{+/-} & \multicolumn{1}{l|}{11...11} &  \multicolumn{1}{l|}{0} &  \\ \cline{1-4}
\multicolumn{1}{|l|}{NAN} & \multicolumn{1}{l|}{+/-} & \multicolumn{1}{l|}{11...11} &  \multicolumn{1}{l|}{Cualquier patron de bits != 0} &  \\ \cline{1-4}
\end{tabular}
\end{table}

\begin{itemize}
\item Infinito dividido Infinito = NAN
\item Cero + = 0 00000000 00000000000000000000000
\item Cero - = 1 00000000 00000000000000000000000
\item Infinito + = 0 11111111 00000000000000000000000
\item Infinito - = 1 11111111 00000000000000000000000
\item No normalizados/Subnormales (no se asume que haya que añadir un 1 al significando para obtener su valor).
\end{itemize}

\paragraph{Valores no numericos}\mbox{}\\\\
NaN (Not a number). 2 tipos, QNaN (quiet nan) y SNaN (signalling nan) Qnan = indeterminado Snan = operacion no valida

\begin{itemize}
\item Infinito dividido Infinito = NAN
\item qnan = 0 11111111 10000100000000000000000
\item snan = 1 11111111 00100010001001010101010
\end{itemize}

\begin{table}[h]
\begin{tabular}{lllll}
\cline{1-2}
\multicolumn{1}{|l|}{Operacion} & \multicolumn{1}{l|}{Resultado} &   \\ \cline{1-2}
\multicolumn{1}{|l|}{n $\pm$ infinito } & \multicolumn{1}{l|}{0} &   \\ \cline{1-2}
\multicolumn{1}{|l|}{$\pm$ infinito $\cdot$ $\pm$infinito } & \multicolumn{1}{l|}{$\pm$infinito} &   \\ \cline{1-2}
\multicolumn{1}{|l|}{ $\pm$n $\div$ 0   } & \multicolumn{1}{l|}{ $\pm$infinito } &   \\ \cline{1-2}
\multicolumn{1}{|l|}{ Infinito + Infinito   } & \multicolumn{1}{l|}{ Infinito } &   \\ \cline{1-2}
\multicolumn{1}{|l|}{Cualquier operación contra un NaN   } & \multicolumn{1}{l|}{ NaN } &   \\ \cline{1-2}
\multicolumn{1}{|l|}{ $\pm$0 $\div$ $\pm$0   } & \multicolumn{1}{l|}{ NaN } &   \\ \cline{1-2}
\multicolumn{1}{|l|}{ Infinito - Infinito   } & \multicolumn{1}{l|}{  NaN} &   \\ \cline{1-2}
\multicolumn{1}{|l|}{ $\pm$Infinito  $\div$ $\pm$Infinito  } & \multicolumn{1}{l|}{  NaN} &   \\ \cline{1-2}
\multicolumn{1}{|l|}{ $\pm$Infinito  $\cdot$ $\pm$0 } & \multicolumn{1}{l|}{  NaN} &   \\ \cline{1-2}
\end{tabular}
\end{table}

\paragraph{Almacenar -6,12510}\mbox{}\\
\begin{enumerate}    
  \item El bit 31 tomará el valor del signo del número. 
  \item Pasar a binario la mantisa decimal.
  \begin{itemize}
    \item $6=110_2$
    \item $0,125=0,001_2$
    \item $6,125=110,001_2$
  \end{itemize}
  \item Normalizar
  \begin{itemize}
    \item Desplazamiento a la derecha  -> Exponente negativo
    \item Desplazamiento a la izquierda -> Exponente positivo
    \item 1,10001 , exponente = 2
    \item 2 expresado en Exceso 127 es 129 $0000001_2$
  \end{itemize}
  \item Mantisa representada con bit implícito: 1,10001 -> 10001
   \item El número final es 1 10000001 100010000000000000000002 (Se agregan a la derecha los “0” necesarios para completar los 23 bits de la mantisa)
\end{enumerate}

\paragraph{Recuperar}\mbox{}\\\\
\begin{enumerate}    
  \item Se realizan los pasos en orden inverso
\end{enumerate}

\paragraph{Binario de Punto Flotante (IBM mainframe)}
\begin{itemize}
\item Base: 16
\item Representa: enteros con coma decimal positivos y negativos
\item Precision : imple 4 bytes, doble, 8 bytes, extendida 16 bytes.
\item estructura: S nnnnnnn dddddd
\begin{itemize}
    \item s = signo 1- 0+
    \item n = digitos de la caracteristica, en total son 7 bits se usan para calcular el exponente 
    \item C = E + 4016 donde E corresponde al exponente
    \item d =  mantisa normalizada -> 0,dddddd x $10^e_{16}$
  \end{itemize}
\end{itemize}

\paragraph{Almacenar - 321,54 10 -> Binario de punto flotante precisión simple}\mbox{}\\\\
\begin{enumerate}    
  \item $321,54_{10} = 141,8A3_{16}$
  \item $0,1418A3 x 10^3 _{16}$
  \item $C = E + 40_{16} = 3 + 40_{16} = 43_{16} en base 2 sería 100011_{2}$
  \item Agregamos el bit de signo: $1100011_{2}$ que en base 16 es $C3_{16}$
  \item Resultado final : $C31418A3_{16}$
\end{enumerate}

\paragraph{Ancho de paso }\mbox{}\\\\
distancia entre un flotante y su siguiente numero representable en el formato

\paragraph{Absorcion}\mbox{}\\\\
se da en las operaciones de suma y resta entre flotantes

\begin{itemize}    
  \item $A = 0,15A4 x 105_{16}$
  \item $B = 0,54F x 10^-2_{16}$
\end{itemize}

para poder operar entre flotantes debemos igualar los exponentes llevandolos al mayor de todos
\begin{itemize}    
  \item $A+B = 0.15A400 * 10^5 + 0,00...00 * 10^5 = 0,15A4000 * 10^5 _{16}$
\end{itemize}
lo minimo que se puede sumar es el ancho de paso del numero de mayor exponente

\subsection{Interrupciones}
\subsubsection{Definicion}
Mecanismos por los cuales otros modulos (E/S y memoria) interrumpen el normal procesamiento del CPU
\subsubsection{¿Para que existen?}
Para mejorar la eficiencia de procesamiento de un computador
\subsubsection{Clases de interrupciones}
\begin{itemize}
	\item programa
	\item  reloj
	\item e/s
	\item fallas de hardware
\end{itemize}

\subsubsection{Ciclo de instruccion}
\begin{itemize}
	\item Fetch instruction
	\item  Decode instruction
	\item Fetch operand
	\item Execute instruction
	\item Store result
	\item ----------------> interrupt breakpoint
	\item process interrupt
\end{itemize}

\subsubsection{Transferencia de control al S.O. (Handler)}
/*4pdf*/
	
\subsubsection{Procesamiento de interrupciones }
/*5pdf*/

/*6pdf ejemplo*/
	
\subsubsection{Multiples interrupciones}


\paragraph{Deshabilitar interrupciones (secuencia)}\mbox{}\\\\%%
/*7pdf*/
\paragraph{Priorizar interrupciones (anidadas)}\mbox{}\\\\%%
/*8pdf*/
	
\paragraph{Múltiples interrupciones - ejemplo}\mbox{}\\\\%%
Tres dispositivos de E/S
\begin{itemize}
	\item Línea de comunicación (Prioridad 1)
	\item  Disco (Prioridad 2)
	\item Impresora (Prioridad 3)
\end{itemize}
Eventos
\begin{itemize}
	\item T=10 Interrupción de Impresora
	\item T=15 Interrupción de línea de comunicación
	\item T=20 Interrupción de disco
\end{itemize}

/*10pdf*/


\section{Intel x86}
\section{Formato y Configuracion}

\subsection{Definición}
\begin{itemize}
\item Formato: Representación computacional
\item Configuración: Representación en una determinada base de un número en un formato
\end{itemize}

\subsection{Expansión y truncamiento}
\subsection{Definición}
\begin{itemize}
\item Expandir formato: Significa completar la representación computacional sin alterar el numero representado
en el mismo. 
\item Truncar formato: Descartar digitos de su representación sin alterar el número representado en el mismo.
\end{itemize}

\subsection{Formatos}

%Binario punto fijo sin signo
\subsubsection{Binario punto fijo sin signo}
\begin{itemize}
\item Base: 2
\item Representa: números enteros positivos
\item Máximo: $2^{n-1}_{10}$
\item Mínimo: 0
\end{itemize}

\paragraph{Almacenar}
\begin{enumerate}
\item Pasar el numero a base 2
\item completar con ceros a izquierda la capacidad del formato
\end{enumerate}
\paragraph{Recuperar}
\begin{enumerate}
\item Pasamos el numero de base 2 a la base deseada 
\end{enumerate}

%Binario de punto fijo con signo
\subsubsection{Binario de punto fijo con signo}
\begin{itemize}
\item Base: 2
\item Representa: Enteros positivos y negativos
\item Primer bit: reservado para el signo
\item Máximo: $2^{n-1}-1$
\item Mínimo: $-2^{n-1}$
\end{itemize}

\paragraph{Almacenar}
\begin{enumerate}
\item Pasar el numero a base 2
\item completar con ceros a izquierda la capacidad del formato
\item Si es un numero negativo hacerle el complemento a 2
\end{enumerate}
\paragraph{Recuperar}
\begin{enumerate}
\item Si el bit de signo es cero, se pasa de base 2 a base 10
\item Si el bit es 1, el numero es negativo, por lo que debemos complementarlo
\item Quitamos los ceros a la izquierda
\item Numero de base 2 a base 10
\item Colocamos el signo
\end{enumerate}

\paragraph{Expansión}
\begin{enumerate}
\item Se completa con el bit de signo a la izquierda
\end{enumerate}
\paragraph{Truncamiento}
\begin{enumerate}
\item Se extraen bits a la izquierda siempre y cuando no se esté alterando el bit de signo del número.
\end{enumerate}

%Empaquetado
\subsubsection{Empaquetado}
\begin{itemize}
\item Base: 16
\item Representa: Enteros positivos y negativos
\item Máximo: $10^{2n-1} -1$
\item Mínimo: $-10^{2n-1} +1$
\end{itemize}

\paragraph{Almacenar}
\begin{enumerate}
\item numero a base 10
\item Colocal cada digito decimal en un nibble dejando el ultimo nibble ya que en el mismo se almacena el signo.
\item Colocar en el ultimo nible el signo, CAFE = + , DF = -
\item Se rellena con 0 hasta alcanzar la cantidad de bytes usados
\end{enumerate}
\paragraph{Recuperar}
\begin{enumerate}
\item los pasos en orden inverso
\end{enumerate}

%Zoneado
\subsubsection{Zoneado}
\begin{itemize}
\item Base: 16
\item Representa: Enteros positivos y negativos
\item Máximo: $10^{n} -1$
\item Mínimo: $-10^{n} +1$
\end{itemize}

\paragraph{Almacenar}
\begin{enumerate}
\item numero a base 10
\item colocar cada uno de los digitos decimales en un nibble derecho
\item completar todos los nibbles de izquierda con F salvo el ultimo que se completa con el signo siguiendo las mismas reglas que para empaquetados.  CAFE = + , DF = -
\item se rellena con F0 hasta alcanzar la cantidad de bytes usados
\end{enumerate}
\paragraph{Recuperar}
\begin{enumerate}
\item los pasos en orden inverso
\end{enumerate}

\subsubsection{Binario de punto Flotante}
es la manera que tiene una arquitectura de representar a los numeros reales.
su notación cientifica se expresa de la siguiente manera $M x B^E$
M: Mantissa B: base E: Exponente

Un numero binario está normalizado si el digito de la izquierda del punto es igual a 1.

\paragraph{IEEE754}
\begin{itemize}
\item Precision Simple: signo: 1 bit exponente: 8 bits fraccion: 23 bits Exponente: Exceso 127
\item Precision Doble: signo: 1 bit exponente 11bits fraccion: 52 bits Exponente: Exceso 1023
\end{itemize}

\paragraph{Ancho de paso}\mbox{}\\\\
Marca cual es la distancia entre un flotante y su siguiente numero representable en el formato

\paragraph{Overflow}\mbox{}\\\\
el exponente excede el limite superior, tanto para mantisas positivas como para negatvias, dando lugar a +inf, - inf
\paragraph{Underflow}\mbox{}\\\\
el exponente excede el minimo valor permitido y caga en el intervalo (-inf, -0) y (+0,+inf)

\paragraph{Desnormalizados - Subnormales}\mbox{}\\\\
tienen como exponente al cero, y el bit implicito a la izquierda del punto binario, es ahora un cero implicito. la diferencia entre los desnormalizados y los normalizados es que, estos ultimos no permiten al cero como exponente. Los normalizados tienen 24 bits significativos, mientras que los normalizados poseen 23.

\begin{table}[h]
\begin{tabular}{lllll}
\cline{1-4}
\multicolumn{1}{|l|}{Normalizado} & \multicolumn{1}{l|}{+/-} & \multicolumn{1}{l|}{0<exp<max} &  \multicolumn{1}{l|}{cualquier patron de bits} &  \\ \cline{1-4}
\multicolumn{1}{|l|}{Desnormalizado} & \multicolumn{1}{l|}{+/-} & \multicolumn{1}{l|}{0} &  \multicolumn{1}{l|}{cualquier patron de bits != 0} &  \\ \cline{1-4}
\multicolumn{1}{|l|}{Cero} & \multicolumn{1}{l|}{+/-} & \multicolumn{1}{l|}{0} &  \multicolumn{1}{l|}{0} &  \\ \cline{1-4}
\multicolumn{1}{|l|}{Infinito} & \multicolumn{1}{l|}{+/-} & \multicolumn{1}{l|}{11...11} &  \multicolumn{1}{l|}{0} &  \\ \cline{1-4}
\multicolumn{1}{|l|}{NAN} & \multicolumn{1}{l|}{+/-} & \multicolumn{1}{l|}{11...11} &  \multicolumn{1}{l|}{Cualquier patron de bits != 0} &  \\ \cline{1-4}
\end{tabular}
\end{table}

\begin{itemize}
\item Infinito dividido Infinito = NAN
\item Cero + = 0 00000000 00000000000000000000000
\item Cero - = 1 00000000 00000000000000000000000
\item Infinito + = 0 11111111 00000000000000000000000
\item Infinito - = 1 11111111 00000000000000000000000
\item No normalizados/Subnormales (no se asume que haya que añadir un 1 al significando para obtener su valor).
\end{itemize}

\paragraph{Valores no numericos}\mbox{}\\\\
NaN (Not a number). 2 tipos, QNaN (quiet nan) y SNaN (signalling nan) Qnan = indeterminado Snan = operacion no valida

\begin{itemize}
\item Infinito dividido Infinito = NAN
\item qnan = 0 11111111 10000100000000000000000
\item snan = 1 11111111 00100010001001010101010
\end{itemize}

\begin{table}[h]
\begin{tabular}{lllll}
\cline{1-2}
\multicolumn{1}{|l|}{Operacion} & \multicolumn{1}{l|}{Resultado} &   \\ \cline{1-2}
\multicolumn{1}{|l|}{n $\pm$ infinito } & \multicolumn{1}{l|}{0} &   \\ \cline{1-2}
\multicolumn{1}{|l|}{$\pm$ infinito $\cdot$ $\pm$infinito } & \multicolumn{1}{l|}{$\pm$infinito} &   \\ \cline{1-2}
\multicolumn{1}{|l|}{ $\pm$n $\div$ 0   } & \multicolumn{1}{l|}{ $\pm$infinito } &   \\ \cline{1-2}
\multicolumn{1}{|l|}{ Infinito + Infinito   } & \multicolumn{1}{l|}{ Infinito } &   \\ \cline{1-2}
\multicolumn{1}{|l|}{Cualquier operación contra un NaN   } & \multicolumn{1}{l|}{ NaN } &   \\ \cline{1-2}
\multicolumn{1}{|l|}{ $\pm$0 $\div$ $\pm$0   } & \multicolumn{1}{l|}{ NaN } &   \\ \cline{1-2}
\multicolumn{1}{|l|}{ Infinito - Infinito   } & \multicolumn{1}{l|}{  NaN} &   \\ \cline{1-2}
\multicolumn{1}{|l|}{ $\pm$Infinito  $\div$ $\pm$Infinito  } & \multicolumn{1}{l|}{  NaN} &   \\ \cline{1-2}
\multicolumn{1}{|l|}{ $\pm$Infinito  $\cdot$ $\pm$0 } & \multicolumn{1}{l|}{  NaN} &   \\ \cline{1-2}
\end{tabular}
\end{table}

\paragraph{Almacenar -6,12510}\mbox{}\\
\begin{enumerate}    
  \item El bit 31 tomará el valor del signo del número. 
  \item Pasar a binario la mantisa decimal.
  \begin{itemize}
    \item $6=110_2$
    \item $0,125=0,001_2$
    \item $6,125=110,001_2$
  \end{itemize}
  \item Normalizar
  \begin{itemize}
    \item Desplazamiento a la derecha  -> Exponente negativo
    \item Desplazamiento a la izquierda -> Exponente positivo
    \item 1,10001 , exponente = 2
    \item 2 expresado en Exceso 127 es 129 $0000001_2$
  \end{itemize}
  \item Mantisa representada con bit implícito: 1,10001 -> 10001
   \item El número final es 1 10000001 100010000000000000000002 (Se agregan a la derecha los “0” necesarios para completar los 23 bits de la mantisa)
\end{enumerate}

\paragraph{Recuperar}\mbox{}\\\\
\begin{enumerate}    
  \item Se realizan los pasos en orden inverso
\end{enumerate}

\paragraph{Binario de Punto Flotante (IBM mainframe)}
\begin{itemize}
\item Base: 16
\item Representa: enteros con coma decimal positivos y negativos
\item Precision : imple 4 bytes, doble, 8 bytes, extendida 16 bytes.
\item estructura: S nnnnnnn dddddd
\begin{itemize}
    \item s = signo 1- 0+
    \item n = digitos de la caracteristica, en total son 7 bits se usan para calcular el exponente 
    \item C = E + 4016 donde E corresponde al exponente
    \item d =  mantisa normalizada -> 0,dddddd x $10^e_{16}$
  \end{itemize}
\end{itemize}

\paragraph{Almacenar - 321,54 10 -> Binario de punto flotante precisión simple}\mbox{}\\\\
\begin{enumerate}    
  \item $321,54_{10} = 141,8A3_{16}$
  \item $0,1418A3 x 10^3 _{16}$
  \item $C = E + 40_{16} = 3 + 40_{16} = 43_{16} en base 2 sería 100011_{2}$
  \item Agregamos el bit de signo: $1100011_{2}$ que en base 16 es $C3_{16}$
  \item Resultado final : $C31418A3_{16}$
\end{enumerate}

\paragraph{Ancho de paso }\mbox{}\\\\
distancia entre un flotante y su siguiente numero representable en el formato

\paragraph{Absorcion}\mbox{}\\\\
se da en las operaciones de suma y resta entre flotantes

\begin{itemize}    
  \item $A = 0,15A4 x 105_{16}$
  \item $B = 0,54F x 10^-2_{16}$
\end{itemize}

para poder operar entre flotantes debemos igualar los exponentes llevandolos al mayor de todos
\begin{itemize}    
  \item $A+B = 0.15A400 * 10^5 + 0,00...00 * 10^5 = 0,15A4000 * 10^5 _{16}$
\end{itemize}
lo minimo que se puede sumar es el ancho de paso del numero de mayor exponente

\section{Intel x86}
\section{Formato y Configuracion}

\subsection{Definición}
\begin{itemize}
\item Formato: Representación computacional
\item Configuración: Representación en una determinada base de un número en un formato
\end{itemize}

\subsection{Expansión y truncamiento}
\subsection{Definición}
\begin{itemize}
\item Expandir formato: Significa completar la representación computacional sin alterar el numero representado
en el mismo. 
\item Truncar formato: Descartar digitos de su representación sin alterar el número representado en el mismo.
\end{itemize}

\subsection{Formatos}

%Binario punto fijo sin signo
\subsubsection{Binario punto fijo sin signo}
\begin{itemize}
\item Base: 2
\item Representa: números enteros positivos
\item Máximo: $2^{n-1}_{10}$
\item Mínimo: 0
\end{itemize}

\paragraph{Almacenar}
\begin{enumerate}
\item Pasar el numero a base 2
\item completar con ceros a izquierda la capacidad del formato
\end{enumerate}
\paragraph{Recuperar}
\begin{enumerate}
\item Pasamos el numero de base 2 a la base deseada 
\end{enumerate}

%Binario de punto fijo con signo
\subsubsection{Binario de punto fijo con signo}
\begin{itemize}
\item Base: 2
\item Representa: Enteros positivos y negativos
\item Primer bit: reservado para el signo
\item Máximo: $2^{n-1}-1$
\item Mínimo: $-2^{n-1}$
\end{itemize}

\paragraph{Almacenar}
\begin{enumerate}
\item Pasar el numero a base 2
\item completar con ceros a izquierda la capacidad del formato
\item Si es un numero negativo hacerle el complemento a 2
\end{enumerate}
\paragraph{Recuperar}
\begin{enumerate}
\item Si el bit de signo es cero, se pasa de base 2 a base 10
\item Si el bit es 1, el numero es negativo, por lo que debemos complementarlo
\item Quitamos los ceros a la izquierda
\item Numero de base 2 a base 10
\item Colocamos el signo
\end{enumerate}

\paragraph{Expansión}
\begin{enumerate}
\item Se completa con el bit de signo a la izquierda
\end{enumerate}
\paragraph{Truncamiento}
\begin{enumerate}
\item Se extraen bits a la izquierda siempre y cuando no se esté alterando el bit de signo del número.
\end{enumerate}

%Empaquetado
\subsubsection{Empaquetado}
\begin{itemize}
\item Base: 16
\item Representa: Enteros positivos y negativos
\item Máximo: $10^{2n-1} -1$
\item Mínimo: $-10^{2n-1} +1$
\end{itemize}

\paragraph{Almacenar}
\begin{enumerate}
\item numero a base 10
\item Colocal cada digito decimal en un nibble dejando el ultimo nibble ya que en el mismo se almacena el signo.
\item Colocar en el ultimo nible el signo, CAFE = + , DF = -
\item Se rellena con 0 hasta alcanzar la cantidad de bytes usados
\end{enumerate}
\paragraph{Recuperar}
\begin{enumerate}
\item los pasos en orden inverso
\end{enumerate}

%Zoneado
\subsubsection{Zoneado}
\begin{itemize}
\item Base: 16
\item Representa: Enteros positivos y negativos
\item Máximo: $10^{n} -1$
\item Mínimo: $-10^{n} +1$
\end{itemize}

\paragraph{Almacenar}
\begin{enumerate}
\item numero a base 10
\item colocar cada uno de los digitos decimales en un nibble derecho
\item completar todos los nibbles de izquierda con F salvo el ultimo que se completa con el signo siguiendo las mismas reglas que para empaquetados.  CAFE = + , DF = -
\item se rellena con F0 hasta alcanzar la cantidad de bytes usados
\end{enumerate}
\paragraph{Recuperar}
\begin{enumerate}
\item los pasos en orden inverso
\end{enumerate}

\subsubsection{Binario de punto Flotante}
es la manera que tiene una arquitectura de representar a los numeros reales.
su notación cientifica se expresa de la siguiente manera $M x B^E$
M: Mantissa B: base E: Exponente

Un numero binario está normalizado si el digito de la izquierda del punto es igual a 1.

\paragraph{IEEE754}
\begin{itemize}
\item Precision Simple: signo: 1 bit exponente: 8 bits fraccion: 23 bits Exponente: Exceso 127
\item Precision Doble: signo: 1 bit exponente 11bits fraccion: 52 bits Exponente: Exceso 1023
\end{itemize}

\paragraph{Ancho de paso}\mbox{}\\\\
Marca cual es la distancia entre un flotante y su siguiente numero representable en el formato

\paragraph{Overflow}\mbox{}\\\\
el exponente excede el limite superior, tanto para mantisas positivas como para negatvias, dando lugar a +inf, - inf
\paragraph{Underflow}\mbox{}\\\\
el exponente excede el minimo valor permitido y caga en el intervalo (-inf, -0) y (+0,+inf)

\paragraph{Desnormalizados - Subnormales}\mbox{}\\\\
tienen como exponente al cero, y el bit implicito a la izquierda del punto binario, es ahora un cero implicito. la diferencia entre los desnormalizados y los normalizados es que, estos ultimos no permiten al cero como exponente. Los normalizados tienen 24 bits significativos, mientras que los normalizados poseen 23.

\begin{table}[h]
\begin{tabular}{lllll}
\cline{1-4}
\multicolumn{1}{|l|}{Normalizado} & \multicolumn{1}{l|}{+/-} & \multicolumn{1}{l|}{0<exp<max} &  \multicolumn{1}{l|}{cualquier patron de bits} &  \\ \cline{1-4}
\multicolumn{1}{|l|}{Desnormalizado} & \multicolumn{1}{l|}{+/-} & \multicolumn{1}{l|}{0} &  \multicolumn{1}{l|}{cualquier patron de bits != 0} &  \\ \cline{1-4}
\multicolumn{1}{|l|}{Cero} & \multicolumn{1}{l|}{+/-} & \multicolumn{1}{l|}{0} &  \multicolumn{1}{l|}{0} &  \\ \cline{1-4}
\multicolumn{1}{|l|}{Infinito} & \multicolumn{1}{l|}{+/-} & \multicolumn{1}{l|}{11...11} &  \multicolumn{1}{l|}{0} &  \\ \cline{1-4}
\multicolumn{1}{|l|}{NAN} & \multicolumn{1}{l|}{+/-} & \multicolumn{1}{l|}{11...11} &  \multicolumn{1}{l|}{Cualquier patron de bits != 0} &  \\ \cline{1-4}
\end{tabular}
\end{table}

\begin{itemize}
\item Infinito dividido Infinito = NAN
\item Cero + = 0 00000000 00000000000000000000000
\item Cero - = 1 00000000 00000000000000000000000
\item Infinito + = 0 11111111 00000000000000000000000
\item Infinito - = 1 11111111 00000000000000000000000
\item No normalizados/Subnormales (no se asume que haya que añadir un 1 al significando para obtener su valor).
\end{itemize}

\paragraph{Valores no numericos}\mbox{}\\\\
NaN (Not a number). 2 tipos, QNaN (quiet nan) y SNaN (signalling nan) Qnan = indeterminado Snan = operacion no valida

\begin{itemize}
\item Infinito dividido Infinito = NAN
\item qnan = 0 11111111 10000100000000000000000
\item snan = 1 11111111 00100010001001010101010
\end{itemize}

\begin{table}[h]
\begin{tabular}{lllll}
\cline{1-2}
\multicolumn{1}{|l|}{Operacion} & \multicolumn{1}{l|}{Resultado} &   \\ \cline{1-2}
\multicolumn{1}{|l|}{n $\pm$ infinito } & \multicolumn{1}{l|}{0} &   \\ \cline{1-2}
\multicolumn{1}{|l|}{$\pm$ infinito $\cdot$ $\pm$infinito } & \multicolumn{1}{l|}{$\pm$infinito} &   \\ \cline{1-2}
\multicolumn{1}{|l|}{ $\pm$n $\div$ 0   } & \multicolumn{1}{l|}{ $\pm$infinito } &   \\ \cline{1-2}
\multicolumn{1}{|l|}{ Infinito + Infinito   } & \multicolumn{1}{l|}{ Infinito } &   \\ \cline{1-2}
\multicolumn{1}{|l|}{Cualquier operación contra un NaN   } & \multicolumn{1}{l|}{ NaN } &   \\ \cline{1-2}
\multicolumn{1}{|l|}{ $\pm$0 $\div$ $\pm$0   } & \multicolumn{1}{l|}{ NaN } &   \\ \cline{1-2}
\multicolumn{1}{|l|}{ Infinito - Infinito   } & \multicolumn{1}{l|}{  NaN} &   \\ \cline{1-2}
\multicolumn{1}{|l|}{ $\pm$Infinito  $\div$ $\pm$Infinito  } & \multicolumn{1}{l|}{  NaN} &   \\ \cline{1-2}
\multicolumn{1}{|l|}{ $\pm$Infinito  $\cdot$ $\pm$0 } & \multicolumn{1}{l|}{  NaN} &   \\ \cline{1-2}
\end{tabular}
\end{table}

\paragraph{Almacenar -6,12510}\mbox{}\\
\begin{enumerate}    
  \item El bit 31 tomará el valor del signo del número. 
  \item Pasar a binario la mantisa decimal.
  \begin{itemize}
    \item $6=110_2$
    \item $0,125=0,001_2$
    \item $6,125=110,001_2$
  \end{itemize}
  \item Normalizar
  \begin{itemize}
    \item Desplazamiento a la derecha  -> Exponente negativo
    \item Desplazamiento a la izquierda -> Exponente positivo
    \item 1,10001 , exponente = 2
    \item 2 expresado en Exceso 127 es 129 $0000001_2$
  \end{itemize}
  \item Mantisa representada con bit implícito: 1,10001 -> 10001
   \item El número final es 1 10000001 100010000000000000000002 (Se agregan a la derecha los “0” necesarios para completar los 23 bits de la mantisa)
\end{enumerate}

\paragraph{Recuperar}\mbox{}\\\\
\begin{enumerate}    
  \item Se realizan los pasos en orden inverso
\end{enumerate}

\paragraph{Binario de Punto Flotante (IBM mainframe)}
\begin{itemize}
\item Base: 16
\item Representa: enteros con coma decimal positivos y negativos
\item Precision : imple 4 bytes, doble, 8 bytes, extendida 16 bytes.
\item estructura: S nnnnnnn dddddd
\begin{itemize}
    \item s = signo 1- 0+
    \item n = digitos de la caracteristica, en total son 7 bits se usan para calcular el exponente 
    \item C = E + 4016 donde E corresponde al exponente
    \item d =  mantisa normalizada -> 0,dddddd x $10^e_{16}$
  \end{itemize}
\end{itemize}

\paragraph{Almacenar - 321,54 10 -> Binario de punto flotante precisión simple}\mbox{}\\\\
\begin{enumerate}    
  \item $321,54_{10} = 141,8A3_{16}$
  \item $0,1418A3 x 10^3 _{16}$
  \item $C = E + 40_{16} = 3 + 40_{16} = 43_{16} en base 2 sería 100011_{2}$
  \item Agregamos el bit de signo: $1100011_{2}$ que en base 16 es $C3_{16}$
  \item Resultado final : $C31418A3_{16}$
\end{enumerate}

\paragraph{Ancho de paso }\mbox{}\\\\
distancia entre un flotante y su siguiente numero representable en el formato

\paragraph{Absorcion}\mbox{}\\\\
se da en las operaciones de suma y resta entre flotantes

\begin{itemize}    
  \item $A = 0,15A4 x 105_{16}$
  \item $B = 0,54F x 10^-2_{16}$
\end{itemize}

para poder operar entre flotantes debemos igualar los exponentes llevandolos al mayor de todos
\begin{itemize}    
  \item $A+B = 0.15A400 * 10^5 + 0,00...00 * 10^5 = 0,15A4000 * 10^5 _{16}$
\end{itemize}
lo minimo que se puede sumar es el ancho de paso del numero de mayor exponente

\section{Intel x86}
\section{Formato y Configuracion}

\subsection{Definición}
\begin{itemize}
\item Formato: Representación computacional
\item Configuración: Representación en una determinada base de un número en un formato
\end{itemize}

\subsection{Expansión y truncamiento}
\subsection{Definición}
\begin{itemize}
\item Expandir formato: Significa completar la representación computacional sin alterar el numero representado
en el mismo. 
\item Truncar formato: Descartar digitos de su representación sin alterar el número representado en el mismo.
\end{itemize}

\subsection{Formatos}

%Binario punto fijo sin signo
\subsubsection{Binario punto fijo sin signo}
\begin{itemize}
\item Base: 2
\item Representa: números enteros positivos
\item Máximo: $2^{n-1}_{10}$
\item Mínimo: 0
\end{itemize}

\paragraph{Almacenar}
\begin{enumerate}
\item Pasar el numero a base 2
\item completar con ceros a izquierda la capacidad del formato
\end{enumerate}
\paragraph{Recuperar}
\begin{enumerate}
\item Pasamos el numero de base 2 a la base deseada 
\end{enumerate}

%Binario de punto fijo con signo
\subsubsection{Binario de punto fijo con signo}
\begin{itemize}
\item Base: 2
\item Representa: Enteros positivos y negativos
\item Primer bit: reservado para el signo
\item Máximo: $2^{n-1}-1$
\item Mínimo: $-2^{n-1}$
\end{itemize}

\paragraph{Almacenar}
\begin{enumerate}
\item Pasar el numero a base 2
\item completar con ceros a izquierda la capacidad del formato
\item Si es un numero negativo hacerle el complemento a 2
\end{enumerate}
\paragraph{Recuperar}
\begin{enumerate}
\item Si el bit de signo es cero, se pasa de base 2 a base 10
\item Si el bit es 1, el numero es negativo, por lo que debemos complementarlo
\item Quitamos los ceros a la izquierda
\item Numero de base 2 a base 10
\item Colocamos el signo
\end{enumerate}

\paragraph{Expansión}
\begin{enumerate}
\item Se completa con el bit de signo a la izquierda
\end{enumerate}
\paragraph{Truncamiento}
\begin{enumerate}
\item Se extraen bits a la izquierda siempre y cuando no se esté alterando el bit de signo del número.
\end{enumerate}

%Empaquetado
\subsubsection{Empaquetado}
\begin{itemize}
\item Base: 16
\item Representa: Enteros positivos y negativos
\item Máximo: $10^{2n-1} -1$
\item Mínimo: $-10^{2n-1} +1$
\end{itemize}

\paragraph{Almacenar}
\begin{enumerate}
\item numero a base 10
\item Colocal cada digito decimal en un nibble dejando el ultimo nibble ya que en el mismo se almacena el signo.
\item Colocar en el ultimo nible el signo, CAFE = + , DF = -
\item Se rellena con 0 hasta alcanzar la cantidad de bytes usados
\end{enumerate}
\paragraph{Recuperar}
\begin{enumerate}
\item los pasos en orden inverso
\end{enumerate}

%Zoneado
\subsubsection{Zoneado}
\begin{itemize}
\item Base: 16
\item Representa: Enteros positivos y negativos
\item Máximo: $10^{n} -1$
\item Mínimo: $-10^{n} +1$
\end{itemize}

\paragraph{Almacenar}
\begin{enumerate}
\item numero a base 10
\item colocar cada uno de los digitos decimales en un nibble derecho
\item completar todos los nibbles de izquierda con F salvo el ultimo que se completa con el signo siguiendo las mismas reglas que para empaquetados.  CAFE = + , DF = -
\item se rellena con F0 hasta alcanzar la cantidad de bytes usados
\end{enumerate}
\paragraph{Recuperar}
\begin{enumerate}
\item los pasos en orden inverso
\end{enumerate}

\subsubsection{Binario de punto Flotante}
es la manera que tiene una arquitectura de representar a los numeros reales.
su notación cientifica se expresa de la siguiente manera $M x B^E$
M: Mantissa B: base E: Exponente

Un numero binario está normalizado si el digito de la izquierda del punto es igual a 1.

\paragraph{IEEE754}
\begin{itemize}
\item Precision Simple: signo: 1 bit exponente: 8 bits fraccion: 23 bits Exponente: Exceso 127
\item Precision Doble: signo: 1 bit exponente 11bits fraccion: 52 bits Exponente: Exceso 1023
\end{itemize}

\paragraph{Ancho de paso}\mbox{}\\\\
Marca cual es la distancia entre un flotante y su siguiente numero representable en el formato

\paragraph{Overflow}\mbox{}\\\\
el exponente excede el limite superior, tanto para mantisas positivas como para negatvias, dando lugar a +inf, - inf
\paragraph{Underflow}\mbox{}\\\\
el exponente excede el minimo valor permitido y caga en el intervalo (-inf, -0) y (+0,+inf)

\paragraph{Desnormalizados - Subnormales}\mbox{}\\\\
tienen como exponente al cero, y el bit implicito a la izquierda del punto binario, es ahora un cero implicito. la diferencia entre los desnormalizados y los normalizados es que, estos ultimos no permiten al cero como exponente. Los normalizados tienen 24 bits significativos, mientras que los normalizados poseen 23.

\begin{table}[h]
\begin{tabular}{lllll}
\cline{1-4}
\multicolumn{1}{|l|}{Normalizado} & \multicolumn{1}{l|}{+/-} & \multicolumn{1}{l|}{0<exp<max} &  \multicolumn{1}{l|}{cualquier patron de bits} &  \\ \cline{1-4}
\multicolumn{1}{|l|}{Desnormalizado} & \multicolumn{1}{l|}{+/-} & \multicolumn{1}{l|}{0} &  \multicolumn{1}{l|}{cualquier patron de bits != 0} &  \\ \cline{1-4}
\multicolumn{1}{|l|}{Cero} & \multicolumn{1}{l|}{+/-} & \multicolumn{1}{l|}{0} &  \multicolumn{1}{l|}{0} &  \\ \cline{1-4}
\multicolumn{1}{|l|}{Infinito} & \multicolumn{1}{l|}{+/-} & \multicolumn{1}{l|}{11...11} &  \multicolumn{1}{l|}{0} &  \\ \cline{1-4}
\multicolumn{1}{|l|}{NAN} & \multicolumn{1}{l|}{+/-} & \multicolumn{1}{l|}{11...11} &  \multicolumn{1}{l|}{Cualquier patron de bits != 0} &  \\ \cline{1-4}
\end{tabular}
\end{table}

\begin{itemize}
\item Infinito dividido Infinito = NAN
\item Cero + = 0 00000000 00000000000000000000000
\item Cero - = 1 00000000 00000000000000000000000
\item Infinito + = 0 11111111 00000000000000000000000
\item Infinito - = 1 11111111 00000000000000000000000
\item No normalizados/Subnormales (no se asume que haya que añadir un 1 al significando para obtener su valor).
\end{itemize}

\paragraph{Valores no numericos}\mbox{}\\\\
NaN (Not a number). 2 tipos, QNaN (quiet nan) y SNaN (signalling nan) Qnan = indeterminado Snan = operacion no valida

\begin{itemize}
\item Infinito dividido Infinito = NAN
\item qnan = 0 11111111 10000100000000000000000
\item snan = 1 11111111 00100010001001010101010
\end{itemize}

\begin{table}[h]
\begin{tabular}{lllll}
\cline{1-2}
\multicolumn{1}{|l|}{Operacion} & \multicolumn{1}{l|}{Resultado} &   \\ \cline{1-2}
\multicolumn{1}{|l|}{n $\pm$ infinito } & \multicolumn{1}{l|}{0} &   \\ \cline{1-2}
\multicolumn{1}{|l|}{$\pm$ infinito $\cdot$ $\pm$infinito } & \multicolumn{1}{l|}{$\pm$infinito} &   \\ \cline{1-2}
\multicolumn{1}{|l|}{ $\pm$n $\div$ 0   } & \multicolumn{1}{l|}{ $\pm$infinito } &   \\ \cline{1-2}
\multicolumn{1}{|l|}{ Infinito + Infinito   } & \multicolumn{1}{l|}{ Infinito } &   \\ \cline{1-2}
\multicolumn{1}{|l|}{Cualquier operación contra un NaN   } & \multicolumn{1}{l|}{ NaN } &   \\ \cline{1-2}
\multicolumn{1}{|l|}{ $\pm$0 $\div$ $\pm$0   } & \multicolumn{1}{l|}{ NaN } &   \\ \cline{1-2}
\multicolumn{1}{|l|}{ Infinito - Infinito   } & \multicolumn{1}{l|}{  NaN} &   \\ \cline{1-2}
\multicolumn{1}{|l|}{ $\pm$Infinito  $\div$ $\pm$Infinito  } & \multicolumn{1}{l|}{  NaN} &   \\ \cline{1-2}
\multicolumn{1}{|l|}{ $\pm$Infinito  $\cdot$ $\pm$0 } & \multicolumn{1}{l|}{  NaN} &   \\ \cline{1-2}
\end{tabular}
\end{table}

\paragraph{Almacenar -6,12510}\mbox{}\\
\begin{enumerate}    
  \item El bit 31 tomará el valor del signo del número. 
  \item Pasar a binario la mantisa decimal.
  \begin{itemize}
    \item $6=110_2$
    \item $0,125=0,001_2$
    \item $6,125=110,001_2$
  \end{itemize}
  \item Normalizar
  \begin{itemize}
    \item Desplazamiento a la derecha  -> Exponente negativo
    \item Desplazamiento a la izquierda -> Exponente positivo
    \item 1,10001 , exponente = 2
    \item 2 expresado en Exceso 127 es 129 $0000001_2$
  \end{itemize}
  \item Mantisa representada con bit implícito: 1,10001 -> 10001
   \item El número final es 1 10000001 100010000000000000000002 (Se agregan a la derecha los “0” necesarios para completar los 23 bits de la mantisa)
\end{enumerate}

\paragraph{Recuperar}\mbox{}\\\\
\begin{enumerate}    
  \item Se realizan los pasos en orden inverso
\end{enumerate}

\paragraph{Binario de Punto Flotante (IBM mainframe)}
\begin{itemize}
\item Base: 16
\item Representa: enteros con coma decimal positivos y negativos
\item Precision : imple 4 bytes, doble, 8 bytes, extendida 16 bytes.
\item estructura: S nnnnnnn dddddd
\begin{itemize}
    \item s = signo 1- 0+
    \item n = digitos de la caracteristica, en total son 7 bits se usan para calcular el exponente 
    \item C = E + 4016 donde E corresponde al exponente
    \item d =  mantisa normalizada -> 0,dddddd x $10^e_{16}$
  \end{itemize}
\end{itemize}

\paragraph{Almacenar - 321,54 10 -> Binario de punto flotante precisión simple}\mbox{}\\\\
\begin{enumerate}    
  \item $321,54_{10} = 141,8A3_{16}$
  \item $0,1418A3 x 10^3 _{16}$
  \item $C = E + 40_{16} = 3 + 40_{16} = 43_{16} en base 2 sería 100011_{2}$
  \item Agregamos el bit de signo: $1100011_{2}$ que en base 16 es $C3_{16}$
  \item Resultado final : $C31418A3_{16}$
\end{enumerate}

\paragraph{Ancho de paso }\mbox{}\\\\
distancia entre un flotante y su siguiente numero representable en el formato

\paragraph{Absorcion}\mbox{}\\\\
se da en las operaciones de suma y resta entre flotantes

\begin{itemize}    
  \item $A = 0,15A4 x 105_{16}$
  \item $B = 0,54F x 10^-2_{16}$
\end{itemize}

para poder operar entre flotantes debemos igualar los exponentes llevandolos al mayor de todos
\begin{itemize}    
  \item $A+B = 0.15A400 * 10^5 + 0,00...00 * 10^5 = 0,15A4000 * 10^5 _{16}$
\end{itemize}
lo minimo que se puede sumar es el ancho de paso del numero de mayor exponente

\section{Intel x86}
\section{Formato y Configuracion}

\subsection{Definición}
\begin{itemize}
\item Formato: Representación computacional
\item Configuración: Representación en una determinada base de un número en un formato
\end{itemize}

\subsection{Expansión y truncamiento}
\subsection{Definición}
\begin{itemize}
\item Expandir formato: Significa completar la representación computacional sin alterar el numero representado
en el mismo. 
\item Truncar formato: Descartar digitos de su representación sin alterar el número representado en el mismo.
\end{itemize}

\subsection{Formatos}

%Binario punto fijo sin signo
\subsubsection{Binario punto fijo sin signo}
\begin{itemize}
\item Base: 2
\item Representa: números enteros positivos
\item Máximo: $2^{n-1}_{10}$
\item Mínimo: 0
\end{itemize}

\paragraph{Almacenar}
\begin{enumerate}
\item Pasar el numero a base 2
\item completar con ceros a izquierda la capacidad del formato
\end{enumerate}
\paragraph{Recuperar}
\begin{enumerate}
\item Pasamos el numero de base 2 a la base deseada 
\end{enumerate}

%Binario de punto fijo con signo
\subsubsection{Binario de punto fijo con signo}
\begin{itemize}
\item Base: 2
\item Representa: Enteros positivos y negativos
\item Primer bit: reservado para el signo
\item Máximo: $2^{n-1}-1$
\item Mínimo: $-2^{n-1}$
\end{itemize}

\paragraph{Almacenar}
\begin{enumerate}
\item Pasar el numero a base 2
\item completar con ceros a izquierda la capacidad del formato
\item Si es un numero negativo hacerle el complemento a 2
\end{enumerate}
\paragraph{Recuperar}
\begin{enumerate}
\item Si el bit de signo es cero, se pasa de base 2 a base 10
\item Si el bit es 1, el numero es negativo, por lo que debemos complementarlo
\item Quitamos los ceros a la izquierda
\item Numero de base 2 a base 10
\item Colocamos el signo
\end{enumerate}

\paragraph{Expansión}
\begin{enumerate}
\item Se completa con el bit de signo a la izquierda
\end{enumerate}
\paragraph{Truncamiento}
\begin{enumerate}
\item Se extraen bits a la izquierda siempre y cuando no se esté alterando el bit de signo del número.
\end{enumerate}

%Empaquetado
\subsubsection{Empaquetado}
\begin{itemize}
\item Base: 16
\item Representa: Enteros positivos y negativos
\item Máximo: $10^{2n-1} -1$
\item Mínimo: $-10^{2n-1} +1$
\end{itemize}

\paragraph{Almacenar}
\begin{enumerate}
\item numero a base 10
\item Colocal cada digito decimal en un nibble dejando el ultimo nibble ya que en el mismo se almacena el signo.
\item Colocar en el ultimo nible el signo, CAFE = + , DF = -
\item Se rellena con 0 hasta alcanzar la cantidad de bytes usados
\end{enumerate}
\paragraph{Recuperar}
\begin{enumerate}
\item los pasos en orden inverso
\end{enumerate}

%Zoneado
\subsubsection{Zoneado}
\begin{itemize}
\item Base: 16
\item Representa: Enteros positivos y negativos
\item Máximo: $10^{n} -1$
\item Mínimo: $-10^{n} +1$
\end{itemize}

\paragraph{Almacenar}
\begin{enumerate}
\item numero a base 10
\item colocar cada uno de los digitos decimales en un nibble derecho
\item completar todos los nibbles de izquierda con F salvo el ultimo que se completa con el signo siguiendo las mismas reglas que para empaquetados.  CAFE = + , DF = -
\item se rellena con F0 hasta alcanzar la cantidad de bytes usados
\end{enumerate}
\paragraph{Recuperar}
\begin{enumerate}
\item los pasos en orden inverso
\end{enumerate}

\subsubsection{Binario de punto Flotante}
es la manera que tiene una arquitectura de representar a los numeros reales.
su notación cientifica se expresa de la siguiente manera $M x B^E$
M: Mantissa B: base E: Exponente

Un numero binario está normalizado si el digito de la izquierda del punto es igual a 1.

\paragraph{IEEE754}
\begin{itemize}
\item Precision Simple: signo: 1 bit exponente: 8 bits fraccion: 23 bits Exponente: Exceso 127
\item Precision Doble: signo: 1 bit exponente 11bits fraccion: 52 bits Exponente: Exceso 1023
\end{itemize}

\paragraph{Ancho de paso}\mbox{}\\\\
Marca cual es la distancia entre un flotante y su siguiente numero representable en el formato

\paragraph{Overflow}\mbox{}\\\\
el exponente excede el limite superior, tanto para mantisas positivas como para negatvias, dando lugar a +inf, - inf
\paragraph{Underflow}\mbox{}\\\\
el exponente excede el minimo valor permitido y caga en el intervalo (-inf, -0) y (+0,+inf)

\paragraph{Desnormalizados - Subnormales}\mbox{}\\\\
tienen como exponente al cero, y el bit implicito a la izquierda del punto binario, es ahora un cero implicito. la diferencia entre los desnormalizados y los normalizados es que, estos ultimos no permiten al cero como exponente. Los normalizados tienen 24 bits significativos, mientras que los normalizados poseen 23.

\begin{table}[h]
\begin{tabular}{lllll}
\cline{1-4}
\multicolumn{1}{|l|}{Normalizado} & \multicolumn{1}{l|}{+/-} & \multicolumn{1}{l|}{0<exp<max} &  \multicolumn{1}{l|}{cualquier patron de bits} &  \\ \cline{1-4}
\multicolumn{1}{|l|}{Desnormalizado} & \multicolumn{1}{l|}{+/-} & \multicolumn{1}{l|}{0} &  \multicolumn{1}{l|}{cualquier patron de bits != 0} &  \\ \cline{1-4}
\multicolumn{1}{|l|}{Cero} & \multicolumn{1}{l|}{+/-} & \multicolumn{1}{l|}{0} &  \multicolumn{1}{l|}{0} &  \\ \cline{1-4}
\multicolumn{1}{|l|}{Infinito} & \multicolumn{1}{l|}{+/-} & \multicolumn{1}{l|}{11...11} &  \multicolumn{1}{l|}{0} &  \\ \cline{1-4}
\multicolumn{1}{|l|}{NAN} & \multicolumn{1}{l|}{+/-} & \multicolumn{1}{l|}{11...11} &  \multicolumn{1}{l|}{Cualquier patron de bits != 0} &  \\ \cline{1-4}
\end{tabular}
\end{table}

\begin{itemize}
\item Infinito dividido Infinito = NAN
\item Cero + = 0 00000000 00000000000000000000000
\item Cero - = 1 00000000 00000000000000000000000
\item Infinito + = 0 11111111 00000000000000000000000
\item Infinito - = 1 11111111 00000000000000000000000
\item No normalizados/Subnormales (no se asume que haya que añadir un 1 al significando para obtener su valor).
\end{itemize}

\paragraph{Valores no numericos}\mbox{}\\\\
NaN (Not a number). 2 tipos, QNaN (quiet nan) y SNaN (signalling nan) Qnan = indeterminado Snan = operacion no valida

\begin{itemize}
\item Infinito dividido Infinito = NAN
\item qnan = 0 11111111 10000100000000000000000
\item snan = 1 11111111 00100010001001010101010
\end{itemize}

\begin{table}[h]
\begin{tabular}{lllll}
\cline{1-2}
\multicolumn{1}{|l|}{Operacion} & \multicolumn{1}{l|}{Resultado} &   \\ \cline{1-2}
\multicolumn{1}{|l|}{n $\pm$ infinito } & \multicolumn{1}{l|}{0} &   \\ \cline{1-2}
\multicolumn{1}{|l|}{$\pm$ infinito $\cdot$ $\pm$infinito } & \multicolumn{1}{l|}{$\pm$infinito} &   \\ \cline{1-2}
\multicolumn{1}{|l|}{ $\pm$n $\div$ 0   } & \multicolumn{1}{l|}{ $\pm$infinito } &   \\ \cline{1-2}
\multicolumn{1}{|l|}{ Infinito + Infinito   } & \multicolumn{1}{l|}{ Infinito } &   \\ \cline{1-2}
\multicolumn{1}{|l|}{Cualquier operación contra un NaN   } & \multicolumn{1}{l|}{ NaN } &   \\ \cline{1-2}
\multicolumn{1}{|l|}{ $\pm$0 $\div$ $\pm$0   } & \multicolumn{1}{l|}{ NaN } &   \\ \cline{1-2}
\multicolumn{1}{|l|}{ Infinito - Infinito   } & \multicolumn{1}{l|}{  NaN} &   \\ \cline{1-2}
\multicolumn{1}{|l|}{ $\pm$Infinito  $\div$ $\pm$Infinito  } & \multicolumn{1}{l|}{  NaN} &   \\ \cline{1-2}
\multicolumn{1}{|l|}{ $\pm$Infinito  $\cdot$ $\pm$0 } & \multicolumn{1}{l|}{  NaN} &   \\ \cline{1-2}
\end{tabular}
\end{table}

\paragraph{Almacenar -6,12510}\mbox{}\\
\begin{enumerate}    
  \item El bit 31 tomará el valor del signo del número. 
  \item Pasar a binario la mantisa decimal.
  \begin{itemize}
    \item $6=110_2$
    \item $0,125=0,001_2$
    \item $6,125=110,001_2$
  \end{itemize}
  \item Normalizar
  \begin{itemize}
    \item Desplazamiento a la derecha  -> Exponente negativo
    \item Desplazamiento a la izquierda -> Exponente positivo
    \item 1,10001 , exponente = 2
    \item 2 expresado en Exceso 127 es 129 $0000001_2$
  \end{itemize}
  \item Mantisa representada con bit implícito: 1,10001 -> 10001
   \item El número final es 1 10000001 100010000000000000000002 (Se agregan a la derecha los “0” necesarios para completar los 23 bits de la mantisa)
\end{enumerate}

\paragraph{Recuperar}\mbox{}\\\\
\begin{enumerate}    
  \item Se realizan los pasos en orden inverso
\end{enumerate}

\paragraph{Binario de Punto Flotante (IBM mainframe)}
\begin{itemize}
\item Base: 16
\item Representa: enteros con coma decimal positivos y negativos
\item Precision : imple 4 bytes, doble, 8 bytes, extendida 16 bytes.
\item estructura: S nnnnnnn dddddd
\begin{itemize}
    \item s = signo 1- 0+
    \item n = digitos de la caracteristica, en total son 7 bits se usan para calcular el exponente 
    \item C = E + 4016 donde E corresponde al exponente
    \item d =  mantisa normalizada -> 0,dddddd x $10^e_{16}$
  \end{itemize}
\end{itemize}

\paragraph{Almacenar - 321,54 10 -> Binario de punto flotante precisión simple}\mbox{}\\\\
\begin{enumerate}    
  \item $321,54_{10} = 141,8A3_{16}$
  \item $0,1418A3 x 10^3 _{16}$
  \item $C = E + 40_{16} = 3 + 40_{16} = 43_{16} en base 2 sería 100011_{2}$
  \item Agregamos el bit de signo: $1100011_{2}$ que en base 16 es $C3_{16}$
  \item Resultado final : $C31418A3_{16}$
\end{enumerate}

\paragraph{Ancho de paso }\mbox{}\\\\
distancia entre un flotante y su siguiente numero representable en el formato

\paragraph{Absorcion}\mbox{}\\\\
se da en las operaciones de suma y resta entre flotantes

\begin{itemize}    
  \item $A = 0,15A4 x 105_{16}$
  \item $B = 0,54F x 10^-2_{16}$
\end{itemize}

para poder operar entre flotantes debemos igualar los exponentes llevandolos al mayor de todos
\begin{itemize}    
  \item $A+B = 0.15A400 * 10^5 + 0,00...00 * 10^5 = 0,15A4000 * 10^5 _{16}$
\end{itemize}
lo minimo que se puede sumar es el ancho de paso del numero de mayor exponente

\subsection{Interrupciones}
\subsubsection{Definicion}
Mecanismos por los cuales otros modulos (E/S y memoria) interrumpen el normal procesamiento del CPU
\subsubsection{¿Para que existen?}
Para mejorar la eficiencia de procesamiento de un computador
\subsubsection{Clases de interrupciones}
\begin{itemize}
	\item programa
	\item  reloj
	\item e/s
	\item fallas de hardware
\end{itemize}

\subsubsection{Ciclo de instruccion}
\begin{itemize}
	\item Fetch instruction
	\item  Decode instruction
	\item Fetch operand
	\item Execute instruction
	\item Store result
	\item ----------------> interrupt breakpoint
	\item process interrupt
\end{itemize}

\subsubsection{Transferencia de control al S.O. (Handler)}
/*4pdf*/
	
\subsubsection{Procesamiento de interrupciones }
/*5pdf*/

/*6pdf ejemplo*/
	
\subsubsection{Multiples interrupciones}


\paragraph{Deshabilitar interrupciones (secuencia)}\mbox{}\\\\%%
/*7pdf*/
\paragraph{Priorizar interrupciones (anidadas)}\mbox{}\\\\%%
/*8pdf*/
	
\paragraph{Múltiples interrupciones - ejemplo}\mbox{}\\\\%%
Tres dispositivos de E/S
\begin{itemize}
	\item Línea de comunicación (Prioridad 1)
	\item  Disco (Prioridad 2)
	\item Impresora (Prioridad 3)
\end{itemize}
Eventos
\begin{itemize}
	\item T=10 Interrupción de Impresora
	\item T=15 Interrupción de línea de comunicación
	\item T=20 Interrupción de disco
\end{itemize}

/*10pdf*/

\subsection{MODULO DE E/S }

\subsubsection{Que hace}
Conecta a los periféricos con la CPU y la memoria a través del bus del sistema o switch central y permite la comunicación entre ellos
\subsubsection{Para que sirve}
Oculta detalles de timing, formatos y electro mecánica de los dispositivos periféricos
\subsubsection{Por que existe?}
\begin{itemize}
\item Amplia variedad de periféricos con distintos métodos de operación
\item La tasa de transferencia de los periféricos es generalmente mucho más lenta que la de la memoria y procesador
\item Los periféricos usan distintos formatos de datos y tamaños de palabra
\end{itemize}

/*4pdf*/ 230 stalling

\subsubsection{Interface interna - bus del sistema}
\begin{itemize}
\item Datos
\item Direcciones
\item Control
\end{itemize}

\subsubsection{Interface externa - perifericos}
\begin{itemize}
\item Datos
\item Estado
\item Control
\end{itemize}


\subsubsection{Funciones}
\paragraph{Control and Timing}\mbox{}\\\\%%
ontrola flujo de tráfico entre CPU/Memoria y periféricos
\paragraph{Comunicación con el procesador}\mbox{}\\\\%%
Decodificación de comandos: The I/O module accepts commands from the processor, typically sent as signals on the control bus. For example, an I/O module for a disk drive might accept the following commands: READ SECTOR, WRITE SECTOR, SEEK track number, and SCAN record ID. The latter two commands each include a parameter that is sent on the data bus
\begin{itemize}
	\item Datos: Data are exchanged between the processor and the I/O module over the data bus.
	\item Información de estado: Because peripherals are so slow, it is important to know the	status of the I/O module. For example, if an I/O module is asked to send data to the processor (read), it may not be ready to do so because it is still working on the previous I/O command. This fact can be reported with a status signal.	Common status signals are BUSY and READY. There may also be signals to report various error conditions.
	\item Reconocimiento de direcciones: Just as each word of memory has an address, so does each I/O device. Thus, an I/O module must recognize one unique address for each peripheral it controls.
\end{itemize}
\paragraph{Comunicación con el dispositivo}\mbox{}\\\\%%
\begin{itemize}
\item Comandos
\item Información de estado
\item Datos
\end{itemize}
\paragraph{Buffering de datos}\mbox{}\\\\%%
\paragraph{Detección de errores}\mbox{}\\\\%%

\subsubsection{The control of the transfer of data from an external device to the processor might involve the following sequence of steps}
\begin{enumerate}
\item The processor interrogates the I/O module to check the status of the attached device.
\item The I/O module returns the device status.
\item If the device is operational and ready to transmit, the processor requests the transfer of data, by means of a command to the IO module.
\item  The I/O module obtains a unit of data (e.g., 8 or 16 bits) from the external device.
\item The data are transferred from the I/O module to the processor.
\end{enumerate}

/*Diagrama I/O pag 234 Will, 7pdf */

\subsubsection{Tecnicas para operaciones de E/S}
\begin{enumerate}
\item  E/S programada
\item  E/S manejada por interrupciones
\item  Acceso directo a memoria (DMA)
\end{enumerate}

/*pag 237*/

When large volumes of data are to be moved, a more efficient technique is
required: direct memory access (DMA).
DMA involves an additional module on the system bus. The DMA module
(Figure 7.12) is capable of mimicking the processor and, indeed, of taking over control
of the system from the processor. It needs to do this to transfer data to and from
memory over the system bus. For this purpose, the DMA module must use the bus
only when the processor does not need it, or it must force the processor to suspend
operation temporarily. The latter technique is more common and is referred to as
cycle stealing, because the DMA module in effect steals a bus cycle.
When the processor wishes to read or write a block of data, it issues a command
to the DMA module, by sending to the DMA module the following information:
\begin{enumerate}
\item  Whether a read or write is requested, using the read or write control line
between the processor and the DMA module.
\item  The address of the I/O device involved, communicated on the data lines.
\item  The starting location in memory to read from or write to, communicated on
the data lines and stored by the DMA module in its address register.
\item  The number of words to be read or written, again communicated via the data
lines and stored in the data count register.
\end{enumerate}
/*pag 249*/ 
The processor then continues with other work. It has delegated this I/O operation
to the DMA module. The DMA module transfers the entire block of data,
one word at a time, directly to or from memory, without going through the processor.
When the transfer is complete, the DMA module sends an interrupt signal to
the processor. Thus, the processor is involved only at the beginning and end of the
transfer

/*  250 Figure 7.13 DMA and Interrupt Breakpoints during an Instruction Cycle*/

/* 251 Figure 7.14 Alternative DMA Configurations*/

The DMA mechanism can be configured in a variety of ways. Some possibilities
are shown in Figure 7.14. In the first example, all modules share the same system
bus. The DMA module, acting as a surrogate processor, uses programmed I/O to
exchange data between memory and an I/O module through the DMA module. This
configuration, while it may be inexpensive, is clearly inefficient. As with processor-
controlled programmed I/O, each transfer of a word consumes two bus cycles
The number of required bus cycles can be cut substantially by integrating the
DMA and I/O functions. As Figure 7.14b indicates, this means that there is a path
between the DMA module and one or more I/O modules that does not include
the system bus. The DMA logic may actually be a part of an I/O module, or it may
be a separate module that controls one or more I/O modules. This concept can
be taken one step further by connecting I/O modules to the DMA module using
an I/O bus (Figure 7.14c). This reduces the number of I/O interfaces in the DMA
module to one and provides for an easily expandable configuration. In both of
these cases (Figures 7.14b and c), the system bus that the DMA module shares with
the processor and memory is used by the DMA module only to exchange data with
memory. The exchange of data between the DMA and I/O modules takes place off
the system bus.

\subsubsection{I /O Channels and Processors}
The Evolution of the I/O Function
As computer systems have evolved, there has been a pattern of increasing complexity
and sophistication of individual components. Nowhere is this more evident than
in the I/O function. We have already seen part of that evolution. The evolutionary
steps can be summarized as follows:
\begin{enumerate}
\item The CPU directly controls a peripheral device. This is seen in simple
microprocessor-controlled devices.
\item A controller or I/O module is added. The CPU uses programmed I/O without
interrupts. With this step, the CPU becomes somewhat divorced from the specific
details of external device interfaces.
\item  The same configuration as in step 2 is used, but now interrupts are employed.
The CPU need not spend time waiting for an I/O operation to be performed,
thus increasing efficiency.
\item The I/O module is given direct access to memory via DMA. It can now move
a block of data to or from memory without involving the CPU, except at the
beginning and end of the transfer.
\item  The I/O module is enhanced to become a processor in its own right, with a
specialized instruction set tailored for I/O. The CPU directs the I/O processor
to execute an I/O program in memory. The I/O processor fetches and executes
these instructions without CPU intervention. This allows the CPU to specify a
sequence of I/O activities and to be interrupted only when the entire sequence
has been performed.
\item The I/O module has a local memory of its own and is, in fact, a computer in its
own right. With this architecture, a large set of I/O devices can be controlled,
with minimal CPU involvement. A common use for such an architecture has
been to control communication with interactive terminals. The I/O processor
takes care of most of the tasks involved in controlling the terminals.
\end{enumerate}

\subsubsection{Characteristics of I/O Channels}
	The I/O channel represents an extension of the DMA concept. An I/O channel
	has the ability to execute I/O instructions, which gives it complete control over
	I/O operations. In a computer system with such devices, the CPU does not execute
	I/O instructions. Such instructions are stored in main memory to be executed by a
	special-purpose processor in the I/O channel itself. Thus, the CPU initiates an I/O
	transfer by instructing the I/O channel to execute a program in memory. The program
	will specify the device or devices, the area or areas of memory for storage,
	priority, and actions to be taken for certain error conditions. The I/O channel follows
	these instructions and controls the data transfer

	Two types of I/O channels are common, as illustrated in Figure 7.18. A
	selector channel controls multiple high-speed devices and, at any one time, is
	dedicated to the transfer of data with one of those devices. Thus, the I/O channel
	selects one device and effects the data transfer. Each device, or a small set of
	devices, is handled by a controller, or I/O module, that is much like the I/O modules
	we have been discussing. Thus, the I/O channel serves in place of the CPU
	in controlling these I/O controllers. A multiplexor channel can handle I/O with
	multiple devices at the same time. For low-speed devices, a byte multiplexor
	accepts or transmits characters as fast as possible to multiple devices. For example,
	the resultant character stream from three devices with different rates and individual
	streams A1A2A3A4 c, B1B2B3B4 c, and C1C2C3C4 c might be A1B1C1A2C2A3B2C3A4, and so on.
	

	/*263*/

\subsection{ADMINISTRACION DE MEMORIA }

\subsubsection{Sistema Operativo}
Software que administra los recursos del computador, provee servicios y controla la ejecución de otros programas
\subsubsection{Algunos servicios que provee}
\begin{itemize}
\item Schedule de procesos
\item Administración de memoria
\item Monitor
\item Parte residente del Sistema Operativo
\end{itemize}
/*284*/
In a uniprogramming system, main memory is divided into two parts: one part for
the OS (resident monitor) and one part for the program currently being executed.
In a multiprogramming system, the “user” part of memory is subdivided to accommodate
multiple processes. The task of subdivision is carried out dynamically by the
OS and is known as memory management.

\subsubsection{Administración de memoria simple}
\subsubsection{Sistema con uniprogramación}

Se divide la memoria en dos partes
Monitor del S.O.
Programa en ejecución en ese momento
\paragraph{Ventajas:}\mbox{}\\\\%%

Simplicidad

\paragraph{DESVentajas:}\mbox{}\\\\%%
\begin{itemize}
\item Desperdicio de memoria
\item Desaprovechamiento de los recursos del computador
\end{itemize}
Administración de memoria simple 
/*281*/

simple batch systems Early processors were very expensive, and therefore it
was important to maximize processor utilization. The wasted time due to scheduling
and setup time was unacceptable.
To improve utilization, simple batch operating systems were developed. With
such a system, also called a monitor, the user no longer has direct access to the processor.
Rather, the user submits the job on cards or tape to a computer operator,
who batches the jobs together sequentially and places the entire batch on an input
device, for use by the monitor.
To understand how this scheme works, let us look at it from two points of
view: that of the monitor and that of the processor. From the point of view of the
monitor, the monitor controls the sequence of events. For this to be so, much of the
monitor must always be in main memory and available for execution (Figure 8.3).
That portion is referred to as the resident monitor. The rest of the monitor consists
of utilities and common functions that are loaded as subroutines to the user program
at the beginning of any job that requires them. The monitor reads in jobs one
at a time from the input device (typically a card reader or magnetic tape drive). As it
is read in, the current job is placed in the user program area, and control is passed to
this job. When the job is completed, it returns control to the monitor, which immediately
reads in the next job. The results of each job are printed out for delivery to
the user.
Now consider this sequence from the point of view of the processor. At a certain
point in time, the processor is executing instructions from the portion of main memory
containing the monitor. These instructions cause the next job to be read in to
another portion of main memory. Once a job has been read in, the processor will
encounter in the monitor a branch instruction that instructs the processor to continue
execution at the start of the user program. The processor will then execute
the instruction in the user’s program until it encounters an ending or error condition.
Either event causes the processor to fetch its next instruction from the monitor
program. Thus the phrase “control is passed to a job” simply means that the processor
is now fetching and executing instructions in a user program, and “control is
returned to the monitor” means that the processor is now fetching and executing
instructions from the monitor program.
It should be clear that the monitor handles the scheduling problem. A batch of
jobs is queued up, and jobs are executed as rapidly as possible, with no intervening
idle time.
How about the job setup time? The monitor handles this as well. With each
job, instructions are included in a job control language (JCL). This is a special type
of programming language used to provide instructions to the monitor.


\subsubsection{Multiprogramming}
\begin{itemize}
\item Varios procesos de usuario en ejecución a lavez
\item Se divide la memoria de usuario entre los procesos en ejecución
\item Se comparte el tiempo de procesador entre los procesos en ejecución ( timeslice
\item Condiciones de finalización:
	\begin{itemize}
	\item Termina el trabajo
	\item Se detecta un error y se cancela
	\item Requiere una operación de E/S (suspensión)
	\item Termina el timeslice (suspención
	\end{itemize}
\end{itemize}

The simplest scheme for partitioning available memory is to use fixed-
sizepartitions, as shown in Figure 8.13. Note that, although the partitions are of fixed size, they need not be of equal size. When a process is brought into memory, it is placed in the
smallest available partition that will hold it.
Even with the use of unequal fixed-size partitions, there will be wasted memory.
In most cases, a process will not require exactly as much memory as provided by the partition.
A more efficient approach is to use variable-size partitions. When a process is
brought into memory, it is allocated exactly as much memory as it requires and no more.
As this example shows, this method starts out well, but eventually it leads to a
situation in which there are a lot of small holes in memory. As time goes on, memory
becomes more and more fragmented, and memory utilization declines. One
technique for overcoming this problem is compaction: From time to time, the OS
shifts the processes in memory to place all the free memory together in one block.
This is a time-consuming procedure, wasteful of processor time.

\subsubsection{Memory management: partitioning}
\begin{itemize}
\item Sistema con multiprogramación
\item La memoria de usuario se divide en particiones detamaño fijo:
	\begin{itemize}
	\item Iguales
	\item Distintas
	\end{itemize}
\end{itemize}
Ventajas:
\begin{itemize}
\item Permite compartir la memoria entre varios procesos
\end{itemize}
Desventajas:
\begin{itemize}
\item Desperdicio de memoria
\item Fragmentación interna (dentro de una partición)
\item Fragmentación externa (particiones no usadas)
\end{itemize}

\subsubsection{Memory management: swapping}
\begin{itemize}
\item Sistema con multiprogramación
\item Swapping
\item La memoria de usuario se divide en particiones de tamaño variable
\item Compactación para eliminar la fragmentación
\item Se usa un recurso de hardware (registro de reasignación) para la realocación
\item Realocación dinámica en tiempo de ejecución
\end{itemize}
Ventajas:
\begin{itemize}
\item Permite compartir la memoria entre varios procesos
\item Elimina el desperdicio por fragmentación interna.
\item Con la compactación se elimina además la fragmentación externa
\end{itemize}
Desventajas:
\begin{itemize}
\item La tarea de compactación es costosa
\end{itemize}

\subsubsection{Memory management: paging}
Both unequal fixed-size and variable-size
partitions are inefficient in the use of memory.
Suppose, however, that memory is partitioned into equal fixed-size
chunks that are relatively small, and that each process is also divided into small fixed-
size chunks of some size. Then the chunks of a program, known as pages, could be assigned to
available chunks of memory, known as frames, or page frames. At most, then, the
wasted space in memory for that process is a fraction of the last page.

\paragraph{Administración de memoria paginada simple}\mbox{}\\\\%%
\begin{itemize}
\item Sistema con multiprogramación
\item Se divide el address space del proceso en partes iguales (páginas)
\item Se divide la memoria principal en partes iguales ( frames)
\item Hay una tabla de páginas por proceso
\item Hay una lista de frames disponibles
\item Se cargan a memoria las páginas del proceso en los frames disponibles (no es necesario que sean contiguos)
\item Las direcciones lógicas se ven como número de página y un offset
\item Se traducen las direcciones lógicas en físicas con soporte del hardware
\item La paginación es transparente para el programador
\end{itemize}
Ventajas:
\begin{itemize}
\item Permite compartir la memoria entre varios procesos
\item Minimiza la fragmentación interna (solo existe dentro de la última página de cada proceso)
\item Elimina la fragmentación externa
\end{itemize}
Desventajas
\begin{itemize}
\item Se requiere subir todas las páginas del proceso a memoria
\item Se requieren estructuras de datos adicionales para mantener información de páginas y frames
\end{itemize}

\paragraph{Administración de memoria paginada simple}\mbox{}\\\\%%
\begin{itemize}
\item Sistema con multiprogramación
\item Solo se cargan las páginas necesarias para la ejecución de un proceso
\item Cuando se quiere acceder a una posición de memoria de una página no cargada se produce un page fault
\item El page fault dispara una interrupción por hardware atendida por el sistema operativo
\item Se levanta la página solicitada desde memoria secundaria (memoria virtual)
\item Algoritmos para reemplazo de páginas
\item Thrashing : el CPU pasa más tiempo reemplazando páginas que ejecutando instrucciones
\end{itemize}
Ventajas
\begin{itemize}
\item No es necesario cargar todas las páginas de un proceso a la vez
\item Maximiza el uso de la memoria al permitir cargar más procesos a la vez
\item Un proceso puede ocupar más memoria de la efectivamente instalada en el computador
\end{itemize}
Desventajas
\begin{itemize}
\item Mayor complejidad por la necesidad de implementar el reemplazo de páginas
\end{itemize}

\subsubsection{Administración de memoria por segmentación}
\begin{itemize}
\item Sistemas con multiprogramación
\item Generalmente visible al programador
\item La memoria del programa se ve como un conjunto de segmentos (múltiples espacios de direcciones)
\item Los segmentos son de tamaño variable y dinámico
\item El sistema operativo administra una tabla de segmentos por proceso
\item Permite separar datos e instrucciones
\item Permite dar privilegios y protección de memoria como por ej . lectura, escritura, ejecución. ( segmentation faults como mecanismos de excepción de hardware para accesos indebidos)
\item Las referencias a memoria se forman con un número de segmento y un offset dentro de él. Con ayuda de hardware (MMU \item Memory Management Unit ) se hacen las traducciones de las direcciones lógicas a físicas
\item Se pueden usar para implementar memoria virtual (solo se suben a
memoria física algunos segmentos por proceso)
\end{itemize}

Ventajas:
\begin{itemize}
\item Simplifica el manejo de estructuras de datos con crecimiento
\item Permite compartir información entre procesos dentro de un segmento
\item Permite aplicar protección/privilegios sobre un segmento fácilmente
\end{itemize}
Desventajas:
\begin{itemize}
\item Fragmentación externa en la memoria principal por no poder alojar un segmento
\item Hardware más complejo que memoria paginada para la traducción de direcciones
\end{itemize}

\subsubsection{Address Spaces}
The x86 includes hardware for both segmentation and paging. Both mechanisms can
be disabled, allowing the user to choose from four distinct views of memory:

\begin{itemize}
\item Unsegmented unpaged memory: In this case, the virtual address is the same
as the physical address. This is useful, for example, in low- complexity, high- performance controller applications.
\item Unsegmented paged memory: Here memory is viewed as a paged linear
address space. Protection and management of memory is done via paging.
This is favored by some operating systems (e.g., Berkeley UNIX).
\item Segmented unpaged memory: Here memory is viewed as a collection of
logical address spaces. The advantage of this view over a paged approach is
that it affords protection down to the level of a single byte, if necessary. Furthermore,
unlike paging, it guarantees that the translation table needed (the segment table) is on-chip
when the segment is in memory. Hence, segmented unpaged memory results in predictable access times.
\item Segmented paged memory: Segmentation is used to define logical memory
partitions subject to access control, and paging is used to manage the allocation
of memory within the partitions. Operating systems such as UNIX System favor this view.
\end{itemize}
%%%%%%%%%%%%%%%%%%%%


\section{Intel x86}
\section{Formato y Configuracion}

\subsection{Definición}
\begin{itemize}
\item Formato: Representación computacional
\item Configuración: Representación en una determinada base de un número en un formato
\end{itemize}

\subsection{Expansión y truncamiento}
\subsection{Definición}
\begin{itemize}
\item Expandir formato: Significa completar la representación computacional sin alterar el numero representado
en el mismo. 
\item Truncar formato: Descartar digitos de su representación sin alterar el número representado en el mismo.
\end{itemize}

\subsection{Formatos}

%Binario punto fijo sin signo
\subsubsection{Binario punto fijo sin signo}
\begin{itemize}
\item Base: 2
\item Representa: números enteros positivos
\item Máximo: $2^{n-1}_{10}$
\item Mínimo: 0
\end{itemize}

\paragraph{Almacenar}
\begin{enumerate}
\item Pasar el numero a base 2
\item completar con ceros a izquierda la capacidad del formato
\end{enumerate}
\paragraph{Recuperar}
\begin{enumerate}
\item Pasamos el numero de base 2 a la base deseada 
\end{enumerate}

%Binario de punto fijo con signo
\subsubsection{Binario de punto fijo con signo}
\begin{itemize}
\item Base: 2
\item Representa: Enteros positivos y negativos
\item Primer bit: reservado para el signo
\item Máximo: $2^{n-1}-1$
\item Mínimo: $-2^{n-1}$
\end{itemize}

\paragraph{Almacenar}
\begin{enumerate}
\item Pasar el numero a base 2
\item completar con ceros a izquierda la capacidad del formato
\item Si es un numero negativo hacerle el complemento a 2
\end{enumerate}
\paragraph{Recuperar}
\begin{enumerate}
\item Si el bit de signo es cero, se pasa de base 2 a base 10
\item Si el bit es 1, el numero es negativo, por lo que debemos complementarlo
\item Quitamos los ceros a la izquierda
\item Numero de base 2 a base 10
\item Colocamos el signo
\end{enumerate}

\paragraph{Expansión}
\begin{enumerate}
\item Se completa con el bit de signo a la izquierda
\end{enumerate}
\paragraph{Truncamiento}
\begin{enumerate}
\item Se extraen bits a la izquierda siempre y cuando no se esté alterando el bit de signo del número.
\end{enumerate}

%Empaquetado
\subsubsection{Empaquetado}
\begin{itemize}
\item Base: 16
\item Representa: Enteros positivos y negativos
\item Máximo: $10^{2n-1} -1$
\item Mínimo: $-10^{2n-1} +1$
\end{itemize}

\paragraph{Almacenar}
\begin{enumerate}
\item numero a base 10
\item Colocal cada digito decimal en un nibble dejando el ultimo nibble ya que en el mismo se almacena el signo.
\item Colocar en el ultimo nible el signo, CAFE = + , DF = -
\item Se rellena con 0 hasta alcanzar la cantidad de bytes usados
\end{enumerate}
\paragraph{Recuperar}
\begin{enumerate}
\item los pasos en orden inverso
\end{enumerate}

%Zoneado
\subsubsection{Zoneado}
\begin{itemize}
\item Base: 16
\item Representa: Enteros positivos y negativos
\item Máximo: $10^{n} -1$
\item Mínimo: $-10^{n} +1$
\end{itemize}

\paragraph{Almacenar}
\begin{enumerate}
\item numero a base 10
\item colocar cada uno de los digitos decimales en un nibble derecho
\item completar todos los nibbles de izquierda con F salvo el ultimo que se completa con el signo siguiendo las mismas reglas que para empaquetados.  CAFE = + , DF = -
\item se rellena con F0 hasta alcanzar la cantidad de bytes usados
\end{enumerate}
\paragraph{Recuperar}
\begin{enumerate}
\item los pasos en orden inverso
\end{enumerate}

\subsubsection{Binario de punto Flotante}
es la manera que tiene una arquitectura de representar a los numeros reales.
su notación cientifica se expresa de la siguiente manera $M x B^E$
M: Mantissa B: base E: Exponente

Un numero binario está normalizado si el digito de la izquierda del punto es igual a 1.

\paragraph{IEEE754}
\begin{itemize}
\item Precision Simple: signo: 1 bit exponente: 8 bits fraccion: 23 bits Exponente: Exceso 127
\item Precision Doble: signo: 1 bit exponente 11bits fraccion: 52 bits Exponente: Exceso 1023
\end{itemize}

\paragraph{Ancho de paso}\mbox{}\\\\
Marca cual es la distancia entre un flotante y su siguiente numero representable en el formato

\paragraph{Overflow}\mbox{}\\\\
el exponente excede el limite superior, tanto para mantisas positivas como para negatvias, dando lugar a +inf, - inf
\paragraph{Underflow}\mbox{}\\\\
el exponente excede el minimo valor permitido y caga en el intervalo (-inf, -0) y (+0,+inf)

\paragraph{Desnormalizados - Subnormales}\mbox{}\\\\
tienen como exponente al cero, y el bit implicito a la izquierda del punto binario, es ahora un cero implicito. la diferencia entre los desnormalizados y los normalizados es que, estos ultimos no permiten al cero como exponente. Los normalizados tienen 24 bits significativos, mientras que los normalizados poseen 23.

\begin{table}[h]
\begin{tabular}{lllll}
\cline{1-4}
\multicolumn{1}{|l|}{Normalizado} & \multicolumn{1}{l|}{+/-} & \multicolumn{1}{l|}{0<exp<max} &  \multicolumn{1}{l|}{cualquier patron de bits} &  \\ \cline{1-4}
\multicolumn{1}{|l|}{Desnormalizado} & \multicolumn{1}{l|}{+/-} & \multicolumn{1}{l|}{0} &  \multicolumn{1}{l|}{cualquier patron de bits != 0} &  \\ \cline{1-4}
\multicolumn{1}{|l|}{Cero} & \multicolumn{1}{l|}{+/-} & \multicolumn{1}{l|}{0} &  \multicolumn{1}{l|}{0} &  \\ \cline{1-4}
\multicolumn{1}{|l|}{Infinito} & \multicolumn{1}{l|}{+/-} & \multicolumn{1}{l|}{11...11} &  \multicolumn{1}{l|}{0} &  \\ \cline{1-4}
\multicolumn{1}{|l|}{NAN} & \multicolumn{1}{l|}{+/-} & \multicolumn{1}{l|}{11...11} &  \multicolumn{1}{l|}{Cualquier patron de bits != 0} &  \\ \cline{1-4}
\end{tabular}
\end{table}

\begin{itemize}
\item Infinito dividido Infinito = NAN
\item Cero + = 0 00000000 00000000000000000000000
\item Cero - = 1 00000000 00000000000000000000000
\item Infinito + = 0 11111111 00000000000000000000000
\item Infinito - = 1 11111111 00000000000000000000000
\item No normalizados/Subnormales (no se asume que haya que añadir un 1 al significando para obtener su valor).
\end{itemize}

\paragraph{Valores no numericos}\mbox{}\\\\
NaN (Not a number). 2 tipos, QNaN (quiet nan) y SNaN (signalling nan) Qnan = indeterminado Snan = operacion no valida

\begin{itemize}
\item Infinito dividido Infinito = NAN
\item qnan = 0 11111111 10000100000000000000000
\item snan = 1 11111111 00100010001001010101010
\end{itemize}

\begin{table}[h]
\begin{tabular}{lllll}
\cline{1-2}
\multicolumn{1}{|l|}{Operacion} & \multicolumn{1}{l|}{Resultado} &   \\ \cline{1-2}
\multicolumn{1}{|l|}{n $\pm$ infinito } & \multicolumn{1}{l|}{0} &   \\ \cline{1-2}
\multicolumn{1}{|l|}{$\pm$ infinito $\cdot$ $\pm$infinito } & \multicolumn{1}{l|}{$\pm$infinito} &   \\ \cline{1-2}
\multicolumn{1}{|l|}{ $\pm$n $\div$ 0   } & \multicolumn{1}{l|}{ $\pm$infinito } &   \\ \cline{1-2}
\multicolumn{1}{|l|}{ Infinito + Infinito   } & \multicolumn{1}{l|}{ Infinito } &   \\ \cline{1-2}
\multicolumn{1}{|l|}{Cualquier operación contra un NaN   } & \multicolumn{1}{l|}{ NaN } &   \\ \cline{1-2}
\multicolumn{1}{|l|}{ $\pm$0 $\div$ $\pm$0   } & \multicolumn{1}{l|}{ NaN } &   \\ \cline{1-2}
\multicolumn{1}{|l|}{ Infinito - Infinito   } & \multicolumn{1}{l|}{  NaN} &   \\ \cline{1-2}
\multicolumn{1}{|l|}{ $\pm$Infinito  $\div$ $\pm$Infinito  } & \multicolumn{1}{l|}{  NaN} &   \\ \cline{1-2}
\multicolumn{1}{|l|}{ $\pm$Infinito  $\cdot$ $\pm$0 } & \multicolumn{1}{l|}{  NaN} &   \\ \cline{1-2}
\end{tabular}
\end{table}

\paragraph{Almacenar -6,12510}\mbox{}\\
\begin{enumerate}    
  \item El bit 31 tomará el valor del signo del número. 
  \item Pasar a binario la mantisa decimal.
  \begin{itemize}
    \item $6=110_2$
    \item $0,125=0,001_2$
    \item $6,125=110,001_2$
  \end{itemize}
  \item Normalizar
  \begin{itemize}
    \item Desplazamiento a la derecha  -> Exponente negativo
    \item Desplazamiento a la izquierda -> Exponente positivo
    \item 1,10001 , exponente = 2
    \item 2 expresado en Exceso 127 es 129 $0000001_2$
  \end{itemize}
  \item Mantisa representada con bit implícito: 1,10001 -> 10001
   \item El número final es 1 10000001 100010000000000000000002 (Se agregan a la derecha los “0” necesarios para completar los 23 bits de la mantisa)
\end{enumerate}

\paragraph{Recuperar}\mbox{}\\\\
\begin{enumerate}    
  \item Se realizan los pasos en orden inverso
\end{enumerate}

\paragraph{Binario de Punto Flotante (IBM mainframe)}
\begin{itemize}
\item Base: 16
\item Representa: enteros con coma decimal positivos y negativos
\item Precision : imple 4 bytes, doble, 8 bytes, extendida 16 bytes.
\item estructura: S nnnnnnn dddddd
\begin{itemize}
    \item s = signo 1- 0+
    \item n = digitos de la caracteristica, en total son 7 bits se usan para calcular el exponente 
    \item C = E + 4016 donde E corresponde al exponente
    \item d =  mantisa normalizada -> 0,dddddd x $10^e_{16}$
  \end{itemize}
\end{itemize}

\paragraph{Almacenar - 321,54 10 -> Binario de punto flotante precisión simple}\mbox{}\\\\
\begin{enumerate}    
  \item $321,54_{10} = 141,8A3_{16}$
  \item $0,1418A3 x 10^3 _{16}$
  \item $C = E + 40_{16} = 3 + 40_{16} = 43_{16} en base 2 sería 100011_{2}$
  \item Agregamos el bit de signo: $1100011_{2}$ que en base 16 es $C3_{16}$
  \item Resultado final : $C31418A3_{16}$
\end{enumerate}

\paragraph{Ancho de paso }\mbox{}\\\\
distancia entre un flotante y su siguiente numero representable en el formato

\paragraph{Absorcion}\mbox{}\\\\
se da en las operaciones de suma y resta entre flotantes

\begin{itemize}    
  \item $A = 0,15A4 x 105_{16}$
  \item $B = 0,54F x 10^-2_{16}$
\end{itemize}

para poder operar entre flotantes debemos igualar los exponentes llevandolos al mayor de todos
\begin{itemize}    
  \item $A+B = 0.15A400 * 10^5 + 0,00...00 * 10^5 = 0,15A4000 * 10^5 _{16}$
\end{itemize}
lo minimo que se puede sumar es el ancho de paso del numero de mayor exponente

\subsection{Interrupciones}
\subsubsection{Definicion}
Mecanismos por los cuales otros modulos (E/S y memoria) interrumpen el normal procesamiento del CPU
\subsubsection{¿Para que existen?}
Para mejorar la eficiencia de procesamiento de un computador
\subsubsection{Clases de interrupciones}
\begin{itemize}
	\item programa
	\item  reloj
	\item e/s
	\item fallas de hardware
\end{itemize}

\subsubsection{Ciclo de instruccion}
\begin{itemize}
	\item Fetch instruction
	\item  Decode instruction
	\item Fetch operand
	\item Execute instruction
	\item Store result
	\item ----------------> interrupt breakpoint
	\item process interrupt
\end{itemize}

\subsubsection{Transferencia de control al S.O. (Handler)}
/*4pdf*/
	
\subsubsection{Procesamiento de interrupciones }
/*5pdf*/

/*6pdf ejemplo*/
	
\subsubsection{Multiples interrupciones}


\paragraph{Deshabilitar interrupciones (secuencia)}\mbox{}\\\\%%
/*7pdf*/
\paragraph{Priorizar interrupciones (anidadas)}\mbox{}\\\\%%
/*8pdf*/
	
\paragraph{Múltiples interrupciones - ejemplo}\mbox{}\\\\%%
Tres dispositivos de E/S
\begin{itemize}
	\item Línea de comunicación (Prioridad 1)
	\item  Disco (Prioridad 2)
	\item Impresora (Prioridad 3)
\end{itemize}
Eventos
\begin{itemize}
	\item T=10 Interrupción de Impresora
	\item T=15 Interrupción de línea de comunicación
	\item T=20 Interrupción de disco
\end{itemize}

/*10pdf*/

\subsection{MODULO DE E/S }

\subsubsection{Que hace}
Conecta a los periféricos con la CPU y la memoria a través del bus del sistema o switch central y permite la comunicación entre ellos
\subsubsection{Para que sirve}
Oculta detalles de timing, formatos y electro mecánica de los dispositivos periféricos
\subsubsection{Por que existe?}
\begin{itemize}
\item Amplia variedad de periféricos con distintos métodos de operación
\item La tasa de transferencia de los periféricos es generalmente mucho más lenta que la de la memoria y procesador
\item Los periféricos usan distintos formatos de datos y tamaños de palabra
\end{itemize}

/*4pdf*/ 230 stalling

\subsubsection{Interface interna - bus del sistema}
\begin{itemize}
\item Datos
\item Direcciones
\item Control
\end{itemize}

\subsubsection{Interface externa - perifericos}
\begin{itemize}
\item Datos
\item Estado
\item Control
\end{itemize}


\subsubsection{Funciones}
\paragraph{Control and Timing}\mbox{}\\\\%%
ontrola flujo de tráfico entre CPU/Memoria y periféricos
\paragraph{Comunicación con el procesador}\mbox{}\\\\%%
Decodificación de comandos: The I/O module accepts commands from the processor, typically sent as signals on the control bus. For example, an I/O module for a disk drive might accept the following commands: READ SECTOR, WRITE SECTOR, SEEK track number, and SCAN record ID. The latter two commands each include a parameter that is sent on the data bus
\begin{itemize}
	\item Datos: Data are exchanged between the processor and the I/O module over the data bus.
	\item Información de estado: Because peripherals are so slow, it is important to know the	status of the I/O module. For example, if an I/O module is asked to send data to the processor (read), it may not be ready to do so because it is still working on the previous I/O command. This fact can be reported with a status signal.	Common status signals are BUSY and READY. There may also be signals to report various error conditions.
	\item Reconocimiento de direcciones: Just as each word of memory has an address, so does each I/O device. Thus, an I/O module must recognize one unique address for each peripheral it controls.
\end{itemize}
\paragraph{Comunicación con el dispositivo}\mbox{}\\\\%%
\begin{itemize}
\item Comandos
\item Información de estado
\item Datos
\end{itemize}
\paragraph{Buffering de datos}\mbox{}\\\\%%
\paragraph{Detección de errores}\mbox{}\\\\%%

\subsubsection{The control of the transfer of data from an external device to the processor might involve the following sequence of steps}
\begin{enumerate}
\item The processor interrogates the I/O module to check the status of the attached device.
\item The I/O module returns the device status.
\item If the device is operational and ready to transmit, the processor requests the transfer of data, by means of a command to the IO module.
\item  The I/O module obtains a unit of data (e.g., 8 or 16 bits) from the external device.
\item The data are transferred from the I/O module to the processor.
\end{enumerate}

/*Diagrama I/O pag 234 Will, 7pdf */

\subsubsection{Tecnicas para operaciones de E/S}
\begin{enumerate}
\item  E/S programada
\item  E/S manejada por interrupciones
\item  Acceso directo a memoria (DMA)
\end{enumerate}

/*pag 237*/

When large volumes of data are to be moved, a more efficient technique is
required: direct memory access (DMA).
DMA involves an additional module on the system bus. The DMA module
(Figure 7.12) is capable of mimicking the processor and, indeed, of taking over control
of the system from the processor. It needs to do this to transfer data to and from
memory over the system bus. For this purpose, the DMA module must use the bus
only when the processor does not need it, or it must force the processor to suspend
operation temporarily. The latter technique is more common and is referred to as
cycle stealing, because the DMA module in effect steals a bus cycle.
When the processor wishes to read or write a block of data, it issues a command
to the DMA module, by sending to the DMA module the following information:
\begin{enumerate}
\item  Whether a read or write is requested, using the read or write control line
between the processor and the DMA module.
\item  The address of the I/O device involved, communicated on the data lines.
\item  The starting location in memory to read from or write to, communicated on
the data lines and stored by the DMA module in its address register.
\item  The number of words to be read or written, again communicated via the data
lines and stored in the data count register.
\end{enumerate}
/*pag 249*/ 
The processor then continues with other work. It has delegated this I/O operation
to the DMA module. The DMA module transfers the entire block of data,
one word at a time, directly to or from memory, without going through the processor.
When the transfer is complete, the DMA module sends an interrupt signal to
the processor. Thus, the processor is involved only at the beginning and end of the
transfer

/*  250 Figure 7.13 DMA and Interrupt Breakpoints during an Instruction Cycle*/

/* 251 Figure 7.14 Alternative DMA Configurations*/

The DMA mechanism can be configured in a variety of ways. Some possibilities
are shown in Figure 7.14. In the first example, all modules share the same system
bus. The DMA module, acting as a surrogate processor, uses programmed I/O to
exchange data between memory and an I/O module through the DMA module. This
configuration, while it may be inexpensive, is clearly inefficient. As with processor-
controlled programmed I/O, each transfer of a word consumes two bus cycles
The number of required bus cycles can be cut substantially by integrating the
DMA and I/O functions. As Figure 7.14b indicates, this means that there is a path
between the DMA module and one or more I/O modules that does not include
the system bus. The DMA logic may actually be a part of an I/O module, or it may
be a separate module that controls one or more I/O modules. This concept can
be taken one step further by connecting I/O modules to the DMA module using
an I/O bus (Figure 7.14c). This reduces the number of I/O interfaces in the DMA
module to one and provides for an easily expandable configuration. In both of
these cases (Figures 7.14b and c), the system bus that the DMA module shares with
the processor and memory is used by the DMA module only to exchange data with
memory. The exchange of data between the DMA and I/O modules takes place off
the system bus.

\subsubsection{I /O Channels and Processors}
The Evolution of the I/O Function
As computer systems have evolved, there has been a pattern of increasing complexity
and sophistication of individual components. Nowhere is this more evident than
in the I/O function. We have already seen part of that evolution. The evolutionary
steps can be summarized as follows:
\begin{enumerate}
\item The CPU directly controls a peripheral device. This is seen in simple
microprocessor-controlled devices.
\item A controller or I/O module is added. The CPU uses programmed I/O without
interrupts. With this step, the CPU becomes somewhat divorced from the specific
details of external device interfaces.
\item  The same configuration as in step 2 is used, but now interrupts are employed.
The CPU need not spend time waiting for an I/O operation to be performed,
thus increasing efficiency.
\item The I/O module is given direct access to memory via DMA. It can now move
a block of data to or from memory without involving the CPU, except at the
beginning and end of the transfer.
\item  The I/O module is enhanced to become a processor in its own right, with a
specialized instruction set tailored for I/O. The CPU directs the I/O processor
to execute an I/O program in memory. The I/O processor fetches and executes
these instructions without CPU intervention. This allows the CPU to specify a
sequence of I/O activities and to be interrupted only when the entire sequence
has been performed.
\item The I/O module has a local memory of its own and is, in fact, a computer in its
own right. With this architecture, a large set of I/O devices can be controlled,
with minimal CPU involvement. A common use for such an architecture has
been to control communication with interactive terminals. The I/O processor
takes care of most of the tasks involved in controlling the terminals.
\end{enumerate}

\subsubsection{Characteristics of I/O Channels}
	The I/O channel represents an extension of the DMA concept. An I/O channel
	has the ability to execute I/O instructions, which gives it complete control over
	I/O operations. In a computer system with such devices, the CPU does not execute
	I/O instructions. Such instructions are stored in main memory to be executed by a
	special-purpose processor in the I/O channel itself. Thus, the CPU initiates an I/O
	transfer by instructing the I/O channel to execute a program in memory. The program
	will specify the device or devices, the area or areas of memory for storage,
	priority, and actions to be taken for certain error conditions. The I/O channel follows
	these instructions and controls the data transfer

	Two types of I/O channels are common, as illustrated in Figure 7.18. A
	selector channel controls multiple high-speed devices and, at any one time, is
	dedicated to the transfer of data with one of those devices. Thus, the I/O channel
	selects one device and effects the data transfer. Each device, or a small set of
	devices, is handled by a controller, or I/O module, that is much like the I/O modules
	we have been discussing. Thus, the I/O channel serves in place of the CPU
	in controlling these I/O controllers. A multiplexor channel can handle I/O with
	multiple devices at the same time. For low-speed devices, a byte multiplexor
	accepts or transmits characters as fast as possible to multiple devices. For example,
	the resultant character stream from three devices with different rates and individual
	streams A1A2A3A4 c, B1B2B3B4 c, and C1C2C3C4 c might be A1B1C1A2C2A3B2C3A4, and so on.
	

	/*263*/



%%%%%%%%%%%%%%%%%%%%%%%%%%%%%%%%%%%%%%%%%%
% Academic Title Page
% LaTeX Template
% Version 2.0 (17/7/17)
%
% This template was downloaded from:
% http://www.LaTeXTemplates.com
%
% Original author:
% WikiBooks (LaTeX - Title Creation) with modifications by:
% Vel (vel@latextemplates.com)
%
% License:
% CC BY-NC-SA 3.0 (http://creativecommons.org/licenses/by-nc-sa/3.0/)
% 
% Instructions for using this template:
% This title page is capable of being compiled as is. This is not useful for 
% including it in another document. To do this, you have two options: 
%
% 1) Copy/paste everything between \begin{document} and \end{document} 
% starting at \begin{titlepage} and paste this into another LaTeX file where you 
% want your title page.
% OR
% 2) Remove everything outside the \begin{titlepage} and \end{titlepage}, rename
% this file and move it to the same directory as the LaTeX file you wish to add it to. 
% Then add \input{./<new filename>.tex} to your LaTeX file where you want your
% title page.
%
%%%%%%%%%%%%%%%%%%%%%%%%%%%%%%%%%%%%%%%%%

%----------------------------------------------------------------------------------------
%	PACKAGES AND OTHER DOCUMENT CONFIGURATIONS
%----------------------------------------------------------------------------------------


\documentclass[11pt]{article}
\usepackage{geometry}
\usepackage{graphicx}
\usepackage{url}
\usepackage[utf8]{inputenc} % Required for inputting international characters
\usepackage[T1]{fontenc} % Output font encoding for international characters

\usepackage{mathpazo} % Palatino font

\begin{document}
\setcounter{secnumdepth}{5}
%----------------------------------------------------------------------------------------
%	TITLE PAGE
%----------------------------------------------------------------------------------------

\begin{titlepage} % Suppresses displaying the page number on the title page and the subsequent page counts as page 1
	\newcommand{\HRule}{\rule{\linewidth}{0.5mm}} % Defines a new command for horizontal lines, change thickness here
	
	\center % Centre everything on the page
	
	%------------------------------------------------
	%	Headings
	%------------------------------------------------
	
	\textsc{\LARGE Universidad de Buenos Aires}\\[1.5cm] % Main heading such as the name of your university/college
	
	\textsc{\Large Facultad de Ingeniería}\\[0.5cm] % Major heading such as course name
	
	\textsc{\large Resumen}\\[0.5cm] % Minor heading such as course title
	
	%------------------------------------------------
	%	Title
	%------------------------------------------------
	
	\HRule\\[0.4cm]
	
	{\huge\bfseries Organización del Computador \newline 75.03 \& 95.57}\\[0.4cm] % Title of your document
	
	\HRule\\[1.5cm]
	
	%------------------------------------------------
	%	Author(s)
	%------------------------------------------------
	
	%\begin{minipage}{0.4\textwidth}
	%	\begin{flushleft}
	%		\large
	%		\textit{Autor}\\
	%		\textsc{Anzu} % Your name
	%	\end{flushleft}
	%\end{minipage}
	%~
	%\begin{minipage}{0.4\textwidth}
	%	\begin{flushright}
	%		\large
	%		\textit{Supervisor}\\
	%		\textsc{Anzu} % Supervisor's name
	%	\end{flushright}
	%\end{minipage}
	
	% If you don't want a supervisor, uncomment the two lines below and comment the code above
	{\large\textit{Autor}}\\
	\textsc{Anzu} % Your name
	
	%------------------------------------------------
	%	Date
	%------------------------------------------------
	
	\vfill\vfill\vfill % Position the date 3/4 down the remaining page
	
	{\large\today} % Date, change the \today to a set date if you want to be precise
	
	%------------------------------------------------
	%	Logo
	%------------------------------------------------
	
	%\vfill\vfill
	%\includegraphics[width=0.2\textwidth]{placeholder.jpg}\\[1cm] % Include a department/university logo - this will require the graphicx package
	 
	%----------------------------------------------------------------------------------------
	
	\vfill % Push the date up 1/4 of the remaining page
	
\end{titlepage}

%----------------------------------------------------------------------------------------

\tableofcontents

\newpage

\section{Intel x86}
\section{Formato y Configuracion}

\subsection{Definición}
\begin{itemize}
\item Formato: Representación computacional
\item Configuración: Representación en una determinada base de un número en un formato
\end{itemize}

\subsection{Expansión y truncamiento}
\subsection{Definición}
\begin{itemize}
\item Expandir formato: Significa completar la representación computacional sin alterar el numero representado
en el mismo. 
\item Truncar formato: Descartar digitos de su representación sin alterar el número representado en el mismo.
\end{itemize}

\subsection{Formatos}

%Binario punto fijo sin signo
\subsubsection{Binario punto fijo sin signo}
\begin{itemize}
\item Base: 2
\item Representa: números enteros positivos
\item Máximo: $2^{n-1}_{10}$
\item Mínimo: 0
\end{itemize}

\paragraph{Almacenar}
\begin{enumerate}
\item Pasar el numero a base 2
\item completar con ceros a izquierda la capacidad del formato
\end{enumerate}
\paragraph{Recuperar}
\begin{enumerate}
\item Pasamos el numero de base 2 a la base deseada 
\end{enumerate}

%Binario de punto fijo con signo
\subsubsection{Binario de punto fijo con signo}
\begin{itemize}
\item Base: 2
\item Representa: Enteros positivos y negativos
\item Primer bit: reservado para el signo
\item Máximo: $2^{n-1}-1$
\item Mínimo: $-2^{n-1}$
\end{itemize}

\paragraph{Almacenar}
\begin{enumerate}
\item Pasar el numero a base 2
\item completar con ceros a izquierda la capacidad del formato
\item Si es un numero negativo hacerle el complemento a 2
\end{enumerate}
\paragraph{Recuperar}
\begin{enumerate}
\item Si el bit de signo es cero, se pasa de base 2 a base 10
\item Si el bit es 1, el numero es negativo, por lo que debemos complementarlo
\item Quitamos los ceros a la izquierda
\item Numero de base 2 a base 10
\item Colocamos el signo
\end{enumerate}

\paragraph{Expansión}
\begin{enumerate}
\item Se completa con el bit de signo a la izquierda
\end{enumerate}
\paragraph{Truncamiento}
\begin{enumerate}
\item Se extraen bits a la izquierda siempre y cuando no se esté alterando el bit de signo del número.
\end{enumerate}

%Empaquetado
\subsubsection{Empaquetado}
\begin{itemize}
\item Base: 16
\item Representa: Enteros positivos y negativos
\item Máximo: $10^{2n-1} -1$
\item Mínimo: $-10^{2n-1} +1$
\end{itemize}

\paragraph{Almacenar}
\begin{enumerate}
\item numero a base 10
\item Colocal cada digito decimal en un nibble dejando el ultimo nibble ya que en el mismo se almacena el signo.
\item Colocar en el ultimo nible el signo, CAFE = + , DF = -
\item Se rellena con 0 hasta alcanzar la cantidad de bytes usados
\end{enumerate}
\paragraph{Recuperar}
\begin{enumerate}
\item los pasos en orden inverso
\end{enumerate}

%Zoneado
\subsubsection{Zoneado}
\begin{itemize}
\item Base: 16
\item Representa: Enteros positivos y negativos
\item Máximo: $10^{n} -1$
\item Mínimo: $-10^{n} +1$
\end{itemize}

\paragraph{Almacenar}
\begin{enumerate}
\item numero a base 10
\item colocar cada uno de los digitos decimales en un nibble derecho
\item completar todos los nibbles de izquierda con F salvo el ultimo que se completa con el signo siguiendo las mismas reglas que para empaquetados.  CAFE = + , DF = -
\item se rellena con F0 hasta alcanzar la cantidad de bytes usados
\end{enumerate}
\paragraph{Recuperar}
\begin{enumerate}
\item los pasos en orden inverso
\end{enumerate}

\subsubsection{Binario de punto Flotante}
es la manera que tiene una arquitectura de representar a los numeros reales.
su notación cientifica se expresa de la siguiente manera $M x B^E$
M: Mantissa B: base E: Exponente

Un numero binario está normalizado si el digito de la izquierda del punto es igual a 1.

\paragraph{IEEE754}
\begin{itemize}
\item Precision Simple: signo: 1 bit exponente: 8 bits fraccion: 23 bits Exponente: Exceso 127
\item Precision Doble: signo: 1 bit exponente 11bits fraccion: 52 bits Exponente: Exceso 1023
\end{itemize}

\paragraph{Ancho de paso}\mbox{}\\\\
Marca cual es la distancia entre un flotante y su siguiente numero representable en el formato

\paragraph{Overflow}\mbox{}\\\\
el exponente excede el limite superior, tanto para mantisas positivas como para negatvias, dando lugar a +inf, - inf
\paragraph{Underflow}\mbox{}\\\\
el exponente excede el minimo valor permitido y caga en el intervalo (-inf, -0) y (+0,+inf)

\paragraph{Desnormalizados - Subnormales}\mbox{}\\\\
tienen como exponente al cero, y el bit implicito a la izquierda del punto binario, es ahora un cero implicito. la diferencia entre los desnormalizados y los normalizados es que, estos ultimos no permiten al cero como exponente. Los normalizados tienen 24 bits significativos, mientras que los normalizados poseen 23.

\begin{table}[h]
\begin{tabular}{lllll}
\cline{1-4}
\multicolumn{1}{|l|}{Normalizado} & \multicolumn{1}{l|}{+/-} & \multicolumn{1}{l|}{0<exp<max} &  \multicolumn{1}{l|}{cualquier patron de bits} &  \\ \cline{1-4}
\multicolumn{1}{|l|}{Desnormalizado} & \multicolumn{1}{l|}{+/-} & \multicolumn{1}{l|}{0} &  \multicolumn{1}{l|}{cualquier patron de bits != 0} &  \\ \cline{1-4}
\multicolumn{1}{|l|}{Cero} & \multicolumn{1}{l|}{+/-} & \multicolumn{1}{l|}{0} &  \multicolumn{1}{l|}{0} &  \\ \cline{1-4}
\multicolumn{1}{|l|}{Infinito} & \multicolumn{1}{l|}{+/-} & \multicolumn{1}{l|}{11...11} &  \multicolumn{1}{l|}{0} &  \\ \cline{1-4}
\multicolumn{1}{|l|}{NAN} & \multicolumn{1}{l|}{+/-} & \multicolumn{1}{l|}{11...11} &  \multicolumn{1}{l|}{Cualquier patron de bits != 0} &  \\ \cline{1-4}
\end{tabular}
\end{table}

\begin{itemize}
\item Infinito dividido Infinito = NAN
\item Cero + = 0 00000000 00000000000000000000000
\item Cero - = 1 00000000 00000000000000000000000
\item Infinito + = 0 11111111 00000000000000000000000
\item Infinito - = 1 11111111 00000000000000000000000
\item No normalizados/Subnormales (no se asume que haya que añadir un 1 al significando para obtener su valor).
\end{itemize}

\paragraph{Valores no numericos}\mbox{}\\\\
NaN (Not a number). 2 tipos, QNaN (quiet nan) y SNaN (signalling nan) Qnan = indeterminado Snan = operacion no valida

\begin{itemize}
\item Infinito dividido Infinito = NAN
\item qnan = 0 11111111 10000100000000000000000
\item snan = 1 11111111 00100010001001010101010
\end{itemize}

\begin{table}[h]
\begin{tabular}{lllll}
\cline{1-2}
\multicolumn{1}{|l|}{Operacion} & \multicolumn{1}{l|}{Resultado} &   \\ \cline{1-2}
\multicolumn{1}{|l|}{n $\pm$ infinito } & \multicolumn{1}{l|}{0} &   \\ \cline{1-2}
\multicolumn{1}{|l|}{$\pm$ infinito $\cdot$ $\pm$infinito } & \multicolumn{1}{l|}{$\pm$infinito} &   \\ \cline{1-2}
\multicolumn{1}{|l|}{ $\pm$n $\div$ 0   } & \multicolumn{1}{l|}{ $\pm$infinito } &   \\ \cline{1-2}
\multicolumn{1}{|l|}{ Infinito + Infinito   } & \multicolumn{1}{l|}{ Infinito } &   \\ \cline{1-2}
\multicolumn{1}{|l|}{Cualquier operación contra un NaN   } & \multicolumn{1}{l|}{ NaN } &   \\ \cline{1-2}
\multicolumn{1}{|l|}{ $\pm$0 $\div$ $\pm$0   } & \multicolumn{1}{l|}{ NaN } &   \\ \cline{1-2}
\multicolumn{1}{|l|}{ Infinito - Infinito   } & \multicolumn{1}{l|}{  NaN} &   \\ \cline{1-2}
\multicolumn{1}{|l|}{ $\pm$Infinito  $\div$ $\pm$Infinito  } & \multicolumn{1}{l|}{  NaN} &   \\ \cline{1-2}
\multicolumn{1}{|l|}{ $\pm$Infinito  $\cdot$ $\pm$0 } & \multicolumn{1}{l|}{  NaN} &   \\ \cline{1-2}
\end{tabular}
\end{table}

\paragraph{Almacenar -6,12510}\mbox{}\\
\begin{enumerate}    
  \item El bit 31 tomará el valor del signo del número. 
  \item Pasar a binario la mantisa decimal.
  \begin{itemize}
    \item $6=110_2$
    \item $0,125=0,001_2$
    \item $6,125=110,001_2$
  \end{itemize}
  \item Normalizar
  \begin{itemize}
    \item Desplazamiento a la derecha  -> Exponente negativo
    \item Desplazamiento a la izquierda -> Exponente positivo
    \item 1,10001 , exponente = 2
    \item 2 expresado en Exceso 127 es 129 $0000001_2$
  \end{itemize}
  \item Mantisa representada con bit implícito: 1,10001 -> 10001
   \item El número final es 1 10000001 100010000000000000000002 (Se agregan a la derecha los “0” necesarios para completar los 23 bits de la mantisa)
\end{enumerate}

\paragraph{Recuperar}\mbox{}\\\\
\begin{enumerate}    
  \item Se realizan los pasos en orden inverso
\end{enumerate}

\paragraph{Binario de Punto Flotante (IBM mainframe)}
\begin{itemize}
\item Base: 16
\item Representa: enteros con coma decimal positivos y negativos
\item Precision : imple 4 bytes, doble, 8 bytes, extendida 16 bytes.
\item estructura: S nnnnnnn dddddd
\begin{itemize}
    \item s = signo 1- 0+
    \item n = digitos de la caracteristica, en total son 7 bits se usan para calcular el exponente 
    \item C = E + 4016 donde E corresponde al exponente
    \item d =  mantisa normalizada -> 0,dddddd x $10^e_{16}$
  \end{itemize}
\end{itemize}

\paragraph{Almacenar - 321,54 10 -> Binario de punto flotante precisión simple}\mbox{}\\\\
\begin{enumerate}    
  \item $321,54_{10} = 141,8A3_{16}$
  \item $0,1418A3 x 10^3 _{16}$
  \item $C = E + 40_{16} = 3 + 40_{16} = 43_{16} en base 2 sería 100011_{2}$
  \item Agregamos el bit de signo: $1100011_{2}$ que en base 16 es $C3_{16}$
  \item Resultado final : $C31418A3_{16}$
\end{enumerate}

\paragraph{Ancho de paso }\mbox{}\\\\
distancia entre un flotante y su siguiente numero representable en el formato

\paragraph{Absorcion}\mbox{}\\\\
se da en las operaciones de suma y resta entre flotantes

\begin{itemize}    
  \item $A = 0,15A4 x 105_{16}$
  \item $B = 0,54F x 10^-2_{16}$
\end{itemize}

para poder operar entre flotantes debemos igualar los exponentes llevandolos al mayor de todos
\begin{itemize}    
  \item $A+B = 0.15A400 * 10^5 + 0,00...00 * 10^5 = 0,15A4000 * 10^5 _{16}$
\end{itemize}
lo minimo que se puede sumar es el ancho de paso del numero de mayor exponente

\subsection{Interrupciones}
\subsubsection{Definicion}
Mecanismos por los cuales otros modulos (E/S y memoria) interrumpen el normal procesamiento del CPU
\subsubsection{¿Para que existen?}
Para mejorar la eficiencia de procesamiento de un computador
\subsubsection{Clases de interrupciones}
\begin{itemize}
	\item programa
	\item  reloj
	\item e/s
	\item fallas de hardware
\end{itemize}

\subsubsection{Ciclo de instruccion}
\begin{itemize}
	\item Fetch instruction
	\item  Decode instruction
	\item Fetch operand
	\item Execute instruction
	\item Store result
	\item ----------------> interrupt breakpoint
	\item process interrupt
\end{itemize}

\subsubsection{Transferencia de control al S.O. (Handler)}
/*4pdf*/
	
\subsubsection{Procesamiento de interrupciones }
/*5pdf*/

/*6pdf ejemplo*/
	
\subsubsection{Multiples interrupciones}


\paragraph{Deshabilitar interrupciones (secuencia)}\mbox{}\\\\%%
/*7pdf*/
\paragraph{Priorizar interrupciones (anidadas)}\mbox{}\\\\%%
/*8pdf*/
	
\paragraph{Múltiples interrupciones - ejemplo}\mbox{}\\\\%%
Tres dispositivos de E/S
\begin{itemize}
	\item Línea de comunicación (Prioridad 1)
	\item  Disco (Prioridad 2)
	\item Impresora (Prioridad 3)
\end{itemize}
Eventos
\begin{itemize}
	\item T=10 Interrupción de Impresora
	\item T=15 Interrupción de línea de comunicación
	\item T=20 Interrupción de disco
\end{itemize}

/*10pdf*/


\section{Intel x86}
\section{Formato y Configuracion}

\subsection{Definición}
\begin{itemize}
\item Formato: Representación computacional
\item Configuración: Representación en una determinada base de un número en un formato
\end{itemize}

\subsection{Expansión y truncamiento}
\subsection{Definición}
\begin{itemize}
\item Expandir formato: Significa completar la representación computacional sin alterar el numero representado
en el mismo. 
\item Truncar formato: Descartar digitos de su representación sin alterar el número representado en el mismo.
\end{itemize}

\subsection{Formatos}

%Binario punto fijo sin signo
\subsubsection{Binario punto fijo sin signo}
\begin{itemize}
\item Base: 2
\item Representa: números enteros positivos
\item Máximo: $2^{n-1}_{10}$
\item Mínimo: 0
\end{itemize}

\paragraph{Almacenar}
\begin{enumerate}
\item Pasar el numero a base 2
\item completar con ceros a izquierda la capacidad del formato
\end{enumerate}
\paragraph{Recuperar}
\begin{enumerate}
\item Pasamos el numero de base 2 a la base deseada 
\end{enumerate}

%Binario de punto fijo con signo
\subsubsection{Binario de punto fijo con signo}
\begin{itemize}
\item Base: 2
\item Representa: Enteros positivos y negativos
\item Primer bit: reservado para el signo
\item Máximo: $2^{n-1}-1$
\item Mínimo: $-2^{n-1}$
\end{itemize}

\paragraph{Almacenar}
\begin{enumerate}
\item Pasar el numero a base 2
\item completar con ceros a izquierda la capacidad del formato
\item Si es un numero negativo hacerle el complemento a 2
\end{enumerate}
\paragraph{Recuperar}
\begin{enumerate}
\item Si el bit de signo es cero, se pasa de base 2 a base 10
\item Si el bit es 1, el numero es negativo, por lo que debemos complementarlo
\item Quitamos los ceros a la izquierda
\item Numero de base 2 a base 10
\item Colocamos el signo
\end{enumerate}

\paragraph{Expansión}
\begin{enumerate}
\item Se completa con el bit de signo a la izquierda
\end{enumerate}
\paragraph{Truncamiento}
\begin{enumerate}
\item Se extraen bits a la izquierda siempre y cuando no se esté alterando el bit de signo del número.
\end{enumerate}

%Empaquetado
\subsubsection{Empaquetado}
\begin{itemize}
\item Base: 16
\item Representa: Enteros positivos y negativos
\item Máximo: $10^{2n-1} -1$
\item Mínimo: $-10^{2n-1} +1$
\end{itemize}

\paragraph{Almacenar}
\begin{enumerate}
\item numero a base 10
\item Colocal cada digito decimal en un nibble dejando el ultimo nibble ya que en el mismo se almacena el signo.
\item Colocar en el ultimo nible el signo, CAFE = + , DF = -
\item Se rellena con 0 hasta alcanzar la cantidad de bytes usados
\end{enumerate}
\paragraph{Recuperar}
\begin{enumerate}
\item los pasos en orden inverso
\end{enumerate}

%Zoneado
\subsubsection{Zoneado}
\begin{itemize}
\item Base: 16
\item Representa: Enteros positivos y negativos
\item Máximo: $10^{n} -1$
\item Mínimo: $-10^{n} +1$
\end{itemize}

\paragraph{Almacenar}
\begin{enumerate}
\item numero a base 10
\item colocar cada uno de los digitos decimales en un nibble derecho
\item completar todos los nibbles de izquierda con F salvo el ultimo que se completa con el signo siguiendo las mismas reglas que para empaquetados.  CAFE = + , DF = -
\item se rellena con F0 hasta alcanzar la cantidad de bytes usados
\end{enumerate}
\paragraph{Recuperar}
\begin{enumerate}
\item los pasos en orden inverso
\end{enumerate}

\subsubsection{Binario de punto Flotante}
es la manera que tiene una arquitectura de representar a los numeros reales.
su notación cientifica se expresa de la siguiente manera $M x B^E$
M: Mantissa B: base E: Exponente

Un numero binario está normalizado si el digito de la izquierda del punto es igual a 1.

\paragraph{IEEE754}
\begin{itemize}
\item Precision Simple: signo: 1 bit exponente: 8 bits fraccion: 23 bits Exponente: Exceso 127
\item Precision Doble: signo: 1 bit exponente 11bits fraccion: 52 bits Exponente: Exceso 1023
\end{itemize}

\paragraph{Ancho de paso}\mbox{}\\\\
Marca cual es la distancia entre un flotante y su siguiente numero representable en el formato

\paragraph{Overflow}\mbox{}\\\\
el exponente excede el limite superior, tanto para mantisas positivas como para negatvias, dando lugar a +inf, - inf
\paragraph{Underflow}\mbox{}\\\\
el exponente excede el minimo valor permitido y caga en el intervalo (-inf, -0) y (+0,+inf)

\paragraph{Desnormalizados - Subnormales}\mbox{}\\\\
tienen como exponente al cero, y el bit implicito a la izquierda del punto binario, es ahora un cero implicito. la diferencia entre los desnormalizados y los normalizados es que, estos ultimos no permiten al cero como exponente. Los normalizados tienen 24 bits significativos, mientras que los normalizados poseen 23.

\begin{table}[h]
\begin{tabular}{lllll}
\cline{1-4}
\multicolumn{1}{|l|}{Normalizado} & \multicolumn{1}{l|}{+/-} & \multicolumn{1}{l|}{0<exp<max} &  \multicolumn{1}{l|}{cualquier patron de bits} &  \\ \cline{1-4}
\multicolumn{1}{|l|}{Desnormalizado} & \multicolumn{1}{l|}{+/-} & \multicolumn{1}{l|}{0} &  \multicolumn{1}{l|}{cualquier patron de bits != 0} &  \\ \cline{1-4}
\multicolumn{1}{|l|}{Cero} & \multicolumn{1}{l|}{+/-} & \multicolumn{1}{l|}{0} &  \multicolumn{1}{l|}{0} &  \\ \cline{1-4}
\multicolumn{1}{|l|}{Infinito} & \multicolumn{1}{l|}{+/-} & \multicolumn{1}{l|}{11...11} &  \multicolumn{1}{l|}{0} &  \\ \cline{1-4}
\multicolumn{1}{|l|}{NAN} & \multicolumn{1}{l|}{+/-} & \multicolumn{1}{l|}{11...11} &  \multicolumn{1}{l|}{Cualquier patron de bits != 0} &  \\ \cline{1-4}
\end{tabular}
\end{table}

\begin{itemize}
\item Infinito dividido Infinito = NAN
\item Cero + = 0 00000000 00000000000000000000000
\item Cero - = 1 00000000 00000000000000000000000
\item Infinito + = 0 11111111 00000000000000000000000
\item Infinito - = 1 11111111 00000000000000000000000
\item No normalizados/Subnormales (no se asume que haya que añadir un 1 al significando para obtener su valor).
\end{itemize}

\paragraph{Valores no numericos}\mbox{}\\\\
NaN (Not a number). 2 tipos, QNaN (quiet nan) y SNaN (signalling nan) Qnan = indeterminado Snan = operacion no valida

\begin{itemize}
\item Infinito dividido Infinito = NAN
\item qnan = 0 11111111 10000100000000000000000
\item snan = 1 11111111 00100010001001010101010
\end{itemize}

\begin{table}[h]
\begin{tabular}{lllll}
\cline{1-2}
\multicolumn{1}{|l|}{Operacion} & \multicolumn{1}{l|}{Resultado} &   \\ \cline{1-2}
\multicolumn{1}{|l|}{n $\pm$ infinito } & \multicolumn{1}{l|}{0} &   \\ \cline{1-2}
\multicolumn{1}{|l|}{$\pm$ infinito $\cdot$ $\pm$infinito } & \multicolumn{1}{l|}{$\pm$infinito} &   \\ \cline{1-2}
\multicolumn{1}{|l|}{ $\pm$n $\div$ 0   } & \multicolumn{1}{l|}{ $\pm$infinito } &   \\ \cline{1-2}
\multicolumn{1}{|l|}{ Infinito + Infinito   } & \multicolumn{1}{l|}{ Infinito } &   \\ \cline{1-2}
\multicolumn{1}{|l|}{Cualquier operación contra un NaN   } & \multicolumn{1}{l|}{ NaN } &   \\ \cline{1-2}
\multicolumn{1}{|l|}{ $\pm$0 $\div$ $\pm$0   } & \multicolumn{1}{l|}{ NaN } &   \\ \cline{1-2}
\multicolumn{1}{|l|}{ Infinito - Infinito   } & \multicolumn{1}{l|}{  NaN} &   \\ \cline{1-2}
\multicolumn{1}{|l|}{ $\pm$Infinito  $\div$ $\pm$Infinito  } & \multicolumn{1}{l|}{  NaN} &   \\ \cline{1-2}
\multicolumn{1}{|l|}{ $\pm$Infinito  $\cdot$ $\pm$0 } & \multicolumn{1}{l|}{  NaN} &   \\ \cline{1-2}
\end{tabular}
\end{table}

\paragraph{Almacenar -6,12510}\mbox{}\\
\begin{enumerate}    
  \item El bit 31 tomará el valor del signo del número. 
  \item Pasar a binario la mantisa decimal.
  \begin{itemize}
    \item $6=110_2$
    \item $0,125=0,001_2$
    \item $6,125=110,001_2$
  \end{itemize}
  \item Normalizar
  \begin{itemize}
    \item Desplazamiento a la derecha  -> Exponente negativo
    \item Desplazamiento a la izquierda -> Exponente positivo
    \item 1,10001 , exponente = 2
    \item 2 expresado en Exceso 127 es 129 $0000001_2$
  \end{itemize}
  \item Mantisa representada con bit implícito: 1,10001 -> 10001
   \item El número final es 1 10000001 100010000000000000000002 (Se agregan a la derecha los “0” necesarios para completar los 23 bits de la mantisa)
\end{enumerate}

\paragraph{Recuperar}\mbox{}\\\\
\begin{enumerate}    
  \item Se realizan los pasos en orden inverso
\end{enumerate}

\paragraph{Binario de Punto Flotante (IBM mainframe)}
\begin{itemize}
\item Base: 16
\item Representa: enteros con coma decimal positivos y negativos
\item Precision : imple 4 bytes, doble, 8 bytes, extendida 16 bytes.
\item estructura: S nnnnnnn dddddd
\begin{itemize}
    \item s = signo 1- 0+
    \item n = digitos de la caracteristica, en total son 7 bits se usan para calcular el exponente 
    \item C = E + 4016 donde E corresponde al exponente
    \item d =  mantisa normalizada -> 0,dddddd x $10^e_{16}$
  \end{itemize}
\end{itemize}

\paragraph{Almacenar - 321,54 10 -> Binario de punto flotante precisión simple}\mbox{}\\\\
\begin{enumerate}    
  \item $321,54_{10} = 141,8A3_{16}$
  \item $0,1418A3 x 10^3 _{16}$
  \item $C = E + 40_{16} = 3 + 40_{16} = 43_{16} en base 2 sería 100011_{2}$
  \item Agregamos el bit de signo: $1100011_{2}$ que en base 16 es $C3_{16}$
  \item Resultado final : $C31418A3_{16}$
\end{enumerate}

\paragraph{Ancho de paso }\mbox{}\\\\
distancia entre un flotante y su siguiente numero representable en el formato

\paragraph{Absorcion}\mbox{}\\\\
se da en las operaciones de suma y resta entre flotantes

\begin{itemize}    
  \item $A = 0,15A4 x 105_{16}$
  \item $B = 0,54F x 10^-2_{16}$
\end{itemize}

para poder operar entre flotantes debemos igualar los exponentes llevandolos al mayor de todos
\begin{itemize}    
  \item $A+B = 0.15A400 * 10^5 + 0,00...00 * 10^5 = 0,15A4000 * 10^5 _{16}$
\end{itemize}
lo minimo que se puede sumar es el ancho de paso del numero de mayor exponente

\section{Intel x86}
\section{Formato y Configuracion}

\subsection{Definición}
\begin{itemize}
\item Formato: Representación computacional
\item Configuración: Representación en una determinada base de un número en un formato
\end{itemize}

\subsection{Expansión y truncamiento}
\subsection{Definición}
\begin{itemize}
\item Expandir formato: Significa completar la representación computacional sin alterar el numero representado
en el mismo. 
\item Truncar formato: Descartar digitos de su representación sin alterar el número representado en el mismo.
\end{itemize}

\subsection{Formatos}

%Binario punto fijo sin signo
\subsubsection{Binario punto fijo sin signo}
\begin{itemize}
\item Base: 2
\item Representa: números enteros positivos
\item Máximo: $2^{n-1}_{10}$
\item Mínimo: 0
\end{itemize}

\paragraph{Almacenar}
\begin{enumerate}
\item Pasar el numero a base 2
\item completar con ceros a izquierda la capacidad del formato
\end{enumerate}
\paragraph{Recuperar}
\begin{enumerate}
\item Pasamos el numero de base 2 a la base deseada 
\end{enumerate}

%Binario de punto fijo con signo
\subsubsection{Binario de punto fijo con signo}
\begin{itemize}
\item Base: 2
\item Representa: Enteros positivos y negativos
\item Primer bit: reservado para el signo
\item Máximo: $2^{n-1}-1$
\item Mínimo: $-2^{n-1}$
\end{itemize}

\paragraph{Almacenar}
\begin{enumerate}
\item Pasar el numero a base 2
\item completar con ceros a izquierda la capacidad del formato
\item Si es un numero negativo hacerle el complemento a 2
\end{enumerate}
\paragraph{Recuperar}
\begin{enumerate}
\item Si el bit de signo es cero, se pasa de base 2 a base 10
\item Si el bit es 1, el numero es negativo, por lo que debemos complementarlo
\item Quitamos los ceros a la izquierda
\item Numero de base 2 a base 10
\item Colocamos el signo
\end{enumerate}

\paragraph{Expansión}
\begin{enumerate}
\item Se completa con el bit de signo a la izquierda
\end{enumerate}
\paragraph{Truncamiento}
\begin{enumerate}
\item Se extraen bits a la izquierda siempre y cuando no se esté alterando el bit de signo del número.
\end{enumerate}

%Empaquetado
\subsubsection{Empaquetado}
\begin{itemize}
\item Base: 16
\item Representa: Enteros positivos y negativos
\item Máximo: $10^{2n-1} -1$
\item Mínimo: $-10^{2n-1} +1$
\end{itemize}

\paragraph{Almacenar}
\begin{enumerate}
\item numero a base 10
\item Colocal cada digito decimal en un nibble dejando el ultimo nibble ya que en el mismo se almacena el signo.
\item Colocar en el ultimo nible el signo, CAFE = + , DF = -
\item Se rellena con 0 hasta alcanzar la cantidad de bytes usados
\end{enumerate}
\paragraph{Recuperar}
\begin{enumerate}
\item los pasos en orden inverso
\end{enumerate}

%Zoneado
\subsubsection{Zoneado}
\begin{itemize}
\item Base: 16
\item Representa: Enteros positivos y negativos
\item Máximo: $10^{n} -1$
\item Mínimo: $-10^{n} +1$
\end{itemize}

\paragraph{Almacenar}
\begin{enumerate}
\item numero a base 10
\item colocar cada uno de los digitos decimales en un nibble derecho
\item completar todos los nibbles de izquierda con F salvo el ultimo que se completa con el signo siguiendo las mismas reglas que para empaquetados.  CAFE = + , DF = -
\item se rellena con F0 hasta alcanzar la cantidad de bytes usados
\end{enumerate}
\paragraph{Recuperar}
\begin{enumerate}
\item los pasos en orden inverso
\end{enumerate}

\subsubsection{Binario de punto Flotante}
es la manera que tiene una arquitectura de representar a los numeros reales.
su notación cientifica se expresa de la siguiente manera $M x B^E$
M: Mantissa B: base E: Exponente

Un numero binario está normalizado si el digito de la izquierda del punto es igual a 1.

\paragraph{IEEE754}
\begin{itemize}
\item Precision Simple: signo: 1 bit exponente: 8 bits fraccion: 23 bits Exponente: Exceso 127
\item Precision Doble: signo: 1 bit exponente 11bits fraccion: 52 bits Exponente: Exceso 1023
\end{itemize}

\paragraph{Ancho de paso}\mbox{}\\\\
Marca cual es la distancia entre un flotante y su siguiente numero representable en el formato

\paragraph{Overflow}\mbox{}\\\\
el exponente excede el limite superior, tanto para mantisas positivas como para negatvias, dando lugar a +inf, - inf
\paragraph{Underflow}\mbox{}\\\\
el exponente excede el minimo valor permitido y caga en el intervalo (-inf, -0) y (+0,+inf)

\paragraph{Desnormalizados - Subnormales}\mbox{}\\\\
tienen como exponente al cero, y el bit implicito a la izquierda del punto binario, es ahora un cero implicito. la diferencia entre los desnormalizados y los normalizados es que, estos ultimos no permiten al cero como exponente. Los normalizados tienen 24 bits significativos, mientras que los normalizados poseen 23.

\begin{table}[h]
\begin{tabular}{lllll}
\cline{1-4}
\multicolumn{1}{|l|}{Normalizado} & \multicolumn{1}{l|}{+/-} & \multicolumn{1}{l|}{0<exp<max} &  \multicolumn{1}{l|}{cualquier patron de bits} &  \\ \cline{1-4}
\multicolumn{1}{|l|}{Desnormalizado} & \multicolumn{1}{l|}{+/-} & \multicolumn{1}{l|}{0} &  \multicolumn{1}{l|}{cualquier patron de bits != 0} &  \\ \cline{1-4}
\multicolumn{1}{|l|}{Cero} & \multicolumn{1}{l|}{+/-} & \multicolumn{1}{l|}{0} &  \multicolumn{1}{l|}{0} &  \\ \cline{1-4}
\multicolumn{1}{|l|}{Infinito} & \multicolumn{1}{l|}{+/-} & \multicolumn{1}{l|}{11...11} &  \multicolumn{1}{l|}{0} &  \\ \cline{1-4}
\multicolumn{1}{|l|}{NAN} & \multicolumn{1}{l|}{+/-} & \multicolumn{1}{l|}{11...11} &  \multicolumn{1}{l|}{Cualquier patron de bits != 0} &  \\ \cline{1-4}
\end{tabular}
\end{table}

\begin{itemize}
\item Infinito dividido Infinito = NAN
\item Cero + = 0 00000000 00000000000000000000000
\item Cero - = 1 00000000 00000000000000000000000
\item Infinito + = 0 11111111 00000000000000000000000
\item Infinito - = 1 11111111 00000000000000000000000
\item No normalizados/Subnormales (no se asume que haya que añadir un 1 al significando para obtener su valor).
\end{itemize}

\paragraph{Valores no numericos}\mbox{}\\\\
NaN (Not a number). 2 tipos, QNaN (quiet nan) y SNaN (signalling nan) Qnan = indeterminado Snan = operacion no valida

\begin{itemize}
\item Infinito dividido Infinito = NAN
\item qnan = 0 11111111 10000100000000000000000
\item snan = 1 11111111 00100010001001010101010
\end{itemize}

\begin{table}[h]
\begin{tabular}{lllll}
\cline{1-2}
\multicolumn{1}{|l|}{Operacion} & \multicolumn{1}{l|}{Resultado} &   \\ \cline{1-2}
\multicolumn{1}{|l|}{n $\pm$ infinito } & \multicolumn{1}{l|}{0} &   \\ \cline{1-2}
\multicolumn{1}{|l|}{$\pm$ infinito $\cdot$ $\pm$infinito } & \multicolumn{1}{l|}{$\pm$infinito} &   \\ \cline{1-2}
\multicolumn{1}{|l|}{ $\pm$n $\div$ 0   } & \multicolumn{1}{l|}{ $\pm$infinito } &   \\ \cline{1-2}
\multicolumn{1}{|l|}{ Infinito + Infinito   } & \multicolumn{1}{l|}{ Infinito } &   \\ \cline{1-2}
\multicolumn{1}{|l|}{Cualquier operación contra un NaN   } & \multicolumn{1}{l|}{ NaN } &   \\ \cline{1-2}
\multicolumn{1}{|l|}{ $\pm$0 $\div$ $\pm$0   } & \multicolumn{1}{l|}{ NaN } &   \\ \cline{1-2}
\multicolumn{1}{|l|}{ Infinito - Infinito   } & \multicolumn{1}{l|}{  NaN} &   \\ \cline{1-2}
\multicolumn{1}{|l|}{ $\pm$Infinito  $\div$ $\pm$Infinito  } & \multicolumn{1}{l|}{  NaN} &   \\ \cline{1-2}
\multicolumn{1}{|l|}{ $\pm$Infinito  $\cdot$ $\pm$0 } & \multicolumn{1}{l|}{  NaN} &   \\ \cline{1-2}
\end{tabular}
\end{table}

\paragraph{Almacenar -6,12510}\mbox{}\\
\begin{enumerate}    
  \item El bit 31 tomará el valor del signo del número. 
  \item Pasar a binario la mantisa decimal.
  \begin{itemize}
    \item $6=110_2$
    \item $0,125=0,001_2$
    \item $6,125=110,001_2$
  \end{itemize}
  \item Normalizar
  \begin{itemize}
    \item Desplazamiento a la derecha  -> Exponente negativo
    \item Desplazamiento a la izquierda -> Exponente positivo
    \item 1,10001 , exponente = 2
    \item 2 expresado en Exceso 127 es 129 $0000001_2$
  \end{itemize}
  \item Mantisa representada con bit implícito: 1,10001 -> 10001
   \item El número final es 1 10000001 100010000000000000000002 (Se agregan a la derecha los “0” necesarios para completar los 23 bits de la mantisa)
\end{enumerate}

\paragraph{Recuperar}\mbox{}\\\\
\begin{enumerate}    
  \item Se realizan los pasos en orden inverso
\end{enumerate}

\paragraph{Binario de Punto Flotante (IBM mainframe)}
\begin{itemize}
\item Base: 16
\item Representa: enteros con coma decimal positivos y negativos
\item Precision : imple 4 bytes, doble, 8 bytes, extendida 16 bytes.
\item estructura: S nnnnnnn dddddd
\begin{itemize}
    \item s = signo 1- 0+
    \item n = digitos de la caracteristica, en total son 7 bits se usan para calcular el exponente 
    \item C = E + 4016 donde E corresponde al exponente
    \item d =  mantisa normalizada -> 0,dddddd x $10^e_{16}$
  \end{itemize}
\end{itemize}

\paragraph{Almacenar - 321,54 10 -> Binario de punto flotante precisión simple}\mbox{}\\\\
\begin{enumerate}    
  \item $321,54_{10} = 141,8A3_{16}$
  \item $0,1418A3 x 10^3 _{16}$
  \item $C = E + 40_{16} = 3 + 40_{16} = 43_{16} en base 2 sería 100011_{2}$
  \item Agregamos el bit de signo: $1100011_{2}$ que en base 16 es $C3_{16}$
  \item Resultado final : $C31418A3_{16}$
\end{enumerate}

\paragraph{Ancho de paso }\mbox{}\\\\
distancia entre un flotante y su siguiente numero representable en el formato

\paragraph{Absorcion}\mbox{}\\\\
se da en las operaciones de suma y resta entre flotantes

\begin{itemize}    
  \item $A = 0,15A4 x 105_{16}$
  \item $B = 0,54F x 10^-2_{16}$
\end{itemize}

para poder operar entre flotantes debemos igualar los exponentes llevandolos al mayor de todos
\begin{itemize}    
  \item $A+B = 0.15A400 * 10^5 + 0,00...00 * 10^5 = 0,15A4000 * 10^5 _{16}$
\end{itemize}
lo minimo que se puede sumar es el ancho de paso del numero de mayor exponente

\section{Intel x86}
\section{Formato y Configuracion}

\subsection{Definición}
\begin{itemize}
\item Formato: Representación computacional
\item Configuración: Representación en una determinada base de un número en un formato
\end{itemize}

\subsection{Expansión y truncamiento}
\subsection{Definición}
\begin{itemize}
\item Expandir formato: Significa completar la representación computacional sin alterar el numero representado
en el mismo. 
\item Truncar formato: Descartar digitos de su representación sin alterar el número representado en el mismo.
\end{itemize}

\subsection{Formatos}

%Binario punto fijo sin signo
\subsubsection{Binario punto fijo sin signo}
\begin{itemize}
\item Base: 2
\item Representa: números enteros positivos
\item Máximo: $2^{n-1}_{10}$
\item Mínimo: 0
\end{itemize}

\paragraph{Almacenar}
\begin{enumerate}
\item Pasar el numero a base 2
\item completar con ceros a izquierda la capacidad del formato
\end{enumerate}
\paragraph{Recuperar}
\begin{enumerate}
\item Pasamos el numero de base 2 a la base deseada 
\end{enumerate}

%Binario de punto fijo con signo
\subsubsection{Binario de punto fijo con signo}
\begin{itemize}
\item Base: 2
\item Representa: Enteros positivos y negativos
\item Primer bit: reservado para el signo
\item Máximo: $2^{n-1}-1$
\item Mínimo: $-2^{n-1}$
\end{itemize}

\paragraph{Almacenar}
\begin{enumerate}
\item Pasar el numero a base 2
\item completar con ceros a izquierda la capacidad del formato
\item Si es un numero negativo hacerle el complemento a 2
\end{enumerate}
\paragraph{Recuperar}
\begin{enumerate}
\item Si el bit de signo es cero, se pasa de base 2 a base 10
\item Si el bit es 1, el numero es negativo, por lo que debemos complementarlo
\item Quitamos los ceros a la izquierda
\item Numero de base 2 a base 10
\item Colocamos el signo
\end{enumerate}

\paragraph{Expansión}
\begin{enumerate}
\item Se completa con el bit de signo a la izquierda
\end{enumerate}
\paragraph{Truncamiento}
\begin{enumerate}
\item Se extraen bits a la izquierda siempre y cuando no se esté alterando el bit de signo del número.
\end{enumerate}

%Empaquetado
\subsubsection{Empaquetado}
\begin{itemize}
\item Base: 16
\item Representa: Enteros positivos y negativos
\item Máximo: $10^{2n-1} -1$
\item Mínimo: $-10^{2n-1} +1$
\end{itemize}

\paragraph{Almacenar}
\begin{enumerate}
\item numero a base 10
\item Colocal cada digito decimal en un nibble dejando el ultimo nibble ya que en el mismo se almacena el signo.
\item Colocar en el ultimo nible el signo, CAFE = + , DF = -
\item Se rellena con 0 hasta alcanzar la cantidad de bytes usados
\end{enumerate}
\paragraph{Recuperar}
\begin{enumerate}
\item los pasos en orden inverso
\end{enumerate}

%Zoneado
\subsubsection{Zoneado}
\begin{itemize}
\item Base: 16
\item Representa: Enteros positivos y negativos
\item Máximo: $10^{n} -1$
\item Mínimo: $-10^{n} +1$
\end{itemize}

\paragraph{Almacenar}
\begin{enumerate}
\item numero a base 10
\item colocar cada uno de los digitos decimales en un nibble derecho
\item completar todos los nibbles de izquierda con F salvo el ultimo que se completa con el signo siguiendo las mismas reglas que para empaquetados.  CAFE = + , DF = -
\item se rellena con F0 hasta alcanzar la cantidad de bytes usados
\end{enumerate}
\paragraph{Recuperar}
\begin{enumerate}
\item los pasos en orden inverso
\end{enumerate}

\subsubsection{Binario de punto Flotante}
es la manera que tiene una arquitectura de representar a los numeros reales.
su notación cientifica se expresa de la siguiente manera $M x B^E$
M: Mantissa B: base E: Exponente

Un numero binario está normalizado si el digito de la izquierda del punto es igual a 1.

\paragraph{IEEE754}
\begin{itemize}
\item Precision Simple: signo: 1 bit exponente: 8 bits fraccion: 23 bits Exponente: Exceso 127
\item Precision Doble: signo: 1 bit exponente 11bits fraccion: 52 bits Exponente: Exceso 1023
\end{itemize}

\paragraph{Ancho de paso}\mbox{}\\\\
Marca cual es la distancia entre un flotante y su siguiente numero representable en el formato

\paragraph{Overflow}\mbox{}\\\\
el exponente excede el limite superior, tanto para mantisas positivas como para negatvias, dando lugar a +inf, - inf
\paragraph{Underflow}\mbox{}\\\\
el exponente excede el minimo valor permitido y caga en el intervalo (-inf, -0) y (+0,+inf)

\paragraph{Desnormalizados - Subnormales}\mbox{}\\\\
tienen como exponente al cero, y el bit implicito a la izquierda del punto binario, es ahora un cero implicito. la diferencia entre los desnormalizados y los normalizados es que, estos ultimos no permiten al cero como exponente. Los normalizados tienen 24 bits significativos, mientras que los normalizados poseen 23.

\begin{table}[h]
\begin{tabular}{lllll}
\cline{1-4}
\multicolumn{1}{|l|}{Normalizado} & \multicolumn{1}{l|}{+/-} & \multicolumn{1}{l|}{0<exp<max} &  \multicolumn{1}{l|}{cualquier patron de bits} &  \\ \cline{1-4}
\multicolumn{1}{|l|}{Desnormalizado} & \multicolumn{1}{l|}{+/-} & \multicolumn{1}{l|}{0} &  \multicolumn{1}{l|}{cualquier patron de bits != 0} &  \\ \cline{1-4}
\multicolumn{1}{|l|}{Cero} & \multicolumn{1}{l|}{+/-} & \multicolumn{1}{l|}{0} &  \multicolumn{1}{l|}{0} &  \\ \cline{1-4}
\multicolumn{1}{|l|}{Infinito} & \multicolumn{1}{l|}{+/-} & \multicolumn{1}{l|}{11...11} &  \multicolumn{1}{l|}{0} &  \\ \cline{1-4}
\multicolumn{1}{|l|}{NAN} & \multicolumn{1}{l|}{+/-} & \multicolumn{1}{l|}{11...11} &  \multicolumn{1}{l|}{Cualquier patron de bits != 0} &  \\ \cline{1-4}
\end{tabular}
\end{table}

\begin{itemize}
\item Infinito dividido Infinito = NAN
\item Cero + = 0 00000000 00000000000000000000000
\item Cero - = 1 00000000 00000000000000000000000
\item Infinito + = 0 11111111 00000000000000000000000
\item Infinito - = 1 11111111 00000000000000000000000
\item No normalizados/Subnormales (no se asume que haya que añadir un 1 al significando para obtener su valor).
\end{itemize}

\paragraph{Valores no numericos}\mbox{}\\\\
NaN (Not a number). 2 tipos, QNaN (quiet nan) y SNaN (signalling nan) Qnan = indeterminado Snan = operacion no valida

\begin{itemize}
\item Infinito dividido Infinito = NAN
\item qnan = 0 11111111 10000100000000000000000
\item snan = 1 11111111 00100010001001010101010
\end{itemize}

\begin{table}[h]
\begin{tabular}{lllll}
\cline{1-2}
\multicolumn{1}{|l|}{Operacion} & \multicolumn{1}{l|}{Resultado} &   \\ \cline{1-2}
\multicolumn{1}{|l|}{n $\pm$ infinito } & \multicolumn{1}{l|}{0} &   \\ \cline{1-2}
\multicolumn{1}{|l|}{$\pm$ infinito $\cdot$ $\pm$infinito } & \multicolumn{1}{l|}{$\pm$infinito} &   \\ \cline{1-2}
\multicolumn{1}{|l|}{ $\pm$n $\div$ 0   } & \multicolumn{1}{l|}{ $\pm$infinito } &   \\ \cline{1-2}
\multicolumn{1}{|l|}{ Infinito + Infinito   } & \multicolumn{1}{l|}{ Infinito } &   \\ \cline{1-2}
\multicolumn{1}{|l|}{Cualquier operación contra un NaN   } & \multicolumn{1}{l|}{ NaN } &   \\ \cline{1-2}
\multicolumn{1}{|l|}{ $\pm$0 $\div$ $\pm$0   } & \multicolumn{1}{l|}{ NaN } &   \\ \cline{1-2}
\multicolumn{1}{|l|}{ Infinito - Infinito   } & \multicolumn{1}{l|}{  NaN} &   \\ \cline{1-2}
\multicolumn{1}{|l|}{ $\pm$Infinito  $\div$ $\pm$Infinito  } & \multicolumn{1}{l|}{  NaN} &   \\ \cline{1-2}
\multicolumn{1}{|l|}{ $\pm$Infinito  $\cdot$ $\pm$0 } & \multicolumn{1}{l|}{  NaN} &   \\ \cline{1-2}
\end{tabular}
\end{table}

\paragraph{Almacenar -6,12510}\mbox{}\\
\begin{enumerate}    
  \item El bit 31 tomará el valor del signo del número. 
  \item Pasar a binario la mantisa decimal.
  \begin{itemize}
    \item $6=110_2$
    \item $0,125=0,001_2$
    \item $6,125=110,001_2$
  \end{itemize}
  \item Normalizar
  \begin{itemize}
    \item Desplazamiento a la derecha  -> Exponente negativo
    \item Desplazamiento a la izquierda -> Exponente positivo
    \item 1,10001 , exponente = 2
    \item 2 expresado en Exceso 127 es 129 $0000001_2$
  \end{itemize}
  \item Mantisa representada con bit implícito: 1,10001 -> 10001
   \item El número final es 1 10000001 100010000000000000000002 (Se agregan a la derecha los “0” necesarios para completar los 23 bits de la mantisa)
\end{enumerate}

\paragraph{Recuperar}\mbox{}\\\\
\begin{enumerate}    
  \item Se realizan los pasos en orden inverso
\end{enumerate}

\paragraph{Binario de Punto Flotante (IBM mainframe)}
\begin{itemize}
\item Base: 16
\item Representa: enteros con coma decimal positivos y negativos
\item Precision : imple 4 bytes, doble, 8 bytes, extendida 16 bytes.
\item estructura: S nnnnnnn dddddd
\begin{itemize}
    \item s = signo 1- 0+
    \item n = digitos de la caracteristica, en total son 7 bits se usan para calcular el exponente 
    \item C = E + 4016 donde E corresponde al exponente
    \item d =  mantisa normalizada -> 0,dddddd x $10^e_{16}$
  \end{itemize}
\end{itemize}

\paragraph{Almacenar - 321,54 10 -> Binario de punto flotante precisión simple}\mbox{}\\\\
\begin{enumerate}    
  \item $321,54_{10} = 141,8A3_{16}$
  \item $0,1418A3 x 10^3 _{16}$
  \item $C = E + 40_{16} = 3 + 40_{16} = 43_{16} en base 2 sería 100011_{2}$
  \item Agregamos el bit de signo: $1100011_{2}$ que en base 16 es $C3_{16}$
  \item Resultado final : $C31418A3_{16}$
\end{enumerate}

\paragraph{Ancho de paso }\mbox{}\\\\
distancia entre un flotante y su siguiente numero representable en el formato

\paragraph{Absorcion}\mbox{}\\\\
se da en las operaciones de suma y resta entre flotantes

\begin{itemize}    
  \item $A = 0,15A4 x 105_{16}$
  \item $B = 0,54F x 10^-2_{16}$
\end{itemize}

para poder operar entre flotantes debemos igualar los exponentes llevandolos al mayor de todos
\begin{itemize}    
  \item $A+B = 0.15A400 * 10^5 + 0,00...00 * 10^5 = 0,15A4000 * 10^5 _{16}$
\end{itemize}
lo minimo que se puede sumar es el ancho de paso del numero de mayor exponente

\section{Intel x86}
\section{Formato y Configuracion}

\subsection{Definición}
\begin{itemize}
\item Formato: Representación computacional
\item Configuración: Representación en una determinada base de un número en un formato
\end{itemize}

\subsection{Expansión y truncamiento}
\subsection{Definición}
\begin{itemize}
\item Expandir formato: Significa completar la representación computacional sin alterar el numero representado
en el mismo. 
\item Truncar formato: Descartar digitos de su representación sin alterar el número representado en el mismo.
\end{itemize}

\subsection{Formatos}

%Binario punto fijo sin signo
\subsubsection{Binario punto fijo sin signo}
\begin{itemize}
\item Base: 2
\item Representa: números enteros positivos
\item Máximo: $2^{n-1}_{10}$
\item Mínimo: 0
\end{itemize}

\paragraph{Almacenar}
\begin{enumerate}
\item Pasar el numero a base 2
\item completar con ceros a izquierda la capacidad del formato
\end{enumerate}
\paragraph{Recuperar}
\begin{enumerate}
\item Pasamos el numero de base 2 a la base deseada 
\end{enumerate}

%Binario de punto fijo con signo
\subsubsection{Binario de punto fijo con signo}
\begin{itemize}
\item Base: 2
\item Representa: Enteros positivos y negativos
\item Primer bit: reservado para el signo
\item Máximo: $2^{n-1}-1$
\item Mínimo: $-2^{n-1}$
\end{itemize}

\paragraph{Almacenar}
\begin{enumerate}
\item Pasar el numero a base 2
\item completar con ceros a izquierda la capacidad del formato
\item Si es un numero negativo hacerle el complemento a 2
\end{enumerate}
\paragraph{Recuperar}
\begin{enumerate}
\item Si el bit de signo es cero, se pasa de base 2 a base 10
\item Si el bit es 1, el numero es negativo, por lo que debemos complementarlo
\item Quitamos los ceros a la izquierda
\item Numero de base 2 a base 10
\item Colocamos el signo
\end{enumerate}

\paragraph{Expansión}
\begin{enumerate}
\item Se completa con el bit de signo a la izquierda
\end{enumerate}
\paragraph{Truncamiento}
\begin{enumerate}
\item Se extraen bits a la izquierda siempre y cuando no se esté alterando el bit de signo del número.
\end{enumerate}

%Empaquetado
\subsubsection{Empaquetado}
\begin{itemize}
\item Base: 16
\item Representa: Enteros positivos y negativos
\item Máximo: $10^{2n-1} -1$
\item Mínimo: $-10^{2n-1} +1$
\end{itemize}

\paragraph{Almacenar}
\begin{enumerate}
\item numero a base 10
\item Colocal cada digito decimal en un nibble dejando el ultimo nibble ya que en el mismo se almacena el signo.
\item Colocar en el ultimo nible el signo, CAFE = + , DF = -
\item Se rellena con 0 hasta alcanzar la cantidad de bytes usados
\end{enumerate}
\paragraph{Recuperar}
\begin{enumerate}
\item los pasos en orden inverso
\end{enumerate}

%Zoneado
\subsubsection{Zoneado}
\begin{itemize}
\item Base: 16
\item Representa: Enteros positivos y negativos
\item Máximo: $10^{n} -1$
\item Mínimo: $-10^{n} +1$
\end{itemize}

\paragraph{Almacenar}
\begin{enumerate}
\item numero a base 10
\item colocar cada uno de los digitos decimales en un nibble derecho
\item completar todos los nibbles de izquierda con F salvo el ultimo que se completa con el signo siguiendo las mismas reglas que para empaquetados.  CAFE = + , DF = -
\item se rellena con F0 hasta alcanzar la cantidad de bytes usados
\end{enumerate}
\paragraph{Recuperar}
\begin{enumerate}
\item los pasos en orden inverso
\end{enumerate}

\subsubsection{Binario de punto Flotante}
es la manera que tiene una arquitectura de representar a los numeros reales.
su notación cientifica se expresa de la siguiente manera $M x B^E$
M: Mantissa B: base E: Exponente

Un numero binario está normalizado si el digito de la izquierda del punto es igual a 1.

\paragraph{IEEE754}
\begin{itemize}
\item Precision Simple: signo: 1 bit exponente: 8 bits fraccion: 23 bits Exponente: Exceso 127
\item Precision Doble: signo: 1 bit exponente 11bits fraccion: 52 bits Exponente: Exceso 1023
\end{itemize}

\paragraph{Ancho de paso}\mbox{}\\\\
Marca cual es la distancia entre un flotante y su siguiente numero representable en el formato

\paragraph{Overflow}\mbox{}\\\\
el exponente excede el limite superior, tanto para mantisas positivas como para negatvias, dando lugar a +inf, - inf
\paragraph{Underflow}\mbox{}\\\\
el exponente excede el minimo valor permitido y caga en el intervalo (-inf, -0) y (+0,+inf)

\paragraph{Desnormalizados - Subnormales}\mbox{}\\\\
tienen como exponente al cero, y el bit implicito a la izquierda del punto binario, es ahora un cero implicito. la diferencia entre los desnormalizados y los normalizados es que, estos ultimos no permiten al cero como exponente. Los normalizados tienen 24 bits significativos, mientras que los normalizados poseen 23.

\begin{table}[h]
\begin{tabular}{lllll}
\cline{1-4}
\multicolumn{1}{|l|}{Normalizado} & \multicolumn{1}{l|}{+/-} & \multicolumn{1}{l|}{0<exp<max} &  \multicolumn{1}{l|}{cualquier patron de bits} &  \\ \cline{1-4}
\multicolumn{1}{|l|}{Desnormalizado} & \multicolumn{1}{l|}{+/-} & \multicolumn{1}{l|}{0} &  \multicolumn{1}{l|}{cualquier patron de bits != 0} &  \\ \cline{1-4}
\multicolumn{1}{|l|}{Cero} & \multicolumn{1}{l|}{+/-} & \multicolumn{1}{l|}{0} &  \multicolumn{1}{l|}{0} &  \\ \cline{1-4}
\multicolumn{1}{|l|}{Infinito} & \multicolumn{1}{l|}{+/-} & \multicolumn{1}{l|}{11...11} &  \multicolumn{1}{l|}{0} &  \\ \cline{1-4}
\multicolumn{1}{|l|}{NAN} & \multicolumn{1}{l|}{+/-} & \multicolumn{1}{l|}{11...11} &  \multicolumn{1}{l|}{Cualquier patron de bits != 0} &  \\ \cline{1-4}
\end{tabular}
\end{table}

\begin{itemize}
\item Infinito dividido Infinito = NAN
\item Cero + = 0 00000000 00000000000000000000000
\item Cero - = 1 00000000 00000000000000000000000
\item Infinito + = 0 11111111 00000000000000000000000
\item Infinito - = 1 11111111 00000000000000000000000
\item No normalizados/Subnormales (no se asume que haya que añadir un 1 al significando para obtener su valor).
\end{itemize}

\paragraph{Valores no numericos}\mbox{}\\\\
NaN (Not a number). 2 tipos, QNaN (quiet nan) y SNaN (signalling nan) Qnan = indeterminado Snan = operacion no valida

\begin{itemize}
\item Infinito dividido Infinito = NAN
\item qnan = 0 11111111 10000100000000000000000
\item snan = 1 11111111 00100010001001010101010
\end{itemize}

\begin{table}[h]
\begin{tabular}{lllll}
\cline{1-2}
\multicolumn{1}{|l|}{Operacion} & \multicolumn{1}{l|}{Resultado} &   \\ \cline{1-2}
\multicolumn{1}{|l|}{n $\pm$ infinito } & \multicolumn{1}{l|}{0} &   \\ \cline{1-2}
\multicolumn{1}{|l|}{$\pm$ infinito $\cdot$ $\pm$infinito } & \multicolumn{1}{l|}{$\pm$infinito} &   \\ \cline{1-2}
\multicolumn{1}{|l|}{ $\pm$n $\div$ 0   } & \multicolumn{1}{l|}{ $\pm$infinito } &   \\ \cline{1-2}
\multicolumn{1}{|l|}{ Infinito + Infinito   } & \multicolumn{1}{l|}{ Infinito } &   \\ \cline{1-2}
\multicolumn{1}{|l|}{Cualquier operación contra un NaN   } & \multicolumn{1}{l|}{ NaN } &   \\ \cline{1-2}
\multicolumn{1}{|l|}{ $\pm$0 $\div$ $\pm$0   } & \multicolumn{1}{l|}{ NaN } &   \\ \cline{1-2}
\multicolumn{1}{|l|}{ Infinito - Infinito   } & \multicolumn{1}{l|}{  NaN} &   \\ \cline{1-2}
\multicolumn{1}{|l|}{ $\pm$Infinito  $\div$ $\pm$Infinito  } & \multicolumn{1}{l|}{  NaN} &   \\ \cline{1-2}
\multicolumn{1}{|l|}{ $\pm$Infinito  $\cdot$ $\pm$0 } & \multicolumn{1}{l|}{  NaN} &   \\ \cline{1-2}
\end{tabular}
\end{table}

\paragraph{Almacenar -6,12510}\mbox{}\\
\begin{enumerate}    
  \item El bit 31 tomará el valor del signo del número. 
  \item Pasar a binario la mantisa decimal.
  \begin{itemize}
    \item $6=110_2$
    \item $0,125=0,001_2$
    \item $6,125=110,001_2$
  \end{itemize}
  \item Normalizar
  \begin{itemize}
    \item Desplazamiento a la derecha  -> Exponente negativo
    \item Desplazamiento a la izquierda -> Exponente positivo
    \item 1,10001 , exponente = 2
    \item 2 expresado en Exceso 127 es 129 $0000001_2$
  \end{itemize}
  \item Mantisa representada con bit implícito: 1,10001 -> 10001
   \item El número final es 1 10000001 100010000000000000000002 (Se agregan a la derecha los “0” necesarios para completar los 23 bits de la mantisa)
\end{enumerate}

\paragraph{Recuperar}\mbox{}\\\\
\begin{enumerate}    
  \item Se realizan los pasos en orden inverso
\end{enumerate}

\paragraph{Binario de Punto Flotante (IBM mainframe)}
\begin{itemize}
\item Base: 16
\item Representa: enteros con coma decimal positivos y negativos
\item Precision : imple 4 bytes, doble, 8 bytes, extendida 16 bytes.
\item estructura: S nnnnnnn dddddd
\begin{itemize}
    \item s = signo 1- 0+
    \item n = digitos de la caracteristica, en total son 7 bits se usan para calcular el exponente 
    \item C = E + 4016 donde E corresponde al exponente
    \item d =  mantisa normalizada -> 0,dddddd x $10^e_{16}$
  \end{itemize}
\end{itemize}

\paragraph{Almacenar - 321,54 10 -> Binario de punto flotante precisión simple}\mbox{}\\\\
\begin{enumerate}    
  \item $321,54_{10} = 141,8A3_{16}$
  \item $0,1418A3 x 10^3 _{16}$
  \item $C = E + 40_{16} = 3 + 40_{16} = 43_{16} en base 2 sería 100011_{2}$
  \item Agregamos el bit de signo: $1100011_{2}$ que en base 16 es $C3_{16}$
  \item Resultado final : $C31418A3_{16}$
\end{enumerate}

\paragraph{Ancho de paso }\mbox{}\\\\
distancia entre un flotante y su siguiente numero representable en el formato

\paragraph{Absorcion}\mbox{}\\\\
se da en las operaciones de suma y resta entre flotantes

\begin{itemize}    
  \item $A = 0,15A4 x 105_{16}$
  \item $B = 0,54F x 10^-2_{16}$
\end{itemize}

para poder operar entre flotantes debemos igualar los exponentes llevandolos al mayor de todos
\begin{itemize}    
  \item $A+B = 0.15A400 * 10^5 + 0,00...00 * 10^5 = 0,15A4000 * 10^5 _{16}$
\end{itemize}
lo minimo que se puede sumar es el ancho de paso del numero de mayor exponente

\subsection{Interrupciones}
\subsubsection{Definicion}
Mecanismos por los cuales otros modulos (E/S y memoria) interrumpen el normal procesamiento del CPU
\subsubsection{¿Para que existen?}
Para mejorar la eficiencia de procesamiento de un computador
\subsubsection{Clases de interrupciones}
\begin{itemize}
	\item programa
	\item  reloj
	\item e/s
	\item fallas de hardware
\end{itemize}

\subsubsection{Ciclo de instruccion}
\begin{itemize}
	\item Fetch instruction
	\item  Decode instruction
	\item Fetch operand
	\item Execute instruction
	\item Store result
	\item ----------------> interrupt breakpoint
	\item process interrupt
\end{itemize}

\subsubsection{Transferencia de control al S.O. (Handler)}
/*4pdf*/
	
\subsubsection{Procesamiento de interrupciones }
/*5pdf*/

/*6pdf ejemplo*/
	
\subsubsection{Multiples interrupciones}


\paragraph{Deshabilitar interrupciones (secuencia)}\mbox{}\\\\%%
/*7pdf*/
\paragraph{Priorizar interrupciones (anidadas)}\mbox{}\\\\%%
/*8pdf*/
	
\paragraph{Múltiples interrupciones - ejemplo}\mbox{}\\\\%%
Tres dispositivos de E/S
\begin{itemize}
	\item Línea de comunicación (Prioridad 1)
	\item  Disco (Prioridad 2)
	\item Impresora (Prioridad 3)
\end{itemize}
Eventos
\begin{itemize}
	\item T=10 Interrupción de Impresora
	\item T=15 Interrupción de línea de comunicación
	\item T=20 Interrupción de disco
\end{itemize}

/*10pdf*/

\subsection{MODULO DE E/S }

\subsubsection{Que hace}
Conecta a los periféricos con la CPU y la memoria a través del bus del sistema o switch central y permite la comunicación entre ellos
\subsubsection{Para que sirve}
Oculta detalles de timing, formatos y electro mecánica de los dispositivos periféricos
\subsubsection{Por que existe?}
\begin{itemize}
\item Amplia variedad de periféricos con distintos métodos de operación
\item La tasa de transferencia de los periféricos es generalmente mucho más lenta que la de la memoria y procesador
\item Los periféricos usan distintos formatos de datos y tamaños de palabra
\end{itemize}

/*4pdf*/ 230 stalling

\subsubsection{Interface interna - bus del sistema}
\begin{itemize}
\item Datos
\item Direcciones
\item Control
\end{itemize}

\subsubsection{Interface externa - perifericos}
\begin{itemize}
\item Datos
\item Estado
\item Control
\end{itemize}


\subsubsection{Funciones}
\paragraph{Control and Timing}\mbox{}\\\\%%
ontrola flujo de tráfico entre CPU/Memoria y periféricos
\paragraph{Comunicación con el procesador}\mbox{}\\\\%%
Decodificación de comandos: The I/O module accepts commands from the processor, typically sent as signals on the control bus. For example, an I/O module for a disk drive might accept the following commands: READ SECTOR, WRITE SECTOR, SEEK track number, and SCAN record ID. The latter two commands each include a parameter that is sent on the data bus
\begin{itemize}
	\item Datos: Data are exchanged between the processor and the I/O module over the data bus.
	\item Información de estado: Because peripherals are so slow, it is important to know the	status of the I/O module. For example, if an I/O module is asked to send data to the processor (read), it may not be ready to do so because it is still working on the previous I/O command. This fact can be reported with a status signal.	Common status signals are BUSY and READY. There may also be signals to report various error conditions.
	\item Reconocimiento de direcciones: Just as each word of memory has an address, so does each I/O device. Thus, an I/O module must recognize one unique address for each peripheral it controls.
\end{itemize}
\paragraph{Comunicación con el dispositivo}\mbox{}\\\\%%
\begin{itemize}
\item Comandos
\item Información de estado
\item Datos
\end{itemize}
\paragraph{Buffering de datos}\mbox{}\\\\%%
\paragraph{Detección de errores}\mbox{}\\\\%%

\subsubsection{The control of the transfer of data from an external device to the processor might involve the following sequence of steps}
\begin{enumerate}
\item The processor interrogates the I/O module to check the status of the attached device.
\item The I/O module returns the device status.
\item If the device is operational and ready to transmit, the processor requests the transfer of data, by means of a command to the IO module.
\item  The I/O module obtains a unit of data (e.g., 8 or 16 bits) from the external device.
\item The data are transferred from the I/O module to the processor.
\end{enumerate}

/*Diagrama I/O pag 234 Will, 7pdf */

\subsubsection{Tecnicas para operaciones de E/S}
\begin{enumerate}
\item  E/S programada
\item  E/S manejada por interrupciones
\item  Acceso directo a memoria (DMA)
\end{enumerate}

/*pag 237*/

When large volumes of data are to be moved, a more efficient technique is
required: direct memory access (DMA).
DMA involves an additional module on the system bus. The DMA module
(Figure 7.12) is capable of mimicking the processor and, indeed, of taking over control
of the system from the processor. It needs to do this to transfer data to and from
memory over the system bus. For this purpose, the DMA module must use the bus
only when the processor does not need it, or it must force the processor to suspend
operation temporarily. The latter technique is more common and is referred to as
cycle stealing, because the DMA module in effect steals a bus cycle.
When the processor wishes to read or write a block of data, it issues a command
to the DMA module, by sending to the DMA module the following information:
\begin{enumerate}
\item  Whether a read or write is requested, using the read or write control line
between the processor and the DMA module.
\item  The address of the I/O device involved, communicated on the data lines.
\item  The starting location in memory to read from or write to, communicated on
the data lines and stored by the DMA module in its address register.
\item  The number of words to be read or written, again communicated via the data
lines and stored in the data count register.
\end{enumerate}
/*pag 249*/ 
The processor then continues with other work. It has delegated this I/O operation
to the DMA module. The DMA module transfers the entire block of data,
one word at a time, directly to or from memory, without going through the processor.
When the transfer is complete, the DMA module sends an interrupt signal to
the processor. Thus, the processor is involved only at the beginning and end of the
transfer

/*  250 Figure 7.13 DMA and Interrupt Breakpoints during an Instruction Cycle*/

/* 251 Figure 7.14 Alternative DMA Configurations*/

The DMA mechanism can be configured in a variety of ways. Some possibilities
are shown in Figure 7.14. In the first example, all modules share the same system
bus. The DMA module, acting as a surrogate processor, uses programmed I/O to
exchange data between memory and an I/O module through the DMA module. This
configuration, while it may be inexpensive, is clearly inefficient. As with processor-
controlled programmed I/O, each transfer of a word consumes two bus cycles
The number of required bus cycles can be cut substantially by integrating the
DMA and I/O functions. As Figure 7.14b indicates, this means that there is a path
between the DMA module and one or more I/O modules that does not include
the system bus. The DMA logic may actually be a part of an I/O module, or it may
be a separate module that controls one or more I/O modules. This concept can
be taken one step further by connecting I/O modules to the DMA module using
an I/O bus (Figure 7.14c). This reduces the number of I/O interfaces in the DMA
module to one and provides for an easily expandable configuration. In both of
these cases (Figures 7.14b and c), the system bus that the DMA module shares with
the processor and memory is used by the DMA module only to exchange data with
memory. The exchange of data between the DMA and I/O modules takes place off
the system bus.

\subsubsection{I /O Channels and Processors}
The Evolution of the I/O Function
As computer systems have evolved, there has been a pattern of increasing complexity
and sophistication of individual components. Nowhere is this more evident than
in the I/O function. We have already seen part of that evolution. The evolutionary
steps can be summarized as follows:
\begin{enumerate}
\item The CPU directly controls a peripheral device. This is seen in simple
microprocessor-controlled devices.
\item A controller or I/O module is added. The CPU uses programmed I/O without
interrupts. With this step, the CPU becomes somewhat divorced from the specific
details of external device interfaces.
\item  The same configuration as in step 2 is used, but now interrupts are employed.
The CPU need not spend time waiting for an I/O operation to be performed,
thus increasing efficiency.
\item The I/O module is given direct access to memory via DMA. It can now move
a block of data to or from memory without involving the CPU, except at the
beginning and end of the transfer.
\item  The I/O module is enhanced to become a processor in its own right, with a
specialized instruction set tailored for I/O. The CPU directs the I/O processor
to execute an I/O program in memory. The I/O processor fetches and executes
these instructions without CPU intervention. This allows the CPU to specify a
sequence of I/O activities and to be interrupted only when the entire sequence
has been performed.
\item The I/O module has a local memory of its own and is, in fact, a computer in its
own right. With this architecture, a large set of I/O devices can be controlled,
with minimal CPU involvement. A common use for such an architecture has
been to control communication with interactive terminals. The I/O processor
takes care of most of the tasks involved in controlling the terminals.
\end{enumerate}

\subsubsection{Characteristics of I/O Channels}
	The I/O channel represents an extension of the DMA concept. An I/O channel
	has the ability to execute I/O instructions, which gives it complete control over
	I/O operations. In a computer system with such devices, the CPU does not execute
	I/O instructions. Such instructions are stored in main memory to be executed by a
	special-purpose processor in the I/O channel itself. Thus, the CPU initiates an I/O
	transfer by instructing the I/O channel to execute a program in memory. The program
	will specify the device or devices, the area or areas of memory for storage,
	priority, and actions to be taken for certain error conditions. The I/O channel follows
	these instructions and controls the data transfer

	Two types of I/O channels are common, as illustrated in Figure 7.18. A
	selector channel controls multiple high-speed devices and, at any one time, is
	dedicated to the transfer of data with one of those devices. Thus, the I/O channel
	selects one device and effects the data transfer. Each device, or a small set of
	devices, is handled by a controller, or I/O module, that is much like the I/O modules
	we have been discussing. Thus, the I/O channel serves in place of the CPU
	in controlling these I/O controllers. A multiplexor channel can handle I/O with
	multiple devices at the same time. For low-speed devices, a byte multiplexor
	accepts or transmits characters as fast as possible to multiple devices. For example,
	the resultant character stream from three devices with different rates and individual
	streams A1A2A3A4 c, B1B2B3B4 c, and C1C2C3C4 c might be A1B1C1A2C2A3B2C3A4, and so on.
	

	/*263*/

\subsection{ADMINISTRACION DE MEMORIA }

\subsubsection{Sistema Operativo}
Software que administra los recursos del computador, provee servicios y controla la ejecución de otros programas
\subsubsection{Algunos servicios que provee}
\begin{itemize}
\item Schedule de procesos
\item Administración de memoria
\item Monitor
\item Parte residente del Sistema Operativo
\end{itemize}
/*284*/
In a uniprogramming system, main memory is divided into two parts: one part for
the OS (resident monitor) and one part for the program currently being executed.
In a multiprogramming system, the “user” part of memory is subdivided to accommodate
multiple processes. The task of subdivision is carried out dynamically by the
OS and is known as memory management.

\subsubsection{Administración de memoria simple}
\subsubsection{Sistema con uniprogramación}

Se divide la memoria en dos partes
Monitor del S.O.
Programa en ejecución en ese momento
\paragraph{Ventajas:}\mbox{}\\\\%%

Simplicidad

\paragraph{DESVentajas:}\mbox{}\\\\%%
\begin{itemize}
\item Desperdicio de memoria
\item Desaprovechamiento de los recursos del computador
\end{itemize}
Administración de memoria simple 
/*281*/

simple batch systems Early processors were very expensive, and therefore it
was important to maximize processor utilization. The wasted time due to scheduling
and setup time was unacceptable.
To improve utilization, simple batch operating systems were developed. With
such a system, also called a monitor, the user no longer has direct access to the processor.
Rather, the user submits the job on cards or tape to a computer operator,
who batches the jobs together sequentially and places the entire batch on an input
device, for use by the monitor.
To understand how this scheme works, let us look at it from two points of
view: that of the monitor and that of the processor. From the point of view of the
monitor, the monitor controls the sequence of events. For this to be so, much of the
monitor must always be in main memory and available for execution (Figure 8.3).
That portion is referred to as the resident monitor. The rest of the monitor consists
of utilities and common functions that are loaded as subroutines to the user program
at the beginning of any job that requires them. The monitor reads in jobs one
at a time from the input device (typically a card reader or magnetic tape drive). As it
is read in, the current job is placed in the user program area, and control is passed to
this job. When the job is completed, it returns control to the monitor, which immediately
reads in the next job. The results of each job are printed out for delivery to
the user.
Now consider this sequence from the point of view of the processor. At a certain
point in time, the processor is executing instructions from the portion of main memory
containing the monitor. These instructions cause the next job to be read in to
another portion of main memory. Once a job has been read in, the processor will
encounter in the monitor a branch instruction that instructs the processor to continue
execution at the start of the user program. The processor will then execute
the instruction in the user’s program until it encounters an ending or error condition.
Either event causes the processor to fetch its next instruction from the monitor
program. Thus the phrase “control is passed to a job” simply means that the processor
is now fetching and executing instructions in a user program, and “control is
returned to the monitor” means that the processor is now fetching and executing
instructions from the monitor program.
It should be clear that the monitor handles the scheduling problem. A batch of
jobs is queued up, and jobs are executed as rapidly as possible, with no intervening
idle time.
How about the job setup time? The monitor handles this as well. With each
job, instructions are included in a job control language (JCL). This is a special type
of programming language used to provide instructions to the monitor.


\subsubsection{Multiprogramming}
\begin{itemize}
\item Varios procesos de usuario en ejecución a lavez
\item Se divide la memoria de usuario entre los procesos en ejecución
\item Se comparte el tiempo de procesador entre los procesos en ejecución ( timeslice
\item Condiciones de finalización:
	\begin{itemize}
	\item Termina el trabajo
	\item Se detecta un error y se cancela
	\item Requiere una operación de E/S (suspensión)
	\item Termina el timeslice (suspención
	\end{itemize}
\end{itemize}

The simplest scheme for partitioning available memory is to use fixed-
sizepartitions, as shown in Figure 8.13. Note that, although the partitions are of fixed size, they need not be of equal size. When a process is brought into memory, it is placed in the
smallest available partition that will hold it.
Even with the use of unequal fixed-size partitions, there will be wasted memory.
In most cases, a process will not require exactly as much memory as provided by the partition.
A more efficient approach is to use variable-size partitions. When a process is
brought into memory, it is allocated exactly as much memory as it requires and no more.
As this example shows, this method starts out well, but eventually it leads to a
situation in which there are a lot of small holes in memory. As time goes on, memory
becomes more and more fragmented, and memory utilization declines. One
technique for overcoming this problem is compaction: From time to time, the OS
shifts the processes in memory to place all the free memory together in one block.
This is a time-consuming procedure, wasteful of processor time.

\subsubsection{Memory management: partitioning}
\begin{itemize}
\item Sistema con multiprogramación
\item La memoria de usuario se divide en particiones detamaño fijo:
	\begin{itemize}
	\item Iguales
	\item Distintas
	\end{itemize}
\end{itemize}
Ventajas:
\begin{itemize}
\item Permite compartir la memoria entre varios procesos
\end{itemize}
Desventajas:
\begin{itemize}
\item Desperdicio de memoria
\item Fragmentación interna (dentro de una partición)
\item Fragmentación externa (particiones no usadas)
\end{itemize}

\subsubsection{Memory management: swapping}
\begin{itemize}
\item Sistema con multiprogramación
\item Swapping
\item La memoria de usuario se divide en particiones de tamaño variable
\item Compactación para eliminar la fragmentación
\item Se usa un recurso de hardware (registro de reasignación) para la realocación
\item Realocación dinámica en tiempo de ejecución
\end{itemize}
Ventajas:
\begin{itemize}
\item Permite compartir la memoria entre varios procesos
\item Elimina el desperdicio por fragmentación interna.
\item Con la compactación se elimina además la fragmentación externa
\end{itemize}
Desventajas:
\begin{itemize}
\item La tarea de compactación es costosa
\end{itemize}

\subsubsection{Memory management: paging}
Both unequal fixed-size and variable-size
partitions are inefficient in the use of memory.
Suppose, however, that memory is partitioned into equal fixed-size
chunks that are relatively small, and that each process is also divided into small fixed-
size chunks of some size. Then the chunks of a program, known as pages, could be assigned to
available chunks of memory, known as frames, or page frames. At most, then, the
wasted space in memory for that process is a fraction of the last page.

\paragraph{Administración de memoria paginada simple}\mbox{}\\\\%%
\begin{itemize}
\item Sistema con multiprogramación
\item Se divide el address space del proceso en partes iguales (páginas)
\item Se divide la memoria principal en partes iguales ( frames)
\item Hay una tabla de páginas por proceso
\item Hay una lista de frames disponibles
\item Se cargan a memoria las páginas del proceso en los frames disponibles (no es necesario que sean contiguos)
\item Las direcciones lógicas se ven como número de página y un offset
\item Se traducen las direcciones lógicas en físicas con soporte del hardware
\item La paginación es transparente para el programador
\end{itemize}
Ventajas:
\begin{itemize}
\item Permite compartir la memoria entre varios procesos
\item Minimiza la fragmentación interna (solo existe dentro de la última página de cada proceso)
\item Elimina la fragmentación externa
\end{itemize}
Desventajas
\begin{itemize}
\item Se requiere subir todas las páginas del proceso a memoria
\item Se requieren estructuras de datos adicionales para mantener información de páginas y frames
\end{itemize}

\paragraph{Administración de memoria paginada simple}\mbox{}\\\\%%
\begin{itemize}
\item Sistema con multiprogramación
\item Solo se cargan las páginas necesarias para la ejecución de un proceso
\item Cuando se quiere acceder a una posición de memoria de una página no cargada se produce un page fault
\item El page fault dispara una interrupción por hardware atendida por el sistema operativo
\item Se levanta la página solicitada desde memoria secundaria (memoria virtual)
\item Algoritmos para reemplazo de páginas
\item Thrashing : el CPU pasa más tiempo reemplazando páginas que ejecutando instrucciones
\end{itemize}
Ventajas
\begin{itemize}
\item No es necesario cargar todas las páginas de un proceso a la vez
\item Maximiza el uso de la memoria al permitir cargar más procesos a la vez
\item Un proceso puede ocupar más memoria de la efectivamente instalada en el computador
\end{itemize}
Desventajas
\begin{itemize}
\item Mayor complejidad por la necesidad de implementar el reemplazo de páginas
\end{itemize}

\subsubsection{Administración de memoria por segmentación}
\begin{itemize}
\item Sistemas con multiprogramación
\item Generalmente visible al programador
\item La memoria del programa se ve como un conjunto de segmentos (múltiples espacios de direcciones)
\item Los segmentos son de tamaño variable y dinámico
\item El sistema operativo administra una tabla de segmentos por proceso
\item Permite separar datos e instrucciones
\item Permite dar privilegios y protección de memoria como por ej . lectura, escritura, ejecución. ( segmentation faults como mecanismos de excepción de hardware para accesos indebidos)
\item Las referencias a memoria se forman con un número de segmento y un offset dentro de él. Con ayuda de hardware (MMU \item Memory Management Unit ) se hacen las traducciones de las direcciones lógicas a físicas
\item Se pueden usar para implementar memoria virtual (solo se suben a
memoria física algunos segmentos por proceso)
\end{itemize}

Ventajas:
\begin{itemize}
\item Simplifica el manejo de estructuras de datos con crecimiento
\item Permite compartir información entre procesos dentro de un segmento
\item Permite aplicar protección/privilegios sobre un segmento fácilmente
\end{itemize}
Desventajas:
\begin{itemize}
\item Fragmentación externa en la memoria principal por no poder alojar un segmento
\item Hardware más complejo que memoria paginada para la traducción de direcciones
\end{itemize}

\subsubsection{Address Spaces}
The x86 includes hardware for both segmentation and paging. Both mechanisms can
be disabled, allowing the user to choose from four distinct views of memory:

\begin{itemize}
\item Unsegmented unpaged memory: In this case, the virtual address is the same
as the physical address. This is useful, for example, in low- complexity, high- performance controller applications.
\item Unsegmented paged memory: Here memory is viewed as a paged linear
address space. Protection and management of memory is done via paging.
This is favored by some operating systems (e.g., Berkeley UNIX).
\item Segmented unpaged memory: Here memory is viewed as a collection of
logical address spaces. The advantage of this view over a paged approach is
that it affords protection down to the level of a single byte, if necessary. Furthermore,
unlike paging, it guarantees that the translation table needed (the segment table) is on-chip
when the segment is in memory. Hence, segmented unpaged memory results in predictable access times.
\item Segmented paged memory: Segmentation is used to define logical memory
partitions subject to access control, and paging is used to manage the allocation
of memory within the partitions. Operating systems such as UNIX System favor this view.
\end{itemize}
%%%%%%%%%%%%%%%%%%%%


\section{Intel x86}
\section{Formato y Configuracion}

\subsection{Definición}
\begin{itemize}
\item Formato: Representación computacional
\item Configuración: Representación en una determinada base de un número en un formato
\end{itemize}

\subsection{Expansión y truncamiento}
\subsection{Definición}
\begin{itemize}
\item Expandir formato: Significa completar la representación computacional sin alterar el numero representado
en el mismo. 
\item Truncar formato: Descartar digitos de su representación sin alterar el número representado en el mismo.
\end{itemize}

\subsection{Formatos}

%Binario punto fijo sin signo
\subsubsection{Binario punto fijo sin signo}
\begin{itemize}
\item Base: 2
\item Representa: números enteros positivos
\item Máximo: $2^{n-1}_{10}$
\item Mínimo: 0
\end{itemize}

\paragraph{Almacenar}
\begin{enumerate}
\item Pasar el numero a base 2
\item completar con ceros a izquierda la capacidad del formato
\end{enumerate}
\paragraph{Recuperar}
\begin{enumerate}
\item Pasamos el numero de base 2 a la base deseada 
\end{enumerate}

%Binario de punto fijo con signo
\subsubsection{Binario de punto fijo con signo}
\begin{itemize}
\item Base: 2
\item Representa: Enteros positivos y negativos
\item Primer bit: reservado para el signo
\item Máximo: $2^{n-1}-1$
\item Mínimo: $-2^{n-1}$
\end{itemize}

\paragraph{Almacenar}
\begin{enumerate}
\item Pasar el numero a base 2
\item completar con ceros a izquierda la capacidad del formato
\item Si es un numero negativo hacerle el complemento a 2
\end{enumerate}
\paragraph{Recuperar}
\begin{enumerate}
\item Si el bit de signo es cero, se pasa de base 2 a base 10
\item Si el bit es 1, el numero es negativo, por lo que debemos complementarlo
\item Quitamos los ceros a la izquierda
\item Numero de base 2 a base 10
\item Colocamos el signo
\end{enumerate}

\paragraph{Expansión}
\begin{enumerate}
\item Se completa con el bit de signo a la izquierda
\end{enumerate}
\paragraph{Truncamiento}
\begin{enumerate}
\item Se extraen bits a la izquierda siempre y cuando no se esté alterando el bit de signo del número.
\end{enumerate}

%Empaquetado
\subsubsection{Empaquetado}
\begin{itemize}
\item Base: 16
\item Representa: Enteros positivos y negativos
\item Máximo: $10^{2n-1} -1$
\item Mínimo: $-10^{2n-1} +1$
\end{itemize}

\paragraph{Almacenar}
\begin{enumerate}
\item numero a base 10
\item Colocal cada digito decimal en un nibble dejando el ultimo nibble ya que en el mismo se almacena el signo.
\item Colocar en el ultimo nible el signo, CAFE = + , DF = -
\item Se rellena con 0 hasta alcanzar la cantidad de bytes usados
\end{enumerate}
\paragraph{Recuperar}
\begin{enumerate}
\item los pasos en orden inverso
\end{enumerate}

%Zoneado
\subsubsection{Zoneado}
\begin{itemize}
\item Base: 16
\item Representa: Enteros positivos y negativos
\item Máximo: $10^{n} -1$
\item Mínimo: $-10^{n} +1$
\end{itemize}

\paragraph{Almacenar}
\begin{enumerate}
\item numero a base 10
\item colocar cada uno de los digitos decimales en un nibble derecho
\item completar todos los nibbles de izquierda con F salvo el ultimo que se completa con el signo siguiendo las mismas reglas que para empaquetados.  CAFE = + , DF = -
\item se rellena con F0 hasta alcanzar la cantidad de bytes usados
\end{enumerate}
\paragraph{Recuperar}
\begin{enumerate}
\item los pasos en orden inverso
\end{enumerate}

\subsubsection{Binario de punto Flotante}
es la manera que tiene una arquitectura de representar a los numeros reales.
su notación cientifica se expresa de la siguiente manera $M x B^E$
M: Mantissa B: base E: Exponente

Un numero binario está normalizado si el digito de la izquierda del punto es igual a 1.

\paragraph{IEEE754}
\begin{itemize}
\item Precision Simple: signo: 1 bit exponente: 8 bits fraccion: 23 bits Exponente: Exceso 127
\item Precision Doble: signo: 1 bit exponente 11bits fraccion: 52 bits Exponente: Exceso 1023
\end{itemize}

\paragraph{Ancho de paso}\mbox{}\\\\
Marca cual es la distancia entre un flotante y su siguiente numero representable en el formato

\paragraph{Overflow}\mbox{}\\\\
el exponente excede el limite superior, tanto para mantisas positivas como para negatvias, dando lugar a +inf, - inf
\paragraph{Underflow}\mbox{}\\\\
el exponente excede el minimo valor permitido y caga en el intervalo (-inf, -0) y (+0,+inf)

\paragraph{Desnormalizados - Subnormales}\mbox{}\\\\
tienen como exponente al cero, y el bit implicito a la izquierda del punto binario, es ahora un cero implicito. la diferencia entre los desnormalizados y los normalizados es que, estos ultimos no permiten al cero como exponente. Los normalizados tienen 24 bits significativos, mientras que los normalizados poseen 23.

\begin{table}[h]
\begin{tabular}{lllll}
\cline{1-4}
\multicolumn{1}{|l|}{Normalizado} & \multicolumn{1}{l|}{+/-} & \multicolumn{1}{l|}{0<exp<max} &  \multicolumn{1}{l|}{cualquier patron de bits} &  \\ \cline{1-4}
\multicolumn{1}{|l|}{Desnormalizado} & \multicolumn{1}{l|}{+/-} & \multicolumn{1}{l|}{0} &  \multicolumn{1}{l|}{cualquier patron de bits != 0} &  \\ \cline{1-4}
\multicolumn{1}{|l|}{Cero} & \multicolumn{1}{l|}{+/-} & \multicolumn{1}{l|}{0} &  \multicolumn{1}{l|}{0} &  \\ \cline{1-4}
\multicolumn{1}{|l|}{Infinito} & \multicolumn{1}{l|}{+/-} & \multicolumn{1}{l|}{11...11} &  \multicolumn{1}{l|}{0} &  \\ \cline{1-4}
\multicolumn{1}{|l|}{NAN} & \multicolumn{1}{l|}{+/-} & \multicolumn{1}{l|}{11...11} &  \multicolumn{1}{l|}{Cualquier patron de bits != 0} &  \\ \cline{1-4}
\end{tabular}
\end{table}

\begin{itemize}
\item Infinito dividido Infinito = NAN
\item Cero + = 0 00000000 00000000000000000000000
\item Cero - = 1 00000000 00000000000000000000000
\item Infinito + = 0 11111111 00000000000000000000000
\item Infinito - = 1 11111111 00000000000000000000000
\item No normalizados/Subnormales (no se asume que haya que añadir un 1 al significando para obtener su valor).
\end{itemize}

\paragraph{Valores no numericos}\mbox{}\\\\
NaN (Not a number). 2 tipos, QNaN (quiet nan) y SNaN (signalling nan) Qnan = indeterminado Snan = operacion no valida

\begin{itemize}
\item Infinito dividido Infinito = NAN
\item qnan = 0 11111111 10000100000000000000000
\item snan = 1 11111111 00100010001001010101010
\end{itemize}

\begin{table}[h]
\begin{tabular}{lllll}
\cline{1-2}
\multicolumn{1}{|l|}{Operacion} & \multicolumn{1}{l|}{Resultado} &   \\ \cline{1-2}
\multicolumn{1}{|l|}{n $\pm$ infinito } & \multicolumn{1}{l|}{0} &   \\ \cline{1-2}
\multicolumn{1}{|l|}{$\pm$ infinito $\cdot$ $\pm$infinito } & \multicolumn{1}{l|}{$\pm$infinito} &   \\ \cline{1-2}
\multicolumn{1}{|l|}{ $\pm$n $\div$ 0   } & \multicolumn{1}{l|}{ $\pm$infinito } &   \\ \cline{1-2}
\multicolumn{1}{|l|}{ Infinito + Infinito   } & \multicolumn{1}{l|}{ Infinito } &   \\ \cline{1-2}
\multicolumn{1}{|l|}{Cualquier operación contra un NaN   } & \multicolumn{1}{l|}{ NaN } &   \\ \cline{1-2}
\multicolumn{1}{|l|}{ $\pm$0 $\div$ $\pm$0   } & \multicolumn{1}{l|}{ NaN } &   \\ \cline{1-2}
\multicolumn{1}{|l|}{ Infinito - Infinito   } & \multicolumn{1}{l|}{  NaN} &   \\ \cline{1-2}
\multicolumn{1}{|l|}{ $\pm$Infinito  $\div$ $\pm$Infinito  } & \multicolumn{1}{l|}{  NaN} &   \\ \cline{1-2}
\multicolumn{1}{|l|}{ $\pm$Infinito  $\cdot$ $\pm$0 } & \multicolumn{1}{l|}{  NaN} &   \\ \cline{1-2}
\end{tabular}
\end{table}

\paragraph{Almacenar -6,12510}\mbox{}\\
\begin{enumerate}    
  \item El bit 31 tomará el valor del signo del número. 
  \item Pasar a binario la mantisa decimal.
  \begin{itemize}
    \item $6=110_2$
    \item $0,125=0,001_2$
    \item $6,125=110,001_2$
  \end{itemize}
  \item Normalizar
  \begin{itemize}
    \item Desplazamiento a la derecha  -> Exponente negativo
    \item Desplazamiento a la izquierda -> Exponente positivo
    \item 1,10001 , exponente = 2
    \item 2 expresado en Exceso 127 es 129 $0000001_2$
  \end{itemize}
  \item Mantisa representada con bit implícito: 1,10001 -> 10001
   \item El número final es 1 10000001 100010000000000000000002 (Se agregan a la derecha los “0” necesarios para completar los 23 bits de la mantisa)
\end{enumerate}

\paragraph{Recuperar}\mbox{}\\\\
\begin{enumerate}    
  \item Se realizan los pasos en orden inverso
\end{enumerate}

\paragraph{Binario de Punto Flotante (IBM mainframe)}
\begin{itemize}
\item Base: 16
\item Representa: enteros con coma decimal positivos y negativos
\item Precision : imple 4 bytes, doble, 8 bytes, extendida 16 bytes.
\item estructura: S nnnnnnn dddddd
\begin{itemize}
    \item s = signo 1- 0+
    \item n = digitos de la caracteristica, en total son 7 bits se usan para calcular el exponente 
    \item C = E + 4016 donde E corresponde al exponente
    \item d =  mantisa normalizada -> 0,dddddd x $10^e_{16}$
  \end{itemize}
\end{itemize}

\paragraph{Almacenar - 321,54 10 -> Binario de punto flotante precisión simple}\mbox{}\\\\
\begin{enumerate}    
  \item $321,54_{10} = 141,8A3_{16}$
  \item $0,1418A3 x 10^3 _{16}$
  \item $C = E + 40_{16} = 3 + 40_{16} = 43_{16} en base 2 sería 100011_{2}$
  \item Agregamos el bit de signo: $1100011_{2}$ que en base 16 es $C3_{16}$
  \item Resultado final : $C31418A3_{16}$
\end{enumerate}

\paragraph{Ancho de paso }\mbox{}\\\\
distancia entre un flotante y su siguiente numero representable en el formato

\paragraph{Absorcion}\mbox{}\\\\
se da en las operaciones de suma y resta entre flotantes

\begin{itemize}    
  \item $A = 0,15A4 x 105_{16}$
  \item $B = 0,54F x 10^-2_{16}$
\end{itemize}

para poder operar entre flotantes debemos igualar los exponentes llevandolos al mayor de todos
\begin{itemize}    
  \item $A+B = 0.15A400 * 10^5 + 0,00...00 * 10^5 = 0,15A4000 * 10^5 _{16}$
\end{itemize}
lo minimo que se puede sumar es el ancho de paso del numero de mayor exponente

\subsection{Interrupciones}
\subsubsection{Definicion}
Mecanismos por los cuales otros modulos (E/S y memoria) interrumpen el normal procesamiento del CPU
\subsubsection{¿Para que existen?}
Para mejorar la eficiencia de procesamiento de un computador
\subsubsection{Clases de interrupciones}
\begin{itemize}
	\item programa
	\item  reloj
	\item e/s
	\item fallas de hardware
\end{itemize}

\subsubsection{Ciclo de instruccion}
\begin{itemize}
	\item Fetch instruction
	\item  Decode instruction
	\item Fetch operand
	\item Execute instruction
	\item Store result
	\item ----------------> interrupt breakpoint
	\item process interrupt
\end{itemize}

\subsubsection{Transferencia de control al S.O. (Handler)}
/*4pdf*/
	
\subsubsection{Procesamiento de interrupciones }
/*5pdf*/

/*6pdf ejemplo*/
	
\subsubsection{Multiples interrupciones}


\paragraph{Deshabilitar interrupciones (secuencia)}\mbox{}\\\\%%
/*7pdf*/
\paragraph{Priorizar interrupciones (anidadas)}\mbox{}\\\\%%
/*8pdf*/
	
\paragraph{Múltiples interrupciones - ejemplo}\mbox{}\\\\%%
Tres dispositivos de E/S
\begin{itemize}
	\item Línea de comunicación (Prioridad 1)
	\item  Disco (Prioridad 2)
	\item Impresora (Prioridad 3)
\end{itemize}
Eventos
\begin{itemize}
	\item T=10 Interrupción de Impresora
	\item T=15 Interrupción de línea de comunicación
	\item T=20 Interrupción de disco
\end{itemize}

/*10pdf*/

\subsection{MODULO DE E/S }

\subsubsection{Que hace}
Conecta a los periféricos con la CPU y la memoria a través del bus del sistema o switch central y permite la comunicación entre ellos
\subsubsection{Para que sirve}
Oculta detalles de timing, formatos y electro mecánica de los dispositivos periféricos
\subsubsection{Por que existe?}
\begin{itemize}
\item Amplia variedad de periféricos con distintos métodos de operación
\item La tasa de transferencia de los periféricos es generalmente mucho más lenta que la de la memoria y procesador
\item Los periféricos usan distintos formatos de datos y tamaños de palabra
\end{itemize}

/*4pdf*/ 230 stalling

\subsubsection{Interface interna - bus del sistema}
\begin{itemize}
\item Datos
\item Direcciones
\item Control
\end{itemize}

\subsubsection{Interface externa - perifericos}
\begin{itemize}
\item Datos
\item Estado
\item Control
\end{itemize}


\subsubsection{Funciones}
\paragraph{Control and Timing}\mbox{}\\\\%%
ontrola flujo de tráfico entre CPU/Memoria y periféricos
\paragraph{Comunicación con el procesador}\mbox{}\\\\%%
Decodificación de comandos: The I/O module accepts commands from the processor, typically sent as signals on the control bus. For example, an I/O module for a disk drive might accept the following commands: READ SECTOR, WRITE SECTOR, SEEK track number, and SCAN record ID. The latter two commands each include a parameter that is sent on the data bus
\begin{itemize}
	\item Datos: Data are exchanged between the processor and the I/O module over the data bus.
	\item Información de estado: Because peripherals are so slow, it is important to know the	status of the I/O module. For example, if an I/O module is asked to send data to the processor (read), it may not be ready to do so because it is still working on the previous I/O command. This fact can be reported with a status signal.	Common status signals are BUSY and READY. There may also be signals to report various error conditions.
	\item Reconocimiento de direcciones: Just as each word of memory has an address, so does each I/O device. Thus, an I/O module must recognize one unique address for each peripheral it controls.
\end{itemize}
\paragraph{Comunicación con el dispositivo}\mbox{}\\\\%%
\begin{itemize}
\item Comandos
\item Información de estado
\item Datos
\end{itemize}
\paragraph{Buffering de datos}\mbox{}\\\\%%
\paragraph{Detección de errores}\mbox{}\\\\%%

\subsubsection{The control of the transfer of data from an external device to the processor might involve the following sequence of steps}
\begin{enumerate}
\item The processor interrogates the I/O module to check the status of the attached device.
\item The I/O module returns the device status.
\item If the device is operational and ready to transmit, the processor requests the transfer of data, by means of a command to the IO module.
\item  The I/O module obtains a unit of data (e.g., 8 or 16 bits) from the external device.
\item The data are transferred from the I/O module to the processor.
\end{enumerate}

/*Diagrama I/O pag 234 Will, 7pdf */

\subsubsection{Tecnicas para operaciones de E/S}
\begin{enumerate}
\item  E/S programada
\item  E/S manejada por interrupciones
\item  Acceso directo a memoria (DMA)
\end{enumerate}

/*pag 237*/

When large volumes of data are to be moved, a more efficient technique is
required: direct memory access (DMA).
DMA involves an additional module on the system bus. The DMA module
(Figure 7.12) is capable of mimicking the processor and, indeed, of taking over control
of the system from the processor. It needs to do this to transfer data to and from
memory over the system bus. For this purpose, the DMA module must use the bus
only when the processor does not need it, or it must force the processor to suspend
operation temporarily. The latter technique is more common and is referred to as
cycle stealing, because the DMA module in effect steals a bus cycle.
When the processor wishes to read or write a block of data, it issues a command
to the DMA module, by sending to the DMA module the following information:
\begin{enumerate}
\item  Whether a read or write is requested, using the read or write control line
between the processor and the DMA module.
\item  The address of the I/O device involved, communicated on the data lines.
\item  The starting location in memory to read from or write to, communicated on
the data lines and stored by the DMA module in its address register.
\item  The number of words to be read or written, again communicated via the data
lines and stored in the data count register.
\end{enumerate}
/*pag 249*/ 
The processor then continues with other work. It has delegated this I/O operation
to the DMA module. The DMA module transfers the entire block of data,
one word at a time, directly to or from memory, without going through the processor.
When the transfer is complete, the DMA module sends an interrupt signal to
the processor. Thus, the processor is involved only at the beginning and end of the
transfer

/*  250 Figure 7.13 DMA and Interrupt Breakpoints during an Instruction Cycle*/

/* 251 Figure 7.14 Alternative DMA Configurations*/

The DMA mechanism can be configured in a variety of ways. Some possibilities
are shown in Figure 7.14. In the first example, all modules share the same system
bus. The DMA module, acting as a surrogate processor, uses programmed I/O to
exchange data between memory and an I/O module through the DMA module. This
configuration, while it may be inexpensive, is clearly inefficient. As with processor-
controlled programmed I/O, each transfer of a word consumes two bus cycles
The number of required bus cycles can be cut substantially by integrating the
DMA and I/O functions. As Figure 7.14b indicates, this means that there is a path
between the DMA module and one or more I/O modules that does not include
the system bus. The DMA logic may actually be a part of an I/O module, or it may
be a separate module that controls one or more I/O modules. This concept can
be taken one step further by connecting I/O modules to the DMA module using
an I/O bus (Figure 7.14c). This reduces the number of I/O interfaces in the DMA
module to one and provides for an easily expandable configuration. In both of
these cases (Figures 7.14b and c), the system bus that the DMA module shares with
the processor and memory is used by the DMA module only to exchange data with
memory. The exchange of data between the DMA and I/O modules takes place off
the system bus.

\subsubsection{I /O Channels and Processors}
The Evolution of the I/O Function
As computer systems have evolved, there has been a pattern of increasing complexity
and sophistication of individual components. Nowhere is this more evident than
in the I/O function. We have already seen part of that evolution. The evolutionary
steps can be summarized as follows:
\begin{enumerate}
\item The CPU directly controls a peripheral device. This is seen in simple
microprocessor-controlled devices.
\item A controller or I/O module is added. The CPU uses programmed I/O without
interrupts. With this step, the CPU becomes somewhat divorced from the specific
details of external device interfaces.
\item  The same configuration as in step 2 is used, but now interrupts are employed.
The CPU need not spend time waiting for an I/O operation to be performed,
thus increasing efficiency.
\item The I/O module is given direct access to memory via DMA. It can now move
a block of data to or from memory without involving the CPU, except at the
beginning and end of the transfer.
\item  The I/O module is enhanced to become a processor in its own right, with a
specialized instruction set tailored for I/O. The CPU directs the I/O processor
to execute an I/O program in memory. The I/O processor fetches and executes
these instructions without CPU intervention. This allows the CPU to specify a
sequence of I/O activities and to be interrupted only when the entire sequence
has been performed.
\item The I/O module has a local memory of its own and is, in fact, a computer in its
own right. With this architecture, a large set of I/O devices can be controlled,
with minimal CPU involvement. A common use for such an architecture has
been to control communication with interactive terminals. The I/O processor
takes care of most of the tasks involved in controlling the terminals.
\end{enumerate}

\subsubsection{Characteristics of I/O Channels}
	The I/O channel represents an extension of the DMA concept. An I/O channel
	has the ability to execute I/O instructions, which gives it complete control over
	I/O operations. In a computer system with such devices, the CPU does not execute
	I/O instructions. Such instructions are stored in main memory to be executed by a
	special-purpose processor in the I/O channel itself. Thus, the CPU initiates an I/O
	transfer by instructing the I/O channel to execute a program in memory. The program
	will specify the device or devices, the area or areas of memory for storage,
	priority, and actions to be taken for certain error conditions. The I/O channel follows
	these instructions and controls the data transfer

	Two types of I/O channels are common, as illustrated in Figure 7.18. A
	selector channel controls multiple high-speed devices and, at any one time, is
	dedicated to the transfer of data with one of those devices. Thus, the I/O channel
	selects one device and effects the data transfer. Each device, or a small set of
	devices, is handled by a controller, or I/O module, that is much like the I/O modules
	we have been discussing. Thus, the I/O channel serves in place of the CPU
	in controlling these I/O controllers. A multiplexor channel can handle I/O with
	multiple devices at the same time. For low-speed devices, a byte multiplexor
	accepts or transmits characters as fast as possible to multiple devices. For example,
	the resultant character stream from three devices with different rates and individual
	streams A1A2A3A4 c, B1B2B3B4 c, and C1C2C3C4 c might be A1B1C1A2C2A3B2C3A4, and so on.
	

	/*263*/



%%%%%%%%%%%%%%%%%%%%%%%%%%%%%%%%%%%%%%%%%%
% Academic Title Page
% LaTeX Template
% Version 2.0 (17/7/17)
%
% This template was downloaded from:
% http://www.LaTeXTemplates.com
%
% Original author:
% WikiBooks (LaTeX - Title Creation) with modifications by:
% Vel (vel@latextemplates.com)
%
% License:
% CC BY-NC-SA 3.0 (http://creativecommons.org/licenses/by-nc-sa/3.0/)
% 
% Instructions for using this template:
% This title page is capable of being compiled as is. This is not useful for 
% including it in another document. To do this, you have two options: 
%
% 1) Copy/paste everything between \begin{document} and \end{document} 
% starting at \begin{titlepage} and paste this into another LaTeX file where you 
% want your title page.
% OR
% 2) Remove everything outside the \begin{titlepage} and \end{titlepage}, rename
% this file and move it to the same directory as the LaTeX file you wish to add it to. 
% Then add \input{./<new filename>.tex} to your LaTeX file where you want your
% title page.
%
%%%%%%%%%%%%%%%%%%%%%%%%%%%%%%%%%%%%%%%%%

%----------------------------------------------------------------------------------------
%	PACKAGES AND OTHER DOCUMENT CONFIGURATIONS
%----------------------------------------------------------------------------------------


\documentclass[11pt]{article}
\usepackage{geometry}
\usepackage{graphicx}
\usepackage{url}
\usepackage[utf8]{inputenc} % Required for inputting international characters
\usepackage[T1]{fontenc} % Output font encoding for international characters

\usepackage{mathpazo} % Palatino font

\begin{document}
\setcounter{secnumdepth}{5}
%----------------------------------------------------------------------------------------
%	TITLE PAGE
%----------------------------------------------------------------------------------------

\begin{titlepage} % Suppresses displaying the page number on the title page and the subsequent page counts as page 1
	\newcommand{\HRule}{\rule{\linewidth}{0.5mm}} % Defines a new command for horizontal lines, change thickness here
	
	\center % Centre everything on the page
	
	%------------------------------------------------
	%	Headings
	%------------------------------------------------
	
	\textsc{\LARGE Universidad de Buenos Aires}\\[1.5cm] % Main heading such as the name of your university/college
	
	\textsc{\Large Facultad de Ingeniería}\\[0.5cm] % Major heading such as course name
	
	\textsc{\large Resumen}\\[0.5cm] % Minor heading such as course title
	
	%------------------------------------------------
	%	Title
	%------------------------------------------------
	
	\HRule\\[0.4cm]
	
	{\huge\bfseries Organización del Computador \newline 75.03 \& 95.57}\\[0.4cm] % Title of your document
	
	\HRule\\[1.5cm]
	
	%------------------------------------------------
	%	Author(s)
	%------------------------------------------------
	
	%\begin{minipage}{0.4\textwidth}
	%	\begin{flushleft}
	%		\large
	%		\textit{Autor}\\
	%		\textsc{Anzu} % Your name
	%	\end{flushleft}
	%\end{minipage}
	%~
	%\begin{minipage}{0.4\textwidth}
	%	\begin{flushright}
	%		\large
	%		\textit{Supervisor}\\
	%		\textsc{Anzu} % Supervisor's name
	%	\end{flushright}
	%\end{minipage}
	
	% If you don't want a supervisor, uncomment the two lines below and comment the code above
	{\large\textit{Autor}}\\
	\textsc{Anzu} % Your name
	
	%------------------------------------------------
	%	Date
	%------------------------------------------------
	
	\vfill\vfill\vfill % Position the date 3/4 down the remaining page
	
	{\large\today} % Date, change the \today to a set date if you want to be precise
	
	%------------------------------------------------
	%	Logo
	%------------------------------------------------
	
	%\vfill\vfill
	%\includegraphics[width=0.2\textwidth]{placeholder.jpg}\\[1cm] % Include a department/university logo - this will require the graphicx package
	 
	%----------------------------------------------------------------------------------------
	
	\vfill % Push the date up 1/4 of the remaining page
	
\end{titlepage}

%----------------------------------------------------------------------------------------

\tableofcontents

\newpage

\input{U1/part1.tex}
\input{U1/part2.tex}
\input{U1/part3.tex}

\input{U2/part1.tex}
\input{U2/part2.tex}

\input{U3/part1.tex}
\input{U3/part2.tex}

\input{U4/part1.tex}
\input{U4/part2.tex}

\input{U5/part1.tex}
\input{U5/part2.tex}
\input{U5/part3.tex}
\input{U5/part4.tex}
\input{U5/part5.tex}

\input{U6/part1.tex}
\input{U6/part2.tex}
\input{U6/part3.tex}
\input{U6/part4.tex}


%\input{U6/main.tex}

\end{document}

\end{document}

\end{document}

\end{document}